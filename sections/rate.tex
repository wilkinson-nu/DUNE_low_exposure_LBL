\section{Expected Far Detector Event Rate and Oscillation Parameters}
\label{sec:rate}

In this work, \dword{fd} event rates are calculated assuming the following nominal deployment plan, which is based on a technically limited schedule:
\begin{itemize}
    \item Start of beam run: two \dword{fd} module
    volumes for total fiducial mass of 20 kt, 1.2 MW beam
    \item After one year: add one \dword{fd} module  volume for total fiducial mass of 30 kt
    \item After three years: add one \dword{fd} module  volume for total fiducial mass of \fdfiducialmass
    \item After six years: upgrade to 2.4 MW beam
\end{itemize}
Table~\ref{tab:exposures} shows the conversion between number of years under the nominal staging plan, and  kt-MW-years, which are used to indicate the exposure in this analysis. For all studies shown in this work, a 50\%/50\% ratio of FHC to RHC data was assumed, based on studies which showed a roughly equal mix of running produced a nearly optimal \deltacp and mass ordering sensitivity. The exact details of the run plan are not included in the staging plan.

\begin{table}[htbp]
  \centering
  \begin{tabular}{cc}
    \hline
    Years & kt-MW-years \\
    \hline\hline
    7 & 336 \\
    10 & 624 \\
    15 & 1104 \\
    \hline
  \end{tabular}
  \caption{Conversion between number of years in the nominal staging plan, and kt-MW-years, the two quantities used to indicate exposure in this analysis.}
  \label{tab:exposures}
\end{table}

Event rates are calculated with the assumption of 1.1 $\times 10^{21}$ \dword{pot} per year, which assumes a combined uptime and efficiency of the \dword{fnal} accelerator complex and the \dword{lbnf} beamline of 57\%~\cite{Abi:2020evt}.

\begin{table}[htbp]
    \centering
    \begin{tabular}{lcc}
      \hline
 Parameter &    Central value & Relative uncertainty \\
\hline\hline
$\theta_{12}$ & 0.5903 & 2.3\% \\ \hline
$\theta_{23}$ (NO) & 0.866  & 4.1\% \\ 
$\theta_{23}$ (IO) & 0.869  & 4.0\% \\ \hline
$\theta_{13}$ (NO) & 0.150  & 1.5\% \\ 
$\theta_{13}$ (IO) & 0.151  & 1.5\% \\ \hline
$\Delta m^2_{21}$ & 7.39$\times10^{-5}$~eV$^2$ & 2.8\% \\ \hline
$\Delta m^2_{32}$ (NO) & 2.451$\times10^{-3}$~eV$^2$ &  1.3\% \\
$\Delta m^2_{32}$ (IO) & -2.512$\times10^{-3}$~eV$^2$ &  1.3\% \\
\hline
$\rho$ & 2.848 g cm$^{-3}$ & 2\% \\
\hline
    \end{tabular}
    \caption[Parameter values and uncertainties from a global fit to neutrino oscillation data]{Central value and relative uncertainty of neutrino oscillation parameters from a global fit~\cite{Esteban:2018azc,nufitweb} to neutrino oscillation data. The matter density is taken from Ref.~\cite{Roe:2017zdw}. Because the probability distributions are somewhat non-Gaussian (particularly for $\theta_{23}$), the relative uncertainty is computed using 1/6 of the 3$\sigma$ allowed range from the fit, rather than 1/2 of the 1$\sigma$ range. For $\theta_{23}$, $\theta_{13}$, and $\Delta m^2_{31}$, the best-fit values and uncertainties depend on whether normal mass ordering (NO) or inverted mass ordering (IO) is assumed.}
    \label{tab:oscpar_nufit}
\end{table}
