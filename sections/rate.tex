\section{Expected Far Detector Event Rate and Oscillation Parameters}
\label{sec:rate}

In this work, FD event rates are calculated assuming the following nominal deployment plan, which is based on a technically limited schedule:
\begin{itemize}
    \item Start of beam run: two FD module
    volumes for total fiducial mass of 20 kt, 1.2 MW beam
    \item After one year: add one FD module  volume for total fiducial mass of 30 kt
    \item After three years: add one FD module  volume for total fiducial mass of 40 kt
    \item After six years: upgrade to 2.4 MW beam
\end{itemize}
Table~\ref{tab:exposures} shows the conversion between number of years under the nominal staging plan, and  kt-MW-years, which are used to indicate the exposure in this analysis. For all studies shown in this work, a 50\%/50\% ratio of FHC to RHC data was assumed, based on studies which showed a roughly equal mix of running produced a nearly optimal \deltacp and mass ordering sensitivity. The exact details of the run plan are not included in the staging plan.

\begin{table}[htbp]
  \centering
  \begin{tabular}{cc}
    \hline
    Years & kt-MW-years \\
    \hline\hline
    7 & 336 \\
    10 & 624 \\
    15 & 1104 \\
    \hline
  \end{tabular}
  \caption{Conversion between number of years in the nominal staging plan, and kt-MW-years, the two quantities used to indicate exposure in this analysis.}
  \label{tab:exposures}
\end{table}

Event rates are calculated with the assumption of 1.1 $\times 10^{21}$ POT per year, which assumes a combined uptime and efficiency of the FNAL accelerator complex and the LBNF beamline of 57\%~\cite{Abi:2020evt}.
