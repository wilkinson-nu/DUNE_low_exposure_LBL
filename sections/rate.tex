\section{Expected Far Detector Event Rate and Oscillation Parameters}
\label{sec:rate}

In this work, \dword{fd} event rates are calculated assuming the following nominal deployment plan, which is based on a technically limited schedule:
\begin{itemize}
    \item Start of beam run: two \dword{fd} module
    volumes for total fiducial mass of 20 kt, 1.2 MW beam
    \item After one year: add one \dword{fd} module  volume for total fiducial mass of 30 kt
    \item After three years: add one \dword{fd} module  volume for total fiducial mass of \fdfiducialmass
    \item After six years: upgrade to 2.4 MW beam
\end{itemize}
Table~\ref{tab:exposures} shows the conversion between number of years under the nominal staging plan, and  kt-MW-years, which are used to indicate the exposure in this analysis. For all studies shown in this work, a 50\%/50\% ratio of FHC to RHC data was assumed, based on studies which showed a roughly equal mix of running produced a nearly optimal \deltacp and mass ordering sensitivity. The exact details of the run plan are not included in the staging plan.

\begin{table}[htbp]
  \centering
  \begin{tabular}{cc}
    \hline
    Years & kt-MW-years \\
    \hline\hline
    7 & 336 \\
    10 & 624 \\
    15 & 1104 \\
    \hline
  \end{tabular}
  \caption{Conversion between number of years in the nominal staging plan, and kt-MW-years, the two quantities used to indicate exposure in this analysis.}
  \label{tab:exposures}
\end{table}

Event rates are calculated with the assumption of 1.1 $\times 10^{21}$ \dword{pot} per year, which assumes a combined uptime and efficiency of the \dword{fnal} accelerator complex and the \dword{lbnf} beamline of 57\%~\cite{Abi:2020evt}.

Figures~\ref{fig:appspectra} and~\ref{fig:disspectra} show the expected rate of selected events for \nue appearance and \numu disappearance, respectively, including expected flux, cross section, and oscillation probabilities, as a function of reconstructed neutrino energy at a baseline of
\num{1285}~km. The spectra are shown for a \num{3.5}~year (staged) exposure each for \dword{fhc} and \dword{rhc} beam modes, for a total run time of seven %\num{7}
years. The rates shown are scaled to obtain different exposures. Tables~\ref{tab:apprates} and~\ref{tab:disrates} give the integrated rate for the \nue %$\nu_e$ (see defs.tex}
appearance and \numu disappearance spectra, respectively. Note that the total rates are integrated over the range of reconstructed neutrino energies used in the analysis, 0.5--10 GeV. The nominal neutrino oscillation parameters used in Figures~\ref{fig:appspectra} and~\ref{fig:disspectra} and the uncertainty on those parameters (used later in the analysis) are taken from the \dword{nufit}~\cite{Esteban:2018azc,nufitweb} global fit to neutrino data, and their values are given in Table~\ref{tab:oscpar_nufit}. See also
Refs.~\cite{deSalas:2017kay} and \cite{Capozzi:2017yic} for other recent global fits.

As can be seen in Figure~\ref{fig:appspectra}, the background to \nue appearance is composed of: (1) \dword{cc} interactions of \nue and \anue intrinsic to the beam; (2) misidentified \dword{nc} interactions;  (3) misidentified \numu and \anumu \dword{cc} interactions; and (4) $\nu_\tau$ and $\bar{\nu}_\tau$ \dword{cc} interactions in which the $\tau$'s decay leptonically into electrons/positrons. \dword{nc} and $\nu_\tau$ backgrounds emanate from interactions of higher-energy neutrinos that feed down to lower reconstructed neutrino energies due to missing energy in unreconstructed final-state neutrinos. The selected NC and \dword{cc} \numu generally include an asymmetric decay of a relatively high energy $\pi^0$ coupled with a prompt photon conversion. As can be seen in Figure~\ref{fig:disspectra}, the backgrounds to the \numu disappearance are due to wrong-sign \numu interactions, which cannot easily be distinguished in the unmagnetized \dword{dune} \dword{fd}, and NC interactions, where a pion has been misidentified as the primary muon. As expected, the \numu background in RHC is much larger than the \anumu background in FHC.

\begin{figure}[htbp]
 \includegraphics[width=0.98\linewidth]{spec_app_nu_v4_varydcp.eps}\\
 \includegraphics[width=0.98\linewidth]{spec_app_anu_v4_varydcp.eps}
 \caption{\nue and \anue appearance spectra: reconstructed energy distribution of selected \nue \dword{cc}-like events assuming 3.5 years (staged) running in the neutrino-beam mode (top) and antineutrino-beam mode (bottom), for a total of seven years (staged) exposure. Statistical uncertainties are shown on the datapoints. The plots assume normal mass ordering and include curves for $\mdeltacp = -\pi/2, 0$, and $\pi/2$.}
 \label{fig:appspectra}
\end{figure}

\begin{figure}[htbp]
\includegraphics[width=0.98\linewidth]{spec_dis_nu_no_v4.eps}\\
\includegraphics[width=0.98\linewidth]{spec_dis_anu_no_v4.eps}
\caption{\numu and \anumu disappearance spectra: reconstructed energy distribution of selected $\nu_{\mu}$ \dword{cc}-like events assuming 3.5 years (staged) running in the neutrino-beam mode (top) and antineutrino-beam mode (bottom), for a total of seven years (staged) exposure. Statistical uncertainties are shown on the datapoints. The plots assume normal mass ordering.}
\label{fig:disspectra}
\end{figure}

\begin{table}[htbp]
  \centering
  \begin{tabular}{lcccc}
    \hline
    Sample & \multicolumn{4}{c}{Expected Events} \\
    & \multicolumn{2}{c}{$\mdeltacp = 0\;$} & \multicolumn{2}{c}{$\mdeltacp =-\frac{\pi}{2}$} \\
    & NO & IO & NO & IO \\ \hline\hline
    \textbf{$\nu$ mode} & & & & \\
    Oscillated \nue & 1155 & 526 & 1395 & 707 \\
    Oscillated \anue & 19 & 33 & 14 & 28 \\
    \hline
    Total oscillated & 1174 & 559 & 1409 & 735 \\
    \hline 
    Beam $\nu_{e}+\bar{\nu}_{e}$ \dword{cc} background & 228 & 235 & 228 & 235 \\
    \dword{nc} background & 84 & 84 & 84 & 84 \\
    $\nu_{\tau}+\bar{\nu}_{\tau}$ \dword{cc} background & 36 & 36 & 35 & 36 \\
    $\nu_{\mu}+\bar{\nu}_{\mu}$ \dword{cc} background & 15 & 15 & 15 & 15 \\
    \hline
    Total background & 363 & 370 & 362 & 370 \\
    \hline\hline
    \textbf{$\bar{\nu}$ mode} & & & & \\
    Oscillated \nue & 81 & 39 & 95 & 53 \\
    Oscillated \anue & 236 & 492 & 164 & 396 \\
    \hline
    Total oscillated & 317 & 531 & 259 & 449 \\
    \hline 
    Beam $\nu_{e}+\bar{\nu}_{e}$ \dword{cc} background & 145 & 144 & 145 & 144 \\
    \dword{nc} background & 40 & 40 & 40 & 40 \\
    $\nu_{\tau}+\bar{\nu}_{\tau}$ \dword{cc} background & 22 & 22 & 22 & 22 \\
    $\nu_{\mu}+\bar{\nu}_{\mu}$ \dword{cc} background & 6 & 6 & 6 & 6 \\
    \hline 
    Total background & 216 & 215 & 216 & 215 \\
    \hline
  \end{tabular}
 \caption{\nue and \anue appearance rates: integrated rate of selected $\nu_e$ \dword{cc}-like events between 0.5 and 10.0~GeV assuming a \num{3.5}-year (staged) exposure in the neutrino-beam mode and antineutrino-beam mode.  The rates are shown for both \dword{no} and \dword{io}, and signal events are shown for both $\mdeltacp = 0$ and $\mdeltacp = -\pi/2$.}
 \label{tab:apprates}
\end{table}

\begin{table}[htbp]
  \centering
  \begin{tabular}{lcc}
    \hline
    Sample & \multicolumn{2}{c}{Expected Events} \\
    & NO & IO \\
    \hline\hline
    \textbf{$\nu$ mode} & & \\
    \numu Signal & 7235 & 7368 \\
    \hline 
    \anumu \dword{cc} background & 542 & 542 \\
    \dword{nc} background & 213 & 213 \\
    $\nu_{\tau}+\bar{\nu}_{\tau}$ \dword{cc} background & 53 & 54 \\
    $\nu_e+\bar{\nu}_e$ \dword{cc} background & 9 & 5 \\
    \hline\hline
    \textbf{$\bar{\nu}$ mode}  & & \\
    \anumu Signal & 2656 & 2633 \\
    \hline 
    \numu \dword{cc} background & 1590 & 1600 \\
    \dword{nc} background & 109 & 109 \\
    $\nu_{\tau}+\bar{\nu}_{\tau}$ \dword{cc} background & 31 & 31 \\
    $\nu_e+\bar{\nu}_e$ \dword{cc} background & 2 & 2 \\
    \hline
  \end{tabular}
  \caption{\numu and \anumu disappearance rates: integrated rate of selected $\nu_{\mu}$ \dword{cc}-like events between 0.5 and 10.0~GeV assuming a \num{3.5}-year (staged) exposure in the neutrino-beam mode and antineutrino-beam mode. The rates are shown for both \dword{no} and \dword{io}, with $\mdeltacp = 0$.}
 \label{tab:disrates}
\end{table}

\begin{table}[htbp]
    \centering
    \begin{tabular}{lcc}
      \hline
 Parameter &    Central value & Relative uncertainty \\
\hline\hline
$\theta_{12}$ & 0.5903 & 2.3\% \\ \hline
$\theta_{23}$ (NO) & 0.866  & 4.1\% \\ 
$\theta_{23}$ (IO) & 0.869  & 4.0\% \\ \hline
$\theta_{13}$ (NO) & 0.150  & 1.5\% \\ 
$\theta_{13}$ (IO) & 0.151  & 1.5\% \\ \hline
$\Delta m^2_{21}$ & 7.39$\times10^{-5}$~eV$^2$ & 2.8\% \\ \hline
$\Delta m^2_{32}$ (NO) & 2.451$\times10^{-3}$~eV$^2$ &  1.3\% \\
$\Delta m^2_{32}$ (IO) & -2.512$\times10^{-3}$~eV$^2$ &  1.3\% \\
\hline
$\rho$ & 2.848 g cm$^{-3}$ & 2\% \\
\hline
    \end{tabular}
    \caption[Parameter values and uncertainties from a global fit to neutrino oscillation data]{Central value and relative uncertainty of neutrino oscillation parameters from a global fit~\cite{Esteban:2018azc,nufitweb} to neutrino oscillation data. The matter density is taken from Ref.~\cite{Roe:2017zdw}. Because the probability distributions are somewhat non-Gaussian (particularly for $\theta_{23}$), the relative uncertainty is computed using 1/6 of the 3$\sigma$ allowed range from the fit, rather than 1/2 of the 1$\sigma$ range. For $\theta_{23}$, $\theta_{13}$, and $\Delta m^2_{31}$, the best-fit values and uncertainties depend on whether normal mass ordering (NO) or inverted mass ordering (IO) is assumed.}
    \label{tab:oscpar_nufit}
\end{table}
