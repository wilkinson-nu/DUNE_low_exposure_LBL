\section{Introduction}
\label{sec:intro}

The Deep Underground Neutrino Experiment (DUNE) is a next-generation, long-baseline neutrino oscillation experiment which will utilize high-intensity \numu and \anumu beams with peak neutrino energies of $\sim$2.5 GeV over a 1285 km baseline to carry out a detailed study of neutrino mixing. DUNE's key scientific goals are the definitive determination of the neutrino mass ordering, the definitive observation of charge-parity symmetry violation (CPV) for more than 50\% of possible true values of the charge-parity violating phase, \deltacp, and the precise measurement of other three-neutrino oscillation parameters.
These measurements will help guide theory in understanding if there are new symmetries in the neutrino sector and whether there is a relationship between the generational structure of quarks and leptons~\cite{Qian:2015waa}. Observation of CPV in neutrinos would be an important step in understanding the origin of the baryon asymmetry of the universe~\cite{Fukugita:1986hr, Davidson:2008bu}. We note that DUNE has a rich physics program beyond the three-neutrino oscillation accelerator neutrino program described here, which are discussed in other works, including beyond standard model searches~\cite{Abi:2020kei}, supernova neutrino detection~\cite{Abi:2020lpk}, and solar neutrino detection~\cite{Capozzi:2018dat}. Additional physics possibilities with DUNE are discussed in Refs.~\cite{Abi:2020evt} and~\cite{AbedAbud:2021hpb}.

Neutrino oscillation experiments have so far measured five of the neutrino mixing parameters~\cite{Capozzi:2017ipn,deSalas:2020pgw,Esteban:2020cvm}: the three mixing angles $\theta_{12}$, $\theta_{23}$, and $\theta_{13}$; and the two squared-mass differences $\Delta m^{2}_{21}$ and $|\Delta m^{2}_{31}|$, where $\Delta m^2_{ij} = m^2_{i} - m^{2}_{j}$ is the difference between the squares of the neutrino mass states in eV$^{2}$.
The neutrino mass ordering (the sign of $\Delta m^{2}_{31}$) is not currently known, though recent results show a weak preference for the normal ordering (NO) over the inverted ordering (IO)~\cite{Abe:2021gky,PhysRevD.97.072001,PhysRevLett.123.151803}.
The value of \deltacp is not well known, though neutrino oscillation data are beginning to provide some information on its value~\cite{Abe:2019vii,Abe:2021gky}.

The oscillation probability of $\;\nu^{\bracketbar}_\mu \rightarrow \nu^{\bracketbar}_e$ through matter in the standard three-flavor model and a constant matter density approximation is, to first order~\cite{Nunokawa:2007qh}:
\begin{equation}
  \begin{aligned}
    P(\;\nu^{\bracketbar}_\mu \rightarrow \nu^{\bracketbar}_e) & \simeq \sin^2 \theta_{23} \sin^2 2 \theta_{13} 
    \frac{ \sin^2(\Delta_{31} - aL)}{(\Delta_{31}-aL)^2} \Delta_{31}^2 \\
    & + \sin 2 \theta_{23} \sin 2 \theta_{13} \sin 2 \theta_{12}\frac{ \sin(\Delta_{31} - aL)}{(\Delta_{31}-aL)} \Delta_{31} \\
    &\times \frac{\sin(aL)}{(aL)} \Delta_{21} \cos (\Delta_{31} \pm \deltacp) & \\
    & + \cos^2 \theta_{23} \sin^2 2 \theta_{12} \frac {\sin^2(aL)}{(aL)^2} \Delta_{21}^2,
  \end{aligned}
  \label{eqn:appprob}
\end{equation}
where
\begin{equation*}
  a = \pm \frac{G_{\mathrm{F}}N_e}{\sqrt{2}} \approx \pm\frac{1}{3500~\mathrm{km}}\left(\frac{\rho}{3.0~\mathrm{g/cm}^{3}}\right),
\end{equation*}
$G_{\mathrm{F}}$ is the Fermi constant, $N_e$ is the number density of electrons in the Earth's crust, $\Delta_{ij} = 1.267 \Delta m^2_{ij} L/E_\nu$, $L$ is the baseline in km, and $E_\nu$ is the neutrino energy in GeV. 
Both \deltacp and $a$ terms are positive (negative) for $\numu \to \nue$ ($\anumu \to \anue$) oscillations. The matter effect asymmetry arises from the presence of electrons and absence of positrons in the Earth~\cite{Wolfenstein:1977ue,Mikheev:1986gs}.

In Ref.~\cite{Abi:2020qib}, DUNE's sensitivity to CPV and the neutrino mass ordering, as well as other oscillation parameters was explored for large exposures, showing the ultimate sensitivity of the experiment. Sophisticated studies with a detailed systematics treatment were carried out only at large exposures, along with simple studies that showed DUNE's expected sensitivity for lower exposures. In this work, we explore DUNE's sensitivity at low exposures further, with a detailed systematics treatment, and an investigation into how the run plan may be optimized to enhance sensitivity to CPV and/or mass ordering. We show that DUNE will produce world-leading results at relatively short exposures, which highlights the need for a highly performant near detector complex from the beginning of the experiment. 

The DUNE far detector (FD) will ultimately consist of four modules, each with a 10-kt fiducial mass. The neutrino beamline has an initial design intensity of 1.2 MW, with a planned upgrade to 2.4 MW. As the FD deployment schedule and beam power scenarios are both subject to change, the results shown in this work are consistently given in terms of exposure in units of kiloton-megawatt-years (ktMWyrs), which is agnostic to the exact staging scenario, but can easily be expressed in terms of experiment years for any desired scenario. For example, with two FD modules and a beam intensity of 1.2 MW, exposure would accumulate at a rate of 24 ktMWyrs per calendar year.

The analysis framwork used in this work is described in Section~\ref{sec:analysis_framework}, including a description of the flux, neutrino interaction and detector models and associated uncertainties. A study on the dependence of the sensitivity to CPV and mass ordering to the fraction of data collected in neutrino-enhanced or antineutrino-enhanced running is given in Section~\ref{sec:run_plan_opt}. A detailed study on the CPV and mass ordering sensitivities at low exposures are described in Sections~\ref{sec:cp_sens} and~\ref{sec:mh_sens}, respectively. Finally, we present our conclusions in Section~\ref{sec:conclude}.

