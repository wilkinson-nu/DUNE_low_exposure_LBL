\section{Conclusion}
\label{sec:conclude}

In this work a detailed exploration of DUNE's sensitivity to CPV and the mass ordering at low exposures has been presented. The analysis uses the same framework, flux, cross section and detector models and selections as were used in Ref.~\cite{Abi:2020qib}, which showed the ultimate DUNE sensitivity to CPV, the neutrino mass ordering and other oscillation parameters, with large statistics samples after long exposures.

The effect of operating with different run plans, involving different ratios of FHC and RHC beam modes, on the mass ordering and CPV Asimov sensitivities was explored. It was found that for low exposures, the sensitivity to both CPV and the mass ordering can be increased for certain regions of parameter space, but at a cost to the sensitivity in other regions. This sensitivity increase is in part produced by leveraging the strong $\theta_{13}$ constraint available from reactor experiments. If there is a strong reason to favor the exploration of a given region of parameter space when DUNE begins to take data, this issue should be re-visited. However, with no strong motivation to focus on a given ordering or region of \deltacp parameter space, equal FHC and RHC beam running provides a close to optimal sensitivity across all of the parameter space, so was used for the subsequent detailed sensitivity studies. The increase in sensitivity for unequal beam running is also a feature of low exposure running, and degrades the sensitivity almost uniformly across the parameter space investigated for large exposures, with and without a $\theta_{13}$ constraint applied.

The studies presented here demonstrate that a full treatment of DUNE's sensitivity at low exposures supports the conclusions made in Refs.~\cite{Abi:2020qib} and~\cite{Abi:2020evt} using Asimov studies. In particular, the median CPV sensitivity is $\approx$3$\sigma$ for $\deltacp = \pm\pi/2$ after approximately a 100 kt-MW-CY FD exposure. Variations in the expected sensitivity around the median value were also explored. Additionally, it was shown that the CPV sensitivity is not significantly degraded when Feldman-Cousins corrections are included, leading to $\approx$10\% longer exposures to reach a given significance level. Crucially, it was found that after an initial low-exposure rise, the Feldman-Cousins \dchisqcrit do not change as a function of exposure, unlike the rise with exposure which has been observed by the T2K experiment~\cite{Abe:2021gky}.

It has also been shown that strong statements on the mass ordering can be expected with very short exposures of $\approx$12 kt-MW-CY, which supports the results shown in Refs.~\cite{Abi:2020qib} and~\cite{Abi:2020evt} with a more complete treatment of the systematic uncertainty.

Although the analysis used here makes no assumptions about the FD staging scenario, and results are given as a function of exposure only, the results are dependent on having a performant ND complex from the start of the experiment. In particular, the low-exposures necessary to make world-leading statements about the mass ordering can only be given with confidence with ND samples included in the fit. Additional samples of events from detectors other than ND-LAr in the DUNE ND complex are not explicitly included in this analysis, but there is an assumption that it will be possible to control the uncertainties to the level used in the analysis, and it should be understood that that implicitly relies on having a highly capable ND.
