\section{Conclusion}
\label{sec:conclude}

In this work we have presented a detailed exploration of DUNE's sensitivity to CPV and the mass ordering at low exposures. The analysis uses the same framework, flux, cross section and detector models and selections as were used in Ref.~\cite{Abi:2020qib}, which showed the ultimate DUNE sensitivity to CPV, MO and other oscillation parameters, with large statistics samples after long exposures.

\todo{Add comment about run plan optimization...}

The studies presented here demonstrate that a full treatment of DUNE's sensitivity at low exposures supports the conclusions made in Refs.~\cite{Abi:2020qib} and~\cite{Abi:2020evt} using simple Asimov studies. In particular, the median CPV sensitivity is $\sim$3$\sigma$ for $\deltacp = \pm\pi/2$ after approximately a 100 ktMWyr FD exposure. We also explore the variations in the expected sensitivity around the median value. Additionally, we show that the CPV sensitivity is not significantly degraded when Feldman-Cousins corrections are included, leading to $\sim$10\% longer exposures to reach a given significance level. Crucially, we find that after an initial low-exposure rise, the Feldman-Cousins \dchisqcrit do not change as a function of exposure, as has been observed by the T2K experiment~\cite{Abe:2021gky}.

We have also shown that strong statements on the mass ordering can be expected with very short exposures of $\sim$12 ktMWyr, which supports the results shown in Refs.~\cite{Abi:2020qib} and~\cite{Abi:2020evt} with a more complete treatment of the systematic uncertainty.

We note that although the analysis used here makes no assumptions about the FD staging scenario, and results are given as a function of exposure only, the results are dependent on having a performant ND complex from the start of the experiment. In particular, the low-exposures necessary to make world-leading statements about the mass ordering can only be given with confidence with ND samples included in the fit. We note also that additional samples of events from other detectors in the DUNE ND complex are not explicitly included in this analysis, but there is an assumption that we will be able to control the uncertainties to the level used in the analysis, and it should be understood that that implicitly relies on having a highly capable ND.
