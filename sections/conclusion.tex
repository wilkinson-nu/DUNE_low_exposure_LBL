\section{Conclusion}
\label{sec:conclude}

The analyses presented here are based on full, end-to-end simulation, reconstruction, and event selection of \dword{fd} Monte Carlo and parameterized analysis of \dword{nd} Monte Carlo of the \dword{dune} experiment. Detailed uncertainties from flux, the neutrino interaction model, and detector effects have been included in the analysis. Sensitivity results are obtained using a sophisticated, custom fitting framework. These studies demonstrate that DUNE will be able to measure \deltacp to high precision, unequivocally determine the neutrino mass ordering, and make precise measurements of the parameters governing long-baseline neutrino oscillation.

We note that further improvements are expected once the full potential of the \dword{dune} \dword{nd} is included in the analysis. In addition to the samples used explicitly in this analysis, the \dword{lartpc} is expected to measure numerous exclusive final-state \dword{cc} channels, as well as \nue and \dword{nc} events. Additionally, neutrino-electron elastic scattering~\cite{dune_nue} and the low-$\nu$ technique~\cite{Quintas:1992yv,Yang:2000ju,Tzanov:2005kr,Adamson:2009ju,DeVan:2016rkm,Ren:2017xov} may be used to constrain the flux. Additional samples of events from other detectors in the \dword{dune} \dword{nd} complex are not explicitly included here, but there is an assumption that we will be able to control the uncertainties to the level used in the analysis, and it should be understood that that implicitly relies on having a highly capable \dword{nd}.

DUNE will be able to establish the neutrino mass ordering at the 5$\sigma$ level for 100\% of \deltacp values between two and three years. CP violation can be observed with 5$\sigma$ significance after $\sim$7 years if \deltacp = $-\pi/2$ and after $\sim$10 years for 50\% of \deltacp values. CP violation can be observed with 3$\sigma$ significance for 75\% of \deltacp values after $\sim$13 years of running. For 15 years of exposure, \deltacp resolution between five and fifteen degrees are possible, depending on the true value of \deltacp. The DUNE measurement of \sinstt{13} approaches the precision of reactor experiments for high exposure, allowing measurements that do not rely on an external \sinstt{13} constraint and facilitating a comparison between the DUNE and reactor \sinstt{13}  results, which is of interest as a potential signature for physics beyond the standard model. DUNE will have significant sensitivity to the $\theta_{23}$ octant for values of \sinst{23} less than about 0.47 and greater than about 0.55. We note that the results found are broadly consistent with those found in Ref.~\cite{Acciarri:2015uup}, using a much simpler analysis.

The measurements made by \dword{dune} will make significant contributions to completion of the standard three-flavor mixing picture, and provide invaluable inputs to theory work understanding whether there are new symmetries in the neutrino sector and the relationship between the generational structure of quarks and leptons. The observation of \dword{cpv} in neutrinos would be an important step in understanding the origin of the baryon asymmetry of the universe. The precise measurements of the three-flavor mixing parameters that \dword{dune} will provide may also yield inconsistencies that point us to physics beyond the standard three-flavor model.
