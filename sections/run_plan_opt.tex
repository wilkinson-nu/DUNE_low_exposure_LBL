\section{Run plan optimization}
\label{sec:run_plan_opt}
In previous DUNE sensitivity studies~\cite{Abi:2020qib}, equal running times in FHC and RHC were assumed, based on early sensitivity estimates for different scenarios. In this section, the dependence on the median CPV and mass ordering sensitivities are studied, for different fractions of time spent in each beam mode, using the full analysis framework described in Section~\ref{sec:analysis_framework}.
\begin{figure*}[htbp]
  \centering
  \subfloat[NO, with $\theta_{13}$-penalty] {\includegraphics[width=0.4\linewidth]{{cpv_sens_ndfd100kTMWyr_th13_asimov0_nh}.png}}
  \subfloat[IO, with $\theta_{13}$-penalty] {\includegraphics[width=0.4\linewidth]{{cpv_sens_ndfd100kTMWyr_th13_asimov0_ih}.png}}\\
  \subfloat[NO, no $\theta_{13}$-penalty]   {\includegraphics[width=0.4\linewidth]{{cpv_sens_ndfd100kTMWyr_nopen_asimov0_nh}.png}}
  \subfloat[IO, no $\theta_{13}$-penalty]   {\includegraphics[width=0.4\linewidth]{{cpv_sens_ndfd100kTMWyr_nopen_asimov0_ih}.png}}
  \caption{The Asimov CPV sensitivity as a function of the true value of \deltacp, for a total exposure of 100 ktMWyr with different fractions of FHC and RHC running, with and without a $\theta_{13}$ penalty applied in the fit. Results are shown for both true normal and inverted ordering, with the true oscillation parameter values set to the NuFit 4.0 best fit point in each ordering (see Table~\ref{tab:oscpar_nufit}).}
  \label{fig:run_opt_cpv}
\end{figure*}
Figure~\ref{fig:run_opt_cpv} shows DUNE's Asimov sensitivity to CPV for a total 100 ktMWyr far detector exposure, with different fractions of FHC and RHC running, at the NuFIT 4.0 best fit value in both NO and IO (see Table~\ref{tab:oscpar_nufit}), shown with and without a penalty on $\theta_{13}$ applied. For each point tested, all oscillation and nuisance parameters are allowed to vary, and two fits are carried out, one where the \deltacp is fixed at CP-conserving values, and another where it is allowed to vary. The difference in the best-fit $\chi^{2}$ values is calculated:
\begin{equation}
  \Delta\chi^{2} = \chi^{2}_{0,\pm\pi} - \chi^{2}_{\mathrm{CPV}},
  \label{eq:cpv_chi2}
\end{equation}
\noindent and the square root of the difference is  used as the figure of merit on the y-axis in Figure~\ref{fig:run_opt_mh}. We note that there are some caveats associated with this figure of merit, which are discussed in Section~\ref{sec:cp_sens}. A 100 ktMWyr exposure is shown as it was identified in Ref~\cite{Abi:2020qib} as the exposure at which DUNE's median CPV sensitivity exceeds 3$\sigma$ at $\deltacp = \pm\pi/2$, an important milestone in DUNE's physics program (with equal beam mode running). 

Figure~\ref{fig:run_opt_cpv} shows that with a $\theta_{13}$ penalty applied, the sensitivity to CPV can be increased in some regions of \deltacp parameter space, with more FHC than RHC running. However this degrades the sensitivity in other regions, most notably for $\deltacp > 0$ in both orderings, where the octant degeneracy starts to strongly impact the results. For regions of phase space where the octant degeneracy does not affect the result (e.g., $\sin^{2}\theta_{23} \approx 0.5$), there is no degredation in the sensitivity, and enhanced FHC running increases the sensitivity for all values of \deltacp. Increasing the fraction of RHC decreases the sensitivity for the entire \deltacp range when the $\theta_{13}$ penalty is applied, relative to equal beam mode running, which can be understood as being due to the lower statistics of the \anue sample (see Figure~\ref{fig:appspectra}). Although for low exposures, DUNE will not have a strong constraint on $\theta_{13}$, so the main analysis will include a $\theta_{13}$ penalty, it is instructive to look at the results without the penalty applied. In the no penalty case, the sensitivity is severely degraded (as expected) for 100\% running in either beam mode.

\begin{figure*}[htbp]
  \centering
  \subfloat[NO, with $\theta_{13}$-penalty]  {\includegraphics[width=0.4\linewidth]{{mh_sens_ndfd24kTMWyr_th13_asimov0_nh}.png}}
  \subfloat[IO, with $\theta_{13}$-penalty]  {\includegraphics[width=0.4\linewidth]{{mh_sens_ndfd24kTMWyr_th13_asimov0_ih}.png}}\\
  \subfloat[NO, no $\theta_{13}$-penalty]    {\includegraphics[width=0.4\linewidth]{{mh_sens_ndfd24kTMWyr_nopen_asimov0_nh}.png}}
  \subfloat[IO, no $\theta_{13}$-penalty]    {\includegraphics[width=0.4\linewidth]{{mh_sens_ndfd24kTMWyr_nopen_asimov0_ih}.png}}
  \caption{The Asimov mass ordering sensitivity as a function of the true value of \deltacp, for a total exposure of 24 ktMWyr with different fractions of FHC and RHC running, with and without a $\theta_{13}$ penalty applied in the fit. Results are shown for both true normal and inverted ordering, with the true oscillation parameter values set to the NuFIT 4.0 best fit point in each ordering (see Table~\ref{tab:oscpar_nufit}).}
  \label{fig:run_opt_mh}
\end{figure*}
Figure~\ref{fig:run_opt_mh} shows DUNE's Asimov sensitivity to the mass ordering for a total 24 ktMWyr far detector exposure, with different fractions of FHC and RHC running, and the same four true oscillation parameter sets. For each point tested, all oscillation and nuisance parameters are allowed to vary, and two fits are carried out, one using each ordering. The difference in the best-fit $\chi^{2}$ values is calculated:
\begin{equation}
  \Delta\chi^{2} = \chi^{2}_{\mathrm{IO}} - \chi^{2}_{\mathrm{NO}},
  \label{eq:mh_chi2}
\end{equation}
\noindent and the square root of the difference is used as the figure of merit on the y-axis in Figure~\ref{fig:run_opt_mh}. We note that there are some caveats associated with this figure of merit, which are discussed in Section~\ref{sec:mh_sens}. A 24 ktMWyr exposure is used in Figure~\ref{fig:run_opt_mh} as it is around the exposure at which DUNE's median mass ordering sensitivity exceeds 5$\sigma$ for some vales of \deltacp~\cite{Abi:2020qib}.

 It is clear from Figure~\ref{fig:run_opt_mh} that the mass ordering sensitivity has a strong dependence on the fraction of running in each beam mode, and as in the CPV case, the effect is very different with and without a $\theta_{13}$ penalty applied. In normal ordering with the $\theta_{13}$ penalty applied, the sensitivity increases significantly with increasing FHC running, with a full 1$\sigma$ increase in the sensitivity between equal beam running and 100\% FHC for most values of \deltacp values. Conversely, in inverted ordering with the $\theta_{13}$ penalty applied, 100\% FHC running would degrade the sensitivity by $\geq$1$\sigma$ for all values of \deltacp at the NuFIT 4.0 best fit point. Overall the sensitivity in inverted ordering prefers a more equal split between the beam modes. It is clear that 100\% RHC running gives poor sensitivity for all values tested. It is also clear by comparison with the no penalty case, that the increased sensitivity with enhanced FHC running is entirely due to the $\theta_{13}$ penalty, and without it, a more equal split in beam running would be favored.

\begin{figure*}[htbp]
  \centering
  \subfloat[CPV, with $\theta_{13}$-penalty] {\includegraphics[width=0.4\linewidth]{{cpv_sens_ndfd334kTMWyr_th13_asimov0_nh}.png}}
  \subfloat[CPV, no $\theta_{13}$-penalty]   {\includegraphics[width=0.4\linewidth]{{cpv_sens_ndfd334kTMWyr_nopen_asimov0_nh}.png}}\\
  \subfloat[MO, with $\theta_{13}$-penalty]  {\includegraphics[width=0.4\linewidth]{{mh_sens_ndfd334kTMWyr_th13_asimov0_nh}.png}}
  \subfloat[MO, no $\theta_{13}$-penalty]    {\includegraphics[width=0.4\linewidth]{{mh_sens_ndfd334kTMWyr_nopen_asimov0_nh}.png}}
  \caption{The Asimov CPV and mass ordering sensitivities as a function of the true value of \deltacp, for a total exposure of 334 ktMWyr with different fractions of FHC and RHC running, with and without a $\theta_{13}$ penalty applied in the fit. Results are shown for both true normal ordering only, with the true oscillation parameter values set to the NuFIT 4.0 NO best fit point (see Table~\ref{tab:oscpar_nufit}).}
  \label{fig:run_opt_334ktmwyr}
\end{figure*}

For comparison, Figure~\ref{fig:run_opt_334ktmwyr} shows the Asimov CPV and mass ordering sensitivities, with and without a $\theta_{13}$ penalty applied, for true normal ordering only, for a large exposure of 334 ktMWyrs, with different fractions of FHC and RHC running. At large exposures, running with strongly enhanced FHC running no longer improves the sensitivity over equal beam mode running, with or without the $\theta_{!3}$ penalty applied, for either CPV or mass ordering determination. 

Overall, the sensitivity to CPV and the mass ordering is dependent on the division of running time between FHC and RHC, but a choice that increases the sensitivity in some region of parameter space can severely decrease the sensitivity in other regions. If there is strong reason to favor, for example, normal over inverted hierarchy when DUNE starts to take data, Figure~\ref{fig:run_opt_mh} shows that this could be more rapidly verified by running with more FHC data than RHC data, as a $\theta_{13}$ penalty will be used in the main low exposure analysis. However, if this choice is wrong, this might delay the results. Clearly this is an important consideration which should be revisited shortly before DUNE begins to collect data. Similarly, the CPV sensitivity shown in Figure~\ref{fig:run_opt_cpv} might be optimized if there is a strong reason to favor gaining sensitivity in a region greater than or less than \deltacp, at a cost to the other peak. But, it is clear from Figures~\ref{fig:run_opt_cpv} and~\ref{fig:run_opt_mh} that equal running in FHC and RHC gives a close to optimal sensitivity across all of the parameter space, and as such is a reasonable {\it a priori} choice of run plan for studies of the DUNE sensitivity. Additionally, it is clear from Figure~\ref{fig:run_opt_334ktmwyr} that the improvement in the sensitivity with unequal beam running is a feature at low exposures, but not at high exposures, particularly because at high exposures when DUNE is able to constrain all the oscillation parameters with precision~\cite{Abi:2020qib}, there is a stronger motivation to run a DUNE-only analysis, without a $\theta_{13}$ penalty.


