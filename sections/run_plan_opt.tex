\section{Run plan optimization}
\label{sec:run_plan_opt}
In previous DUNE sensitivity studies~\cite{Abi:2020qib}, equal running times in FHC and RHC were assumed, based on early sensitivity estimates for different scenarios. In this section, the dependence of the median CPV and mass ordering significances are studied, for different fractions of time spent in each beam mode, using the full analysis framework described in Section~\ref{sec:analysis_framework}.
\begin{figure*}[htbp]
  \centering
  \subfloat[NO, with $\theta_{13}$-penalty] {\includegraphics[width=0.4\linewidth]{{cpv_sens_ndfd100kTMWyr_th13_asimov0_nh}.pdf}}
  \subfloat[IO, with $\theta_{13}$-penalty] {\includegraphics[width=0.4\linewidth]{{cpv_sens_ndfd100kTMWyr_th13_asimov0_ih}.pdf}}\\
  \subfloat[NO, no $\theta_{13}$-penalty]   {\includegraphics[width=0.4\linewidth]{{cpv_sens_ndfd100kTMWyr_nopen_asimov0_nh}.pdf}}
  \subfloat[IO, no $\theta_{13}$-penalty]   {\includegraphics[width=0.4\linewidth]{{cpv_sens_ndfd100kTMWyr_nopen_asimov0_ih}.pdf}}
  \caption{The Asimov CPV sensitivity as a function of the true value of \deltacp, for a total exposure of 100 kt-MW-yr with different fractions of FHC and RHC running, with and without a $\theta_{13}$ penalty applied in the fit. Results are shown for both true normal and inverted ordering, with the true oscillation parameter values set to the NuFit 4.0 best fit point in each ordering (see Table~\ref{tab:oscpar_nufit}).}
  \label{fig:run_opt_cpv}
\end{figure*}

Figure~\ref{fig:run_opt_cpv} shows DUNE's Asimov sensitivity to CPV for a total 100 kt-MW-yr far detector exposure, with different fractions of FHC and RHC running, at the NuFIT 4.0 best fit value in both NO and IO (see Table~\ref{tab:oscpar_nufit}), shown with and without a penalty on $\theta_{13}$ applied. For each point tested, all oscillation and nuisance parameters are allowed to vary, and three fits are carried out, two where \deltacp is set to the CP-conserving values $\deltacp = 0$ and $\deltacp = \pm\pi$, the minimum of which is the CP-conserving best-fit value, and another where \deltacp is allowed to vary. The difference in the best-fit $\chi^{2}$ values is calculated:
\begin{linenomath*}
  \begin{equation}
    \dchisqCPV = \min\left\{\chi^{2}_{\deltacp = 0},\chi^{2}_{\deltacp = \pm\pi}\right\} - \chi^{2}_{\mathrm{CPV}},
    \label{eq:cpv_chi2}
  \end{equation}
\end{linenomath*}
\noindent and the square root of the difference is used as the figure of merit on the y-axis in Figure~\ref{fig:run_opt_cpv}. There are some caveats associated with this figure of merit, which are discussed in Section~\ref{sec:cp_sens}. A 100 kt-MW-yr exposure is shown as it was identified in Ref~\cite{Abi:2020qib} as the exposure at which DUNE's median CPV significance exceeds 3$\sigma$ at $\deltacp = \pm\pi/2$, an important milestone in DUNE's physics program (with equal beam mode running). 

Figure~\ref{fig:run_opt_cpv} shows that when the reactor constraint on $\theta_{13}$ is included, the sensitivity to CPV can be increased in some regions of \deltacp parameter space with more FHC than RHC running. However, this degrades the sensitivity in other regions, most notably for $\deltacp > 0$ regardless of the true mass ordering. This is due to a degeneracy between \deltacp and the octant of $\sinstt{23}$ because both parameters impact the rate of \nue appearance. The degeneracy is resolved by including antineutrino data; the octant of $\sinstt{23}$ affects the rate of \nue and \anue appearance in the same way, but the effect of \deltacp is reversed for antineutrinos.

For regions of phase space where the octant degeneracy does not affect the result (e.g., $\sin^{2}\theta_{23} \approx 0.5$), there is no degradation in the sensitivity, and enhanced FHC running increases the sensitivity for all values of \deltacp. Increasing the fraction of RHC decreases the sensitivity for the entire \deltacp range when the reactor $\theta_{13}$ constraint is included, relative to equal beam mode running. This is due to the lower statistics of the \anue sample (see Figure~\ref{fig:appspectra}) because of the reduced antineutrino flux and cross section. For short exposures, DUNE will not have a competitive independent measurement of $\theta_{13}$, so the main analysis will include the reactor $\theta_{13}$ constraint. Nonetheless, it is instructive to look at the results without the penalty applied. In this case, the sensitivity is severely degraded (as expected) for 100\% running in either beam mode.


\begin{figure*}[htbp]
  \centering
  \subfloat[NO, with $\theta_{13}$-penalty]  {\includegraphics[width=0.4\linewidth]{{mh_sens_ndfd24kTMWyr_th13_asimov0_nh}.pdf}}
  \subfloat[IO, with $\theta_{13}$-penalty]  {\includegraphics[width=0.4\linewidth]{{mh_sens_ndfd24kTMWyr_th13_asimov0_ih}.pdf}}\\
  \subfloat[NO, no $\theta_{13}$-penalty]    {\includegraphics[width=0.4\linewidth]{{mh_sens_ndfd24kTMWyr_nopen_asimov0_nh}.pdf}}
  \subfloat[IO, no $\theta_{13}$-penalty]    {\includegraphics[width=0.4\linewidth]{{mh_sens_ndfd24kTMWyr_nopen_asimov0_ih}.pdf}}
  \caption{The Asimov mass ordering sensitivity as a function of the true value of \deltacp, for a total exposure of 24 kt-MW-yr with different fractions of FHC and RHC running, with and without a $\theta_{13}$ penalty applied in the fit. Results are shown for both true normal and inverted ordering, with the true oscillation parameter values set to the NuFIT 4.0 best fit point in each ordering (see Table~\ref{tab:oscpar_nufit}).}
  \label{fig:run_opt_mh}
\end{figure*}
Figure~\ref{fig:run_opt_mh} shows DUNE's Asimov sensitivity to the mass ordering for a total 24 kt-MW-yr far detector exposure, with different fractions of FHC and RHC running, and the same four true oscillation parameter sets. A 24 kt-MW-yr exposure is used in Figure~\ref{fig:run_opt_mh} as it is around the exposure at which DUNE's median mass ordering significance exceeds 5$\sigma$ for some vales of \deltacp~\cite{Abi:2020qib}. For each point tested, all oscillation and nuisance parameters are allowed to vary, and two fits are carried out, one using each ordering. The difference in the best-fit $\chi^{2}$ values is calculated:
\begin{linenomath*}
  \begin{equation}
    \dchisqMO = \chi^{2}_{\mathrm{IO}} - \chi^{2}_{\mathrm{NO}},
    \label{eq:mh_chi2}
  \end{equation}
\end{linenomath*}
\noindent and the square root of the difference is used as the figure of merit on the y-axis in Figure~\ref{fig:run_opt_mh}. There are some caveats associated with this figure of merit, which are discussed in Section~\ref{sec:mh_sens}. 

It is clear from Figure~\ref{fig:run_opt_mh} that the mass ordering sensitivity has a strong dependence on the fraction of running in each beam mode. As in the CPV case, the effect is very different with and without the reactor $\theta_{13}$ constraint included. If the true ordering is normal and the reactor $\theta_{13}$ penalty is applied, the sensitivity increases significantly with increasing FHC running, with a full 1$\sigma$ increase in the sensitivity between equal beam running and 100\% FHC for most values of \deltacp. Conversely, if the ordering is inverted, 100\% FHC running would degrade the sensitivity by $\geq$1$\sigma$ for all values of \deltacp at the NuFIT 4.0 best fit point. Overall, the sensitivity to the inverted ordering is improved by a more equal split between the beam modes. It is clear that 100\% RHC running gives poor sensitivity for all values tested. 

Without the reactor $\theta_{13}$ constraint, the greatest sensitivity is obtained with close to an equal split of FHC and RHC running, and the sensitivity is significantly reduced with 100\% FHC running. This is because of a degeneracy between the effect of $\theta_{13}$ and the mass ordering on the rate of \nue appearance in FHC mode. If the mass ordering is normal, the \nue rate in FHC will be enhanced; without the reactor constraint, this excess can be accommodated by increasing the value of $\theta_{13}$.

\begin{figure*}[htbp]
  \centering
  \subfloat[CPV, with $\theta_{13}$-penalty] {\includegraphics[width=0.4\linewidth]{{cpv_sens_ndfd336kTMWyr_th13_asimov0_nh}.pdf}}
  \subfloat[CPV, no $\theta_{13}$-penalty]   {\includegraphics[width=0.4\linewidth]{{cpv_sens_ndfd336kTMWyr_nopen_asimov0_nh}.pdf}}\\
  \subfloat[MO, with $\theta_{13}$-penalty]  {\includegraphics[width=0.4\linewidth]{{mh_sens_ndfd336kTMWyr_th13_asimov0_nh}.pdf}}
  \subfloat[MO, no $\theta_{13}$-penalty]    {\includegraphics[width=0.4\linewidth]{{mh_sens_ndfd336kTMWyr_nopen_asimov0_nh}.pdf}}
  \caption{The Asimov CPV and mass ordering sensitivities as a function of the true value of \deltacp, for a total exposure of 336 kt-MW-yr with different fractions of FHC and RHC running, with and without a $\theta_{13}$ penalty applied in the fit. Results are shown for both true normal ordering only, with the true oscillation parameter values set to the NuFIT 4.0 NO best fit point (see Table~\ref{tab:oscpar_nufit}).}
  \label{fig:run_opt_336ktmwyr}
\end{figure*}

For comparison, Figure~\ref{fig:run_opt_336ktmwyr} shows the Asimov CPV and mass ordering sensitivities, with and without the reactor $\theta_{13}$ constraint included, for true normal ordering only, for a large exposure of 336 kt-MW-yr, with different fractions of FHC and RHC running. At large exposures, running with strongly enhanced FHC no longer improves the sensitivity over equal beam mode running, with or without the $\theta_{13}$ penalty applied, for either CPV or mass ordering determination. This can be understood because the enhancement to the statistics that enhanced FHC brings is no longer as important to the sensitivity, and DUNE is able to place a constraint on the value of $\theta_{13}$ with its own data.

Overall, the sensitivity to CPV and the mass ordering is dependent on the division of running time between FHC and RHC, but a choice that increases the sensitivity in some region of parameter space can severely decrease the sensitivity in other regions. If there is strong reason to favor, for example, normal over inverted ordering when DUNE starts to take data, Figure~\ref{fig:run_opt_mh} shows that this could be more rapidly verified by running with more FHC data than RHC data, as the reactor $\theta_{13}$ constraint will be used in the main low exposure analysis. However, if this choice is wrong, this might cause DUNE to take longer to reach the same significance. Clearly this is an important consideration which should be revisited shortly before DUNE begins to collect data. Similarly, the CPV sensitivity shown in Figure~\ref{fig:run_opt_cpv} might be optimized if there is a strong reason to favor gaining sensitivity for $\deltacp > 0$ or $\deltacp < 0$, at a cost of reducing the sensitivity to CPV if \deltacp has the other sign. But, it is clear from Figures~\ref{fig:run_opt_cpv} and~\ref{fig:run_opt_mh} that equal running in FHC and RHC gives a close to optimal sensitivity across all of the parameter space, and as such is a reasonable {\it a priori} choice of run plan for studies of the DUNE sensitivity. Additionally, it is clear from Figure~\ref{fig:run_opt_336ktmwyr} that the improvement in the sensitivity with unequal beam running is a feature at low exposures, but not at high exposures, particularly because at high exposures when DUNE is able to constrain all the oscillation parameters with precision~\cite{Abi:2020qib}, there is a stronger motivation to run a DUNE-only analysis, without relying on the reactor $\theta_{13}$ measurement.


