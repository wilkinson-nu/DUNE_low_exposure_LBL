\section{Run plan optimization}
\label{sec:run_plan_opt}
In previous DUNE sensitivity studies~\cite{Abi:2020qib}, equal running times in FHC and RHC were assumed, based on early sensitivity estimates for different scenarios. In this section, the dependence on the median mass ordering and CPV sensitivity is studied, for different fractions of time spent in each beam mode, using the full analysis framework described in Section~\ref{sec:analysis_framework}.
\begin{figure*}[htbp]
  \centering
  \subfloat[NO, NuFit 4.0]                 {\includegraphics[width=0.4\linewidth]{{cpv_sens_ndfd100kTMWyr_th13_asimov0_nh}.png}}
  \subfloat[NO, $\sin^{2}\theta_{23} = 0.5$] {\includegraphics[width=0.4\linewidth]{{cpv_sens_ndfd100kTMWyr_th13_ssth23:0.5_nh}.png}}\\
  \subfloat[IO, NuFit 4.0]                 {\includegraphics[width=0.4\linewidth]{{cpv_sens_ndfd100kTMWyr_th13_asimov0_ih}.png}}
  \subfloat[IO, $\sin^{2}\theta_{23} = 0.5$] {\includegraphics[width=0.4\linewidth]{{cpv_sens_ndfd100kTMWyr_th13_ssth23:0.5_ih}.png}}
  \caption{The Asimov CPV sensitivity as a function of the true value of \deltacp, for a total exposure of 100 ktMWyr with different fractions of FHC and RHC running, with a $\theta_{13}$ penalty applied in the fit. Results are shown for both true normal and inverted ordering, and with the true oscillation parameter values set to the NuFit 4.0 best fit point, or the NuFit 4.0 best fit with $\sin^{2}\theta_{23} = 0.5$.}
  \label{fig:run_opt_cpv}
\end{figure*}
Figure~\ref{fig:run_opt_cpv} shows DUNE's Asimov sensitivity to CPV for a total 100 ktMWyr far detector exposure, with different fractions of FHC and RHC running, and for four different true oscillation parameter sets. A 100 ktMWyr exposure is shown as it was identified in Ref~\cite{Abi:2020qib} as the exposure at which DUNE's median CPV sensitivity exceeds 3$\sigma$ at $\deltacp = \pm\pi/2$, an important milestone in DUNE's physics program. The true oscillation parameter values are taken from the NuFIT 4.0 best fit values (see Table~\ref{tab:oscpar_nufit}) in both normal and inverted ordering, and NuFIT 4.0 with $\sin^{2}\theta_{23}$ set to 0.5 to explore the effect of removing the octant degeneracy, again in both normal and inverted hierarchies. It is clear from Figure~\ref{fig:run_opt_cpv} that running purely in either beam mode significantly reduces the sensitivity, which is unsurprising, and that the effect is much more marked for the NuFIT 4.0 best fit point in both normal and inverted hierarchy, where the octant degeneracy can play a role, than at $\sin^{2}\theta_{23} = 0.5$, where it cannot. Increasing the fraction of RHC decreases the sensitivity for the entire \deltacp range for all tested true oscillation points compared to equal running in both beam modes, which can be understood as being due to the lower statistics of the \anue sample (see Figure~\ref{fig:appspectra}). Increasing the fraction of FHC running can improve the sensitivity for some regions of parameter space, but degrades the sensitivity in others, most notably for $\deltacp > 0$ in NO with the NuFIT 4.0 best fit values as the test point, where the octant degeneracy starts to impact the results. Conversely, if there was strong reason to expect NO and $\deltacp < 0$, the senstivity could be increased slightly with respect to the equal mode splitting case, but not by a great deal.

\begin{figure*}[htbp]
  \centering
  \subfloat[NO, NuFit 4.0]                 {\includegraphics[width=0.4\linewidth]{{mh_sens_ndfd24kTMWyr_th13_asimov0_nh}.png}}
  \subfloat[NO, $\sin^{2}\theta_{23} = 0.5$] {\includegraphics[width=0.4\linewidth]{{mh_sens_ndfd24kTMWyr_th13_ssth23:0.5_nh}.png}}\\
  \subfloat[IO, NuFit 4.0]                 {\includegraphics[width=0.4\linewidth]{{mh_sens_ndfd24kTMWyr_th13_asimov0_ih}.png}}
  \subfloat[IO, $\sin^{2}\theta_{23} = 0.5$] {\includegraphics[width=0.4\linewidth]{{mh_sens_ndfd24kTMWyr_th13_ssth23:0.5_ih}.png}}
  \caption{The Asimov mass ordering sensitivity as a function of the true value of \deltacp, for a total exposure of 24 ktMWyr with different fractions of FHC and RHC running, with a $\theta_{13}$ penalty applied in the fit. Results are shown for both true normal and inverted ordering, and with the true oscillation parameter values set to the NuFit 4.0 best fit point, or the NuFit 4.0 best fit with $\sin^{2}\theta_{23} = 0.5$.}
  \label{fig:run_opt_mh}
\end{figure*}
Figure~\ref{fig:run_opt_mh} shows DUNE's Asimov sensitivity to the mass ordering for a total 24 ktMWyr far detector exposure, with different fractions of FHC and RHC running, and the same four true oscillation parameter sets. For each point tested, all oscillation and nuisance parameters are allowed to vary, and  two fits are carried out, one using each ordering. The difference in the best-fit $\chi^{2}$ values is calculated:
\begin{equation}
  \Delta\chi^{2} = \chi^{2}_{\mathrm{IO}} - \chi^{2}_{\mathrm{NO}},
  \label{eq:mh_chi2}
\end{equation}
\noindent and the square root of the difference is used as the figure of merit on the y-axis in Figure~\ref{fig:run_opt_mh}. We note that there are some caveats associated with this figure of merit, which are discussed in Section~\ref{sec:mh_sens}.

A 24 ktMWyr exposure is used in Figure~\ref{fig:run_opt_mh} as it is around the exposure at which DUNE's median mass ordering sensitivity exceeds 5$\sigma$ for some vales of \deltacp~\cite{Abi:2020qib}. It is clear from Figure~\ref{fig:run_opt_mh} that the mass ordering sensitivity has a strong dependence on the fraction of running in each beam mode, although that effect is more marked in normal than inverted ordering. In normal ordering, for both true oscillation parameter points shown, the sensitivity increases with increasing FHC running, with a full 1$\sigma$ increase in the sensitivity between equal beam running and 100\% FHC for most values of \deltacp values. Conversely, in inverted ordering, 100\% FHC running would degrade the sensitivity by $\geq$1$\sigma$ for all values of \deltacp at the NuFIT 4.0 best fit point, although not for $\sin^{2}\theta_{23} = 0.5$. Overall the sensitivity in inverted ordering prefers a more equal split between the beam modes, although whether more FHC or more RHC running would increase the sensitivity depends on the region of parameter space. It is clear that 100\% RHC running gives poor sensitivity for all values tested.

Overall, the sensitivity to CPV and the mass ordering is dependent on the division of running time between FHC and RHC, but a choice that increases the sensitivity in some region of parameter space can severely decrease the sensitivity in other regions. If there is strong reason to favor, for example, normal over inverted hierarchy when DUNE starts to take data, Figure~\ref{fig:run_opt_mh} shows that this could be more rapidly verified by running with more FHC data than RHC data. However, if this choice is wrong, this might delay the results. Clearly this is an important consideration which should be revisited shortly before DUNE begins to collect data. Similarly, the CPV sensitivity shown in Figure~\ref{fig:run_opt_cpv} might be optimized if there is a strong reason to favor gaining sensitivity in a region greater than or less than \deltacp, at a cost to the other peak. But, it is clear from Figures~\ref{fig:run_opt_cpv} and~\ref{fig:run_opt_mh} that equal running in FHC and RHC gives a close to optimal sensitivity across all of the parameter space, and as such is a reasonable {\it a priori} choice of run plan for studies of the DUNE sensitivity.

We note that the increased sensitivity for 100\% FHC running observed in some regions of phase space in Figures~\ref{fig:run_opt_cpv} and~\ref{fig:run_opt_mh} is due to the inclusion of a penalty term on $\theta_{13}$, which removes one set of degenerate solutions from the fit. Similar behaviour is seen at higher exposures for both the mass ordering and CPV sensitivities.
