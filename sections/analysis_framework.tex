\section{Analysis framework}\label{sec:analysis_framework}

This work uses the flux, neutrino interaction and detector model described in detail in Ref.~\cite{Abi:2020qib}, implemented in the CAFAna framework~\cite{CAFAna}. This section provides an overview of the key analysis features. Further details on all aspects can be found in Ref.~\cite{Abi:2020qib}.

%% Flux in brief
\subsection{Neutrino flux}
The neutrino flux is generated with G4LBNF~\cite{Aliaga:2016oaz,Abi:2020evt}, using the LBNF optimized beam design~\cite{Abi:2020evt}. Flux uncertainties are due to uncertainties in hadrons produced off the target and uncertainties in the design parameters of the beamline, such as horn currents and horn and target positioning (commonly called ``focusing uncertainties'')~\cite{Abi:2020evt}. They are evaluated using current measurements of hadron production and LBNF estimates of alignment tolerances, giving flux uncertainties of approximately 8\% at the first oscillation maximum which are highly correlated across energy bins and neutrino flavors. A many universe approach is taken to evaluate the flux prediction for variations of the systematics propagated through the full beamline simulation, to build a flux covariance matrix as a function of neutrino energy, beam mode, detector, and neutrino species. To reduce the number of parameters used in the fit, the covariance matrix is then diagonalized, and each principal component is treated as an uncorrelated nuisance parameter. It was found that only the first $\sim$30 principal components have a significant effect in the analysis and need to be included. Note that the unoscillated fluxes at the ND and FD are similar, and differences between them are well understood.

%% Neutrino interactions in brief
\subsection{Neutrino interaction model}
The interaction model used is based on GENIE v2.12.10~\cite{Andreopoulos:2009rq,Andreopoulos:2015wxa}. The nuclear model used to describe the initial state of nucleons in the nucleus is the Bodek-Ritchie global Fermi gas model~\cite{BodekRitchie}, which includes empirical modifications to the nucleon momentum distribution to account for short-range correlation effects. The quasi-elastic model uses the Llewelyn-Smith formalism~\cite{llewelyn-smith} with a simple dipole axial form factor, and BBBA05 vector form factors~\cite{bbba05}. Nuclear screening effects and uncertainties are included based on the T2K 2017/8 parameterization~\cite{Abe:2018wpn} of the Valencia group's~\cite{nieves1,nieves2} Random Phase Approximation model. The Valencia model of the multi-nucleon, $2p2h$, contribution to the cross section~\cite{nieves1,nieves2} is used, as described in Ref.~\cite{Schwehr:2016pvn}. Both MINERvA~\cite{Rodrigues:2015hik} and NOvA~\cite{NOvA:2018gge} have shown that this model underpredicts observed event rates on carbon. Modifications to the model are constructed to produce agreement with MINERvA CC-inclusive data~\cite{Rodrigues:2015hik}, and are used to introduce additional uncertainties on the $2p2h$ contribution, with energy dependent uncertainties, and extra freedom between neutrinos and antineutrinos. GENIE uses a modified version of the Rein-Sehgal (R-S) model for pion production~\cite{Rein:1980wg}. We include a data-driven modification to the GENIE model based on reanalyzed neutrino--deuterium bubble chamber data~\cite{Wilkinson:2014yfa,Rodrigues:2016xjj}. The Deep Inelastic Scattering (DIS) model implemented in GENIE uses the Bodek-Yang parametrization~\cite{Bodek:2002ps}, using GRV98 parton distribution functions~\cite{Gluck:1998xa}. Hadronization is described by the AKGY model~\cite{Yang:2009zx}, which uses the KNO scaling model~\cite{Koba:1972ng} for invariant masses $W \leq 2.3$ GeV and PYTHIA6~\cite{Sjostrand:2006za} for invariant masses $W \geq 3$ GeV, with a smooth transition between the two for intermediate invariant masses. We include additional uncertainties developed by the NOvA experiment~\cite{nova_2018} to describe their resonance to DIS transition region data. The final state interaction model and uncertainties available in GENIE are retained~\cite{Dytman:2011zz,Dytman:2015taa,intranuke_2009}.

The cross sections include terms proportional to the lepton mass, which are significant at low energies where quasielastic processes dominate. Some of the form factors in these terms have significant uncertainties in the nuclear environment. We adopt separate (and anticorrelated) uncertainties on the cross section ratio $\sigma_\mu/\sigma_e$ for neutrinos and antineutrinos from Ref.~\cite{Day:2012gb}. Additionally, some electron-neutrino interactions occur at four-momentum transfers where muon-neutrino interactions are kinematically forbidden, and so cannot be constrained by muon-neutrino cross-section measurements. To reflect this, a 100\% uncertainty is applied in the phase space present for $\nu_e$ but absent for $\nu_\mu$.

%% ND detector in brief
\subsection{Near detector (ND) simulation and reconstruction}
The ND hall will be located 574 m downstream of the proton target and 60 m underground. The baseline design for the DUNE ND system is fully described in Ref.~\cite{AbedAbud:2021hpb}, and consists of a LArTPC with a downstream magnetized multi-purpose detector (MPD), and an on-axis beam monitor. Additionally, the LArTPC and MPD will be able to move perpendicular to the beam axis, to take measurements at a number of off-axis angles, in the DUNE-PRISM concept. The LArTPC is a modular detector based on the ArgonCube design~\cite{argoncube_loi, Dwyer:2018phu, arclight}, with a total active LAr volume of $105$~m$^{3}$ (a LAr mass of 147 tons). The MPD implementation in the analysis consists of a cylindrical high-pressure gaseous argon TPC (GArTPC), surrounded by a granular, high-performance electromagnetic calorimeter (ECal), which sits immediately downstream of the LAr cryostat~\cite{Emberger:2018pgr}. The entire MPD sits inside a magnetic field, which allows the MPD to precisely measure the momentum and discriminate the sign of particles passing through it.

Neutrino interactions are simulated in the active volume of the LArTPC. The propagation of neutrino interaction products through the LArTPC and MPD detector volumes is simulated using Geant4~\cite{Agostinelli:2002hh}. As pattern recognition and reconstruction software has not yet been fully developed for the ND, this analysis uses a parameterized reconstruction based on the Geant4 simulated energy deposits in active detector volumes.

Only interactions originating in the LAr are considered in this analysis, with a fiducial volume which excludes 50 cm from the sides and upstream edge, and 150 cm from the downstream edge of the active region. Most muons with kinetic energy greater than $\sim$1 GeV exit the LAr. Energetic forward-going muons pass through the ECal and into the GArTPC, where their momentum and charge are reconstructed by curvature. Events with muons that exit the LAr and do not match to the GArTPC are rejected, as the muon momentum is not reconstructed. Muons that stop in the LAr or ECal are reconstructed by range, but their charge cannot be determined event by event. For antineutrino enhanced beam running, there is a significant wrong-sign flux contribution, which is suppressed by a factor of four by requiring a Michel electron at the end of these stopped track.

True muons and charged pions are evaluated as potential muon candidates. Tracks are classified as muon-like if their track length is at least 1 m, and their mean energy deposit per centimeter is less than 3 MeV. The minimum length requirement imposes an effective threshold on true muons of about 200 MeV kinetic energy, but greatly suppresses potential NC backgrounds with low-energy, non-interacting charged pions. Charged-current events are required to have exactly one muon, and if the charge is reconstructed, it must be of the appropriate charge. Hadronic energy in the ND is reconstructed by summing all charge deposits in the LAr active volume that are not associated with the muon. Events where the hadronic energy is badly reconstructed due to charged particles exiting the detector, a veto region is defined as the outer 30 cm of the active volume on all sides. Events with more than 30 MeV total energy deposited in the veto region are rejected. Neutron energy is typically not observed, resulting in poor reconstruction of events with energetic neutrons. The reconstructed neutrino energy is the sum of the reconstructed hadronic energy and the reconstructed muon energy.

%% Move to the bit with the samples
The oscillation analysis presented here includes samples of $\nu_{\mu}$ and $\bar{\nu}_{\mu}$ charged-current interactions originating in the LAr portion of the ND. These samples are binned in two dimensions as a function of reconstructed neutrino energy and inelasticity, $y_{\mathrm{rec}} = 1 - E^{\mathrm{rec}}_{\mu}/E^{\mathrm{rec}}_{\nu}$, where $E^{\mathrm{rec}}_{\mu}$ and $E^{\mathrm{rec}}_{\nu}$ are the reconstructed muon and neutrino energies, respectively. Backgrounds to $\nu^{\bracketbar}_{\mu}$-CC arise from NC $\pi^{\pm}$ production where the pion leaves a long track and does not shower. Muons below about 400 MeV kinetic energy have a significant background from charged pions, so these CC events are excluded from the selected sample. Wrong-sign contamination in the beam is an additional background, particularly at low reconstructed neutrino energies in RHC.

%% FD description
\subsection{Far detector (FD) simulation and reconstruction}
The 40-kt DUNE FD consists of four separate LArTPC detector modules, each with a FV of at least 10 kt, installed $\sim$1.5 km underground at the Sanford Underground Research Facility (SURF)~\cite{Abi:2018dnh}. The technologies to be deployed for the four modules and their order of construction is still under investigation, so in this analysis, only the single-phase design~\cite{Abi:2020loh} is used. A full simulation chain has been developed, from the generation of neutrino events in a Geant4 model of the FD geometry, to efficiencies and reconstructed neutrino energy estimators of all samples used in the sensitivity analysis.

The electronics response to the ionization electrons and scintillation light is simulated in the wire planes and photon detector respectively. Algorithms have been developed to remove the impact of the LArTPC electric field and electronics response from the raw detector signal, to identify hits, and to cluster hits that may be grouped together due to proximity in time and space. Clusters from different wire planes are matched to form high-level objects such as tracks and showers. The energy of the incoming neutrino in CC events is estimated by adding the lepton and hadronic energies reconstructed from these high-level objects using the Pandora toolkit~\cite{Marshall:2015rfa,Acciarri:2017hat}. The neutrino energy for $\nu_{\mu}$-CC ($\nu_{e}$-CC) events is estimated by the sum of the energy of the longest reconstructed track (highest energy reconstructed electromagnetic shower) and the hadronic energy. For both event types, the hadronic energy is estimated from the charge of reconstructed hits that are not in the primary track or shower, and corrections are applied to each hit charge for recombination and the electron lifetime. For $\nu_{\mu}$-CC events, the energy of the longest track is estimated by range (multiple Coulomb scattering) if it is contained (exiting). For 0.5--4 GeV neutrino energies, the observed neutrino energy resolution is $\sim$15--20\%. The muon energy resolution is 4\% for contained tracks and 18\% for exiting tracks. The electron energy resolution is approximately $4\% \oplus 9\%/\sqrt{E}$. The hadronic energy resolution is 34\%. Event classification is carried out through image recognition techniques using a convolutional neural network~\cite{cvn_paper} which classifies events as $\nu_{\mu}$-CC, $\nu_{e}$-CC, $\nu_{\tau}$-CC, and NC. The \nue and \numu efficiencies in both beam modes all exceed 90\% in the neutrino flux peak.

%% Detector systematics
\subsection{Detector systematics}
Detector effects impact the event selection efficiency as well as the reconstruction of neutrino energy and inelasticity (the variable used in the oscillation fit). The main sources of detector systematic uncertainties are limitations of the expected calibration and modeling of particles in the detector. Important differences between the ND and FD LArTPC design, size, detector environment, and calibration strategy, lead to uncertainties that do not fully correlate between the two detectors. The degree of correlation is under active study, but in this analysis they are treated as being completely uncorrelated.

An uncertainty on the overall energy scale is included in the analysis presented here, as well as particle response uncertainties that are separate and uncorrelated between four species: muons, charged hadrons, neutrons, and electromagnetic showers. In the ND, muons reconstructed by range in LAr and by curvature in the MPD are treated separately. For each class of particle, uncertainties on the energy scale are introduced as a function of the reconstructed neutrino energy, as well as proportional to its square-root and one divided by the square-root. Additional uncertainties on the energy resolution is also included for each particle class. Note that the parameters produce a shift to the kinematic variables in an event, as opposed to simply assigning a weight to each simulated event. The scale of the uncertainties is motivated by what has been achieved in recent experiments, including calorimetric based approaches (NOvA, MINERvA) and LArTPCs (LArIAT, MicroBooNE, ArgoNeut).

In addition to impacting energy reconstruction, the E-field model also affects the definition of the FD fiducial volume, which is sensitive to electron drift. An additional 1\% uncertainty is therefore included on the total fiducial mass, which is conservatively treated as uncorrelated between the $\nu_{\mu}$ and $\nu_{e}$ samples due to the potential distortion caused by large electromagnetic showers in the electron sample.

The FD is sufficiently large that acceptance is not expected to vary significantly as a function of event kinematics. However, the ND acceptance does vary as a function of both muon and hadronic kinematics due to various containment criteria. Uncertainties are evaluated on the muon and hadron acceptance at the ND based on the change in the acceptance as a function of muon kinematics and true hadronic energy.

%The response of the detector to neutrons is a source of active study and will couple strongly to detector technology. The validation of neutron interactions in LAr will continue to be characterized by dedicated measurements (e.g., CAPTAIN~\cite{Berns:2013usa,Bhandari:2019rat}) and the LAr program (e.g., ArgoNeut~\cite{Acciarri:2018myr}).  However, the association of the identification of a neutron scatter or capture to the neutron's true energy has not been demonstrated, and significant reconstruction issues exist, so a large uncertainty (20\%) is assigned comparable to the observations made by MINERvA~\cite{Elkins:2019vmy} assuming they are attributed entirely to the detector model. Selection of photon candidates from $\pi^0$ is also a significant reconstruction challenge, but a recent measurement from MicroBooNE indicates this is possible and the reconstructed $\pi^0$ invariant mass has an uncertainty of 5\%, although with some bias~\cite{Adams:2018sgn}.

%% Sensitivity method
\subsection{Sensitivity Methods}
Systematics are implemented in the analysis using one-dimensional response functions for each analysis bin, and oscillation weights are calculated exactly, in fine (50 MeV) bins of true neutrino energy. For a given set of inputs, flux, oscillation parameters, cross sections, detector energy response matrices, and detector efficiency, an expected event rate can be produced. Minimization is performed using the {\sc minuit}~\cite{James:1994vla} package.

Oscillation sensitivities are obtained by simultaneously fitting the \numutonumu, $\bar{\nu}_\mu \rightarrow \bar{\nu}_\mu$ (Figure~\ref{fig:disspectra}), \numutonue, and $\bar{\nu}_\mu \rightarrow \bar{\nu}_e$ (Figure~\ref{fig:appspectra}) FD spectra along with the $\nu_{\mu}$ FHC and $\bar{\nu}_{\mu}$ RHC samples from the ND (Figure~\ref{fig:nd_samples}). In the studies, all oscillation parameters shown in Table~\ref{tab:oscpar_nufit} are allowed to vary. Gaussian penalty terms (taken from Table~\ref{tab:oscpar_nufit}) are applied to $\theta_{12}$ and \dm{12} and the matter density, $\rho$, of the Earth along the DUNE baseline~\cite{Roe:2017zdw}. Unless otherwise stated, studies presented here include a Gaussian penalty term on $\theta_{13}$ (also taken from Table~\ref{tab:oscpar_nufit}), which is precisely measured by experiments sensitive to reactor antineutrino disappearance~\cite{Abe:2014bwa,Adey:2018zwh,Bak:2018ydk}. The remaining parameters, \sinst{23}, $\Delta m^{2}_{32}$, and \deltacp are allowed to vary freely, with no penalty term. Note that the penalty terms are treated as uncorrelated with each other, or other parameters, which is a simplification. In particular, the reactor experiments that drive the constraint on $\theta_{13}$ in the NuFIT 4.0 analysis are also sensitive to \dm{32}, so the constraint on $\theta_{13}$ should be correlated with \dm{32}. We do not expect this to have a significant impact on the fits, and this effect only matters for those results with the $\theta_{13}$ Gaussian penalty term included.

Flux, cross section, and FD detector parameters are allowed to vary in the fit, but constrained by a penalty term proportional to the pre-fit uncertainty. ND detector parameters are not allowed to vary in the fit, but their effect is included via a covariance matrix based on the shape difference between ND prediction and the ``data'' (which comes from the simulation in this sensitivity study). The covariance matrix is constructed with a throwing technique. For each ``throw'', all ND energy scale, resolution, and acceptance parameters are simultaneously thrown according to their respective uncertainties, and the modified prediction is produced by varying the relevant quantities away from the nominal prediction according to the thrown parameter values. The bin-to-bin covariance is determined by comparing the resulting spectra with the nominal prediction, in the same binning as is used in the oscillation sensitivity analysis. This choice protects against overconstraining that could occur given the limitations of the parameterized ND reconstruction described in Section~\ref{sec:nd} taken together with the high statistical power at the ND, but is also a simplification.

The compatibility of a particular oscillation hypothesis with both ND and FD data is evaluated using a negative log-likelihood ratio, which converges to a $\chi^{2}$ at high-statistics~\cite{Tanabashi:2018oca}:
\begin{equation}
\begin{aligned}
  \chi^2(\vec{\vartheta}, \vec{x}) &= -2\log\mathcal{L}(\vec{\vartheta}, \vec{x}) \\
  &= 2\sum_i^{N_{\rm bins}}\left[ M_i(\vec{\vartheta}, \vec{x})-D_i+D_i\ln\left({D_i\over M_i(\vec{\vartheta}, \vec{x})}\right) \right] \\
  &+ \sum_{j}^{N_{\mathrm{systs}}}\left[ \frac{\Delta x_{j}}{\sigma_{j}} \right]^{2} \\
  &+ \sum^{N^{\mathrm{ND}}_{\mathrm{bins}}}_{k}\sum^{N^{\mathrm{ND}}_{\mathrm{bins}}}_{l} \left(M_k(\vec{x})-D_k \right) V^{-1}_{kl}\left(M_l(\vec{x})-D_l \right),
\end{aligned}
\label{eq:chisq}
\end{equation}
where $\vec{\vartheta}$ and $\vec{x}$ are the vector of oscillation parameter and nuisance parameter values respectively; $M_i(\vec{\vartheta}, \vec{x})$ and $D_{i}$ are the MC expectation and fake data in the $i$th reconstructed bin (summed over all selected samples), with the oscillation parameters neglected for the ND; $\Delta x_{j}$ and $\sigma_{j}$ are the difference between the nominal and current value, and the prior uncertainty on the $j$th nuisance parameter with uncertainties evaluated and described in Sections~\ref{sec:flux},~\ref{sec:nuint} and~\ref{sec:syst}; and $V_{kl}$ is the covariance matrix between ND bins described previously. In order to avoid falling into a false minimum, all fits are repeated for four different \deltacp values (-$\pi$, -$\pi$/2, 0, $\pi$/2), both mass orderings, and in both octants, and the lowest $\chi^{2}$ value is taken as the minimum.

\begin{table}
  \centering
  \begin{tabular}{lcc}
    \hline
    Parameter & Prior & Range\\ \hline\hline
    $\sin^{2}\theta_{23}$ & Uniform & [0.4; 0.6] \\
    $|\Delta m^{2}_{32}|$ ($\times 10^{-3}$ eV$^{2}$) & Uniform & $|[2.3;2.7]|$ \\
    \deltacp ($\pi$) & Uniform & [-1;1] \\
    $\theta_{13}$ & Gaussian & NuFIT 4.0 \\
    & Uniform & [0.13; 0.2] \\
    \hline
  \end{tabular}
  \caption{Treatment of the oscillation parameters for the simulated data set studies. Note that for some studies $\theta_{13}$ has a Gaussian penalty term applied based on the NuFIT 4.0 value, and for others it is thrown uniformly within a range determined from the NuFIT 4.0 3$\sigma$ allowed range.}
  \label{table:OA_throw}
\end{table}
Two approaches are used for the sensitivity studies presented in this work. First, Asimov studies~\cite{Cowan:2010js} are carried out in which the fake (Asimov) dataset is the same as the nominal MC. In these, the true value of all systematic uncertainties and oscillation parameters except those of interest (which are fixed at a test point) remain unchanged, and can vary in the fit, but are constrained by their pre-fit uncertainty. Second, studies are performed where many statistical and systematic throws are made according to their pre-fit Gaussian uncertainties, and fits of all parameters are carried out for each throw. A distribution of post-fit values is built up for the parameter of interest. In these, the expected resolution for oscillation parameters is determined from the spread in best-fit values obtained from an ensemble of throws that vary according to both the statistical and systematic uncertainties.  For each throw, the true value of each nuisance parameter is chosen randomly from a distribution determined by the {\it a priori} uncertainty on the parameter. For some studies, oscillation parameters are also randomly chosen as described in Table~\ref{table:OA_throw}. Poisson fluctuations are then applied to all analysis bins, based on the mean event count for each bin after the systematic adjustments have been applied. For each throw in the ensemble, the test statistic is minimized, and the best-fit value of all parameters is determined. The median throw and central 68\% of throws derived from these ensembles are shown.
