\section{Analyis framework}\label{sec:analysis_framework}

This work uses the flux, neutrino interaction and detector model described in Ref.~\cite{Abi:2020qib}, implemented in the CAFAna framework. In this section, an overview of the analysis will be given. Additional details can be found in Ref.~\cite{Abi:2020qib}.

%% Flux in brief
The neutrino flux is generated with G4LBNF~\cite{Aliaga:2016oaz,Abi:2020evt}, using the \dword{lbnf} optimized beam design~\cite{Abi:2020evt}. Flux uncertainties are due to uncertainties in hadrons produced off the target and uncertainties in the design parameters of the beamline, such as horn currents and horn and target positioning (commonly called ``focusing uncertainties'')~\cite{Abi:2020evt}. They are evaluated using current measurements of hadron production and \dword{lbnf} estimates of alignment tolerances, giving flux uncertainties of approximately 8\% at the first oscillation maximum which are highly correlated across energy bins and neutrino flavors. A many universe approach is taken to evaluate the flux prediction for variations of the systematics propagated through the full beamline simulation, to build a flux covariance matrix as a function of neutrino energy, beam mode, detector, and neutrino species. To reduce the number of parameters used in the fit, the covariance matrix is then diagonalized, and each principal component is treated as an uncorrelated nuisance parameter. It was found that only the first $\sim$30 principal components have a significant effect in the analysis and need to be included. Note that the unoscillated fluxes at the ND and FD are similar, and differences between them are well understood.

%% Neutrino interactions in brief
The interaction simulation used is based on GENIE v2.12.10~\cite{Andreopoulos:2009rq,Andreopoulos:2015wxa}. The nuclear model used to describe the initial state of nucleons in the nucleus is the Bodek-Ritchie global Fermi gas model~\cite{BodekRitchie} which includes empirical modifications to the nucleon momentum distribution to account for short-range correlation effects. The quasi-elastic model uses the LLewelyn-Smith formalism~\cite{llewelyn-smith} with a simple dipole axial form factor, and BBBA05 vector form factors~\cite{bbba05}. Nuclear screening effects and uncertainties are included based on the T2K 2017/8 parameterization~\cite{Abe:2018wpn} of the Valencia group's~\cite{nieves1,nieves2} Random Phase Approximation model. The Valencia model of the multi-nucleon, $2p2h$, contribution to the cross section~\cite{nieves1,nieves2} is used, with the implementation in \dword{genie} as described in Ref.~\cite{Schwehr:2016pvn}. Both MINERvA~\cite{Rodrigues:2015hik} and NOvA~\cite{NOvA:2018gge} have shown that this model underpredicts observed event rates on carbon. Modifications to the model are constructed to produce agreement with MINERvA CC-inclusive data~\cite{Rodrigues:2015hik}, and are used to construct additional uncertainties on the $2p2h$ contribution, including energy dependent uncertainties, and extra freedom between neutrinos and antineutrinos. \dword{genie} uses a modified version of the Rein-Sehgal (R-S) model for pion production~\cite{Rein:1980wg}. We include a data-driven modification to the \dword{genie} model based on reanalyzed neutrino--deuterium bubble chamber data~\cite{Wilkinson:2014yfa,Rodrigues:2016xjj}. The \dword{dis} model implemented in \dword{genie} uses the Bodek-Yang parametrization~\cite{Bodek:2002ps}, using GRV98 parton distribution functions~\cite{Gluck:1998xa}. Hadronization is described by the AKGY model~\cite{Yang:2009zx}, which uses the KNO scaling model~\cite{Koba:1972ng} for invariant masses $W \leq 2.3$ GeV and PYTHIA6~\cite{Sjostrand:2006za} for invariant masses $W \geq 3$ GeV, with a smooth transition between the two for intermediate invariant masses. We include additional uncertainties developed by the NOvA experiment~\cite{nova_2018} to describe their resonance to \dword{dis} transition region data. We additionally include the large number of final state uncertainties on the final state cascade model provided by GENIE~\cite{Dytman:2011zz,Dytman:2015taa,intranuke_2009}.

The cross sections include terms proportional to the lepton mass, which are significant at low energies where quasielastic processes dominate. Some of the form factors in these terms have significant uncertainties in the nuclear environment. We adopt separate (and anticorrelated) uncertainties on the cross section ratio $\sigma_\mu/\sigma_e$ for neutrinos and antineutrinos from Ref.~\cite{Day:2012gb}. Additionally, some electron-neutrino interactions occur at four-momentum transfers where muon-neutrino interactions are kinematically forbidden, and so cannot be constrained by muon-neutrino cross-section measurements. A 100\% uncertainty is applied in the phase space present for $\nu_e$ but absent for $\nu_\mu$.

%% Need to dump these all out into this single file when suitably cut down
%\input{sections/nd}
%\input{sections/fd}
%\input{sections/syst}
%\input{sections/methods}
