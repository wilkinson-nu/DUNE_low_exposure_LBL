\section{Analysis framework}
\label{sec:analysis_framework}
This work uses the flux, neutrino interaction and detector model described in detail in Ref.~\cite{Abi:2020qib}, implemented in the CAFAna framework~\cite{CAFAna}. This section provides an overview of the key analysis features. Further details on all aspects can be found in Ref.~\cite{Abi:2020qib}.

%% Flux in brief
\subsection{Neutrino flux}
DUNE will operate with two different beam modes, which depend on the polarity of the electromagnetic horns used to focus secondary particles produced after protons from the primary beamline interact in the target. Forward horn current (FHC) corresponds to neutrino-enhanced running, and reverse-horn current (RHC) corresponds to antineutrino-enhanced running. The neutrino flux prediction is generated with G4LBNF~\cite{Aliaga:2016oaz,Abi:2020evt}, using the LBNF optimized beam design~\cite{Abi:2020evt}. Flux uncertainties are due to uncertainties in the production rates and kinematic distributions of hadrons produced in the target and uncertainties in the parameters of the beamline, such as horn currents and horn and target positioning (``focusing uncertainties'')~\cite{Abi:2020evt}. They are evaluated using current measurements of hadron production and LBNF estimates of alignment tolerances, giving flux uncertainties of approximately 8\% at the first oscillation maximum, which are highly correlated across energy bins and neutrino flavors. A ``toy throw'' approach is taken to evaluate the flux prediction for variations (``throws'') of the systematics propagated through the full beamline simulation, to build a flux covariance matrix as a function of neutrino energy, beam mode, detector, and neutrino species. To reduce the number of parameters used in the fit, the covariance matrix is then diagonalized, and each principal component is treated as an uncorrelated nuisance parameter. Only the first $\sim$30 principal components (out of 108) were found to have a significant effect in the analysis and are included. Note that the unoscillated fluxes at the ND and FD are similar, and the differences between them are understood at the percent level.

%% Neutrino interactions in brief
\subsection{Neutrino interaction model}
The interaction model used is based on GENIE v2.12.10~\cite{Andreopoulos:2009rq,Andreopoulos:2015wxa}, although we note that the combination of models used is much closer to some of the physics tunes available with GENIE v3.00.06, including a number of uncertainties beyond those provided by either GENIE version. The nuclear model which describes the initial state of nucleons in the nucleus is the Bodek-Ritchie global Fermi gas model~\cite{BodekRitchie}, which includes empirical modifications to the nucleon momentum distribution to account for short-range correlation effects. The quasi-elastic model uses the Llewelyn-Smith formalism~\cite{llewelyn-smith} with a simple dipole axial form factor, and BBBA05 vector form factors~\cite{bbba05}. Nuclear screening effects and uncertainties are included based on the T2K 2017/8 parameterization~\cite{Abe:2018wpn} of the Valencia group's~\cite{nieves1,nieves2} Random Phase Approximation model. The Valencia model of the multi-nucleon, $2p2h$, contribution to the cross section~\cite{nieves1,nieves2} is used, as described in Ref.~\cite{Schwehr:2016pvn}. Both MINERvA~\cite{Rodrigues:2015hik} and NOvA~\cite{NOvA:2018gge} have shown that this model underpredicts observed event rates on carbon. Modifications to the model are constructed to produce agreement with MINERvA CC-inclusive data~\cite{Rodrigues:2015hik}, which are used in the analysis to introduce additional uncertainties on the $2p2h$ contribution, with energy dependent uncertainties, and extra freedom between neutrinos and antineutrinos. GENIE uses a modified version of the Rein-Sehgal (R-S) model for pion production~\cite{Rein:1980wg}. We include a data-driven modification to the GENIE model based on reanalyzed neutrino--deuterium bubble chamber data~\cite{Wilkinson:2014yfa,Rodrigues:2016xjj}. The Deep Inelastic Scattering (DIS) model implemented in GENIE uses the Bodek-Yang parametrization~\cite{Bodek:2002ps}, using GRV98 parton distribution functions~\cite{Gluck:1998xa}. Hadronization is described by the AKGY model~\cite{Yang:2009zx}, which uses the KNO scaling model~\cite{Koba:1972ng} for invariant masses $W \leq 2.3$ GeV and PYTHIA6~\cite{Sjostrand:2006za} for invariant masses $W \geq 3$ GeV, with a smooth transition between the two for intermediate invariant masses. We include additional uncertainties developed by the NOvA experiment~\cite{nova_2018} to describe their resonance to DIS transition region data. The final state interaction model and uncertainties available in GENIE are retained~\cite{Dytman:2011zz,Dytman:2015taa,intranuke_2009}.

The cross sections include terms proportional to the lepton mass, which are significant at low energies where quasielastic processes dominate. Some of the form factors in these terms have significant uncertainties in the nuclear environment. We adopt separate (and anticorrelated) uncertainties on the cross section ratio $\sigma_\mu/\sigma_e$ for neutrinos and antineutrinos from Ref.~\cite{Day:2012gb}. Additionally, some $\nu_e$ charged-current (CC) interactions occur at four-momentum transfers where $\nu_\mu$ CC interactions are kinematically forbidden, and so cannot be constrained by $\nu_\mu$ cross-section measurements. To reflect this, a 100\% uncertainty is applied in the phase space present for $\nu_e$ but absent for $\nu_\mu$.

%% ND detector in brief
\subsection{Near detector (ND) simulation and reconstruction}
The ND hall will be located 574 m downstream of the proton target and $\sim$60 m underground. The reference design for the DUNE ND system is fully described in Ref.~\cite{AbedAbud:2021hpb}, and consists of a LArTPC (ND-LAr), a magnetized high-pressure gaseous argon TPC (ND-GAr), and an on-axis beam monitor (SAND). Additionally, ND-LAr and ND-GAr are designed to move perpendicular to the beam axis in order to take data at various off-axis angles (DUNE-PRISM). ND-LAr is a modular detector based on the ArgonCube design~\cite{argoncube_loi, Dwyer:2018phu, arclight}, with a total active LAr volume of $105$~m$^{3}$ (a LAr mass of 147 tons). ND-GAr is implemented in this analysis as a cylindrical TPC filled with a 90/10 mixture of argon and CH$_{4}$ at 10 bar, surrounded by a granular, high-performance electromagnetic calorimeter (ECal). ND-GAr sits immediately downstream of the LAr cryostat and serves as a muon spectrometer for ND-LAr~\cite{Emberger:2018pgr}.

Neutrino interactions are simulated in the active volume of ND-LAr. The propagation of neutrino interaction products through the ND-LAr and ND-GAr detector volumes is simulated using a Geant4-based program~\cite{Agostinelli:2002hh}. As pattern recognition and reconstruction software has not yet been fully developed for the ND, this analysis uses a parameterized reconstruction based on the Geant4 simulated energy deposits in active detector volumes.

Only interactions originating in the LAr are considered in this analysis, with a fiducial volume (FV) which excludes 50 cm from the sides and upstream edge, and 150 cm from the downstream edge of the active region, containing a total fiducial mass of $\sim$50 t. Most muons with kinetic energies greater than 1 GeV exit ND-LAr. Energetic forward-going muons pass into ND-GAr, where they either stop in the ECal or continue into the GAr region, where their momentum and charge are reconstructed by curvature. Events with muons that exit the LAr active volume and do not match to a track in ND-GAr are rejected, as the muon momentum is not well reconstructed. Muons that stop in the LAr or ECal have their muons reconstructed by range, but their charge cannot be determined event by event. For antineutrino-enhanced beam running, a Michel electron is required at the end of these stopped tracks to suppress the wrong sign $\mu^-$ by a factor of four.

True muons and charged pions are evaluated as potential muon candidates. Tracks are classified as muon-like if their length is at least 1 m, and their mean energy deposit is less than 3 MeV/cm. The minimum length requirement imposes an effective threshold on the true muon kinetic energy of about 200 MeV, but greatly suppresses potential NC backgrounds with low-energy, non-interacting charged pions. Charged-current events are required to have exactly one muon, and if the charge is reconstructed, it must be of the appropriate charge. Hadronic energy in the ND is reconstructed by summing all charge deposits in the LAr active volume that are not associated with the muon. To remove events where the hadronic energy is badly reconstructed due to charged particles exiting the detector, a veto region is defined as the outer 30 cm of the active volume on all sides, and events with more than 30 MeV total energy deposited in the veto region are rejected. Neutron energy is typically not observed, resulting in poor energy reconstruction of events with energetic neutrons. The reconstructed neutrino energy is the sum of the reconstructed hadronic energy and the reconstructed muon energy.

%% FD description
\subsection{Far detector (FD) simulation and reconstruction}
The DUNE FD consists of four separate LArTPC detector modules, each with a FV of at least 10 kt, installed $\sim$1.5 km underground at the Sanford Underground Research Facility (SURF)~\cite{Abi:2018dnh}. The technologies to be deployed for the four modules and their order of construction are still under investigation, so in this analysis, only the single-phase design with a horizontal drift~\cite{Abi:2020loh} is used. A full simulation chain has been developed, including the generation of neutrino events in a Geant4 model of the FD geometry and simulation of the electronics readout. A reconstruction package has been developed to calculate efficiencies and reconstructed neutrino energy estimators for all samples used in the analysis. The simulation uses a smaller FD single-phase volume of 6.65 $\times$ 12.0 $\times$ 13.9 m (corresponding to 1 $\times$ 2 $\times$ 6 anode readout planes), which is then scaled up to the full 13.3 $\times$ 12.0 $\times$ 57.5 m active volume of a full single-phase module.

The electronics response to the ionization electrons and scintillation light is simulated in the wire planes and photon detectors, respectively. Algorithms are applied to remove the impact of the LArTPC electric field and electronics response from the raw detector signal, to identify hits, and to cluster hits that may be grouped together due to proximity in time and space. Clusters from different wire planes are matched to form high-level objects such as tracks and showers. The energy of the incoming neutrino in CC events is estimated by adding the lepton and hadronic energies reconstructed from these high-level objects using the Pandora toolkit~\cite{Marshall:2015rfa,Acciarri:2017hat}. The neutrino energy for $\nu^{\bracketbar}_{\mu}$-CC ($\;\nu^{\bracketbar}_{e}$-CC) events is estimated by the sum of the energy of the longest reconstructed track (highest energy reconstructed electromagnetic shower) and the hadronic energy. For both event types, the hadronic energy is estimated from the charge of reconstructed hits that are not in the primary track or shower, and corrections are applied to each hit charge for recombination and the electron lifetime. For $\nu^{\bracketbar}_{\mu}$-CC events, the energy of the longest track is estimated by range if the track is contained or by multiple Coulomb scattering if it is exiting. For 0.5--4 GeV neutrino energies, the observed neutrino energy resolution is $\sim$15--20\%. The muon energy resolution is 4\% for contained tracks and 18\% for exiting tracks. The electron energy resolution is approximately $4\% \oplus 9\%/\sqrt{E}$. The hadronic energy resolution is 34\%. Event classification is carried out through image recognition techniques using a convolutional neural network~\cite{cvn_paper} which classifies events as $\nu^{\bracketbar}_{\mu}$-CC, $\nu^{\bracketbar}_{e}$-CC, $\nu^{\bracketbar}_{\tau}$-CC, and NC. The $\nue^{\bracketbar}$ and $\numu^{\bracketbar}$ efficiencies in both beam modes exceed 90\% in the flux peak.

%% Detector systematics
\subsection{Detector systematics}
Detector effects impact the event selection efficiency as well as the reconstruction of neutrino energy and inelasticity (the variables used in the oscillation fits). The main sources of detector systematic uncertainties are limitations of the expected calibration and modeling of particles in the detector. Important differences between the ND and FD LArTPC design, size, detector environment, and calibration strategy, lead to uncertainties that do not fully correlate between the two detectors. The degree of correlation is under active study, but in this analysis they are treated as being completely uncorrelated.

An uncertainty on the overall energy scale is included in the analysis presented here, as well as particle response uncertainties that are separate and uncorrelated between four particle classes: muons, charged hadrons, neutrons, and electromagnetic showers. In the ND, muons reconstructed by range in LAr and by curvature in the ND-GAr are treated separately. For each class of particle, uncertainties on the energy scale are introduced as a function of the reconstructed particle energy, $E$, with a constant term, a term proportional to $\sqrt{E}$, and a term proportional to $1/\sqrt{E}$. Additional uncertainties on the energy resolution are also included for each particle class. Note that the parameters produce a shift to the kinematic variables in an event, as opposed to simply assigning a weight to each simulated event. The scale of the uncertainties is motivated by what has been achieved in recent experiments, including calorimetric based approaches (NOvA, MINERvA) and LArTPCs (LArIAT, MicroBooNE, ArgoNeut).

In addition to impacting energy reconstruction, the E-field model also affects the definition of the FD FV, which is sensitive to electron drift. An additional 1\% uncertainty is therefore included on the total fiducial mass, which is conservatively treated as uncorrelated between the $\nu^{\bracketbar}_{\mu}$ and $\nu^{\bracketbar}_{e}$ samples due to the potential distortion caused by large electromagnetic showers in the electron sample.

The FD is sufficiently large that acceptance is not expected to vary significantly as a function of event kinematics. However, the ND acceptance does vary as a function of both muon and hadronic kinematics due to various containment criteria. Uncertainties are evaluated on the muon and hadron acceptance at the ND based on the change in the acceptance as a function of muon kinematics and true hadronic energy.

%% Sensitivity method
\subsection{Sensitivity Methods}
Systematics are implemented in the analysis using one-dimensional response functions for each analysis bin, and oscillation weights are calculated exactly, in fine (50 MeV) bins of true neutrino energy. For a given set of inputs, flux, oscillation parameters, cross sections, detector energy response matrices, and detector efficiency, an expected event rate can be produced. Minimization is performed using the {\sc minuit}~\cite{James:1994vla} package.

\begin{table}[htbp]
    \centering
    \begin{tabular}{lcc}
      \hline
      Parameter &    Central value & Relative uncertainty \\
      \hline\hline
      $\theta_{12}$ & 0.5903 & 2.3\% \\ 
      $\theta_{23}$ (NO) & 0.866  & 4.1\% \\ 
      $\theta_{23}$ (IO) & 0.869  & 4.0\% \\
      $\theta_{13}$ (NO) & 0.150  & 1.5\% \\ 
      $\theta_{13}$ (IO) & 0.151  & 1.5\% \\
      $\Delta m^2_{21}$ & 7.39$\times10^{-5}$~eV$^2$ & 2.8\% \\
      $\Delta m^2_{32}$ (NO) & 2.451$\times10^{-3}$~eV$^2$ &  1.3\% \\
      $\Delta m^2_{32}$ (IO) & -2.512$\times10^{-3}$~eV$^2$ &  1.3\% \\
      $\rho$ & 2.848 g cm$^{-3}$ & 2\% \\
      \deltacp & -2.53$\pi$ & -- \\
      \hline
    \end{tabular}
    \caption{Central value and relative uncertainty of neutrino oscillation parameters from a global fit~\cite{Esteban:2018azc,nufitweb} to neutrino oscillation data. The matter density is taken from Ref.~\cite{Roe:2017zdw}. Because the probability distributions are somewhat non-Gaussian (particularly for $\theta_{23}$), the relative uncertainty is computed using 1/6 of the 3$\sigma$ allowed range from the fit, rather than 1/2 of the 1$\sigma$ range. For $\theta_{23}$, $\theta_{13}$, and $\Delta m^2_{31}$, the best-fit values and uncertainties depend on whether NO or IO is assumed. The best fit for \deltacp is used as a test point in the analysis, but no uncertainty is assigned.}
    \label{tab:oscpar_nufit}
\end{table}

Oscillation sensitivities are obtained by simultaneously fitting the \numutonumu, $\bar{\nu}_\mu \rightarrow \bar{\nu}_\mu$, \numutonue, and $\bar{\nu}_\mu \rightarrow \bar{\nu}_e$ FD spectra along with the $\nu_{\mu}$ FHC and $\bar{\nu}_{\mu}$ RHC samples from the ND. In the studies, all oscillation parameters shown in Table~\ref{tab:oscpar_nufit} are allowed to vary. Gaussian penalty terms (taken from Table~\ref{tab:oscpar_nufit}) are applied to $\theta_{12}$, \dm{12}, and the matter density, $\rho$, of the Earth along the DUNE baseline~\cite{Roe:2017zdw}. Some studies presented in this work include a Gaussian penalty term on $\theta_{13}$ (also taken from Table~\ref{tab:oscpar_nufit}), which is precisely measured by experiments sensitive to reactor antineutrino disappearance~\cite{Abrahao:2020ztg,Adey:2018zwh,Bak:2018ydk}. The remaining parameters, \sinst{23}, $\Delta m^{2}_{32}$, and \deltacp are allowed to vary freely, with no penalty term. Note that the penalty terms are treated as uncorrelated with each other, and uncorrelated with other parameters.

Flux, cross-section, and FD detector parameters are allowed to vary in the fit, but are constrained by a penalty term corresponding to the pre-fit uncertainty. ND detector parameters are not allowed to vary in the fit, but their effect is included via a covariance matrix based on the shape difference between ND prediction and the ``data'' (which comes from the simulation in this sensitivity study). The covariance matrix is constructed with a throwing technique. For each ``throw'', all ND energy scale, resolution, and acceptance parameters are simultaneously thrown according to their respective uncertainties, and the modified prediction is produced by varying the relevant quantities away from the nominal prediction according to the thrown parameter values. The bin-to-bin covariance is determined by comparing the resulting spectra with the nominal prediction, in the same binning as is used in the oscillation sensitivity analysis.

The compatibility of a particular oscillation hypothesis with both ND and FD data is evaluated using a negative log-likelihood ratio, which converges to a $\chi^{2}$ distribution at high-statistics~\cite{Tanabashi:2018oca}:
\begin{equation}
\begin{aligned}
  \chi^2(\vec{\vartheta}, \vec{x}) &= -2\log\mathcal{L}(\vec{\vartheta}, \vec{x}) \\
  &= 2\sum_i^{N_{\rm bins}}\left[ M_i(\vec{\vartheta}, \vec{x})-D_i+D_i\ln\left({D_i\over M_i(\vec{\vartheta}, \vec{x})}\right) \right] \\
  &+ \sum_{j}^{N_{\mathrm{systs}}}\left[ \frac{\Delta x_{j}}{\sigma_{j}} \right]^{2} \\
  &+ \sum^{N^{\mathrm{ND}}_{\mathrm{bins}}}_{k}\sum^{N^{\mathrm{ND}}_{\mathrm{bins}}}_{l} \left(M_k(\vec{x})-D_k \right) V^{-1}_{kl}\left(M_l(\vec{x})-D_l \right),
\end{aligned}
\label{eq:chisq}
\end{equation}
where $\vec{\vartheta}$ and $\vec{x}$ are the vector of oscillation parameter and nuisance parameter values, respectively; $M_i(\vec{\vartheta}, \vec{x})$ and $D_{i}$ are the MC expectation and fake data in the $i$th reconstructed bin (summed over all selected samples), with the oscillation parameters neglected for the ND; $\Delta x_{j}$ and $\sigma_{j}$ are the difference between the nominal and current value, and the prior uncertainty on the $j$th nuisance parameter; and $V_{kl}$ is the covariance matrix between ND bins described previously. To protect against false minima, all fits are repeated for four different \deltacp values (-$\pi$, -$\pi$/2, 0, $\pi$/2), both mass orderings, and in both \sinst{23} octants, and the lowest obtained $\chi^{2}$ value is taken as the true minimum.

\begin{table}
  \centering
  \begin{tabular}{lcc}
    \hline
    Parameter & Prior & Range\\ \hline\hline
    $\sin^{2}\theta_{23}$ & Uniform & [0.4; 0.6] \\
    $|\Delta m^{2}_{32}|$ ($\times 10^{-3}$ eV$^{2}$) & Uniform & $|[2.3;2.7]|$ \\
    \deltacp ($\pi$) & Uniform & [-1;1] \\
    $\theta_{13}$ & Gaussian & NuFIT 4.0 \\
    \hline
  \end{tabular}
  \caption{Treatment of the oscillation parameters for the simulated data set studies. Note that for some studies $\theta_{13}$ has a Gaussian penalty term applied based on the NuFIT 4.0 value, and for others it is thrown uniformly within a range determined from the NuFIT 4.0 3$\sigma$ allowed range.}
  \label{table:OA_throw}
\end{table}
Two approaches are used for the sensitivity studies presented in this work. Asimov studies~\cite{Cowan:2010js} are carried out (in Section~\ref{sec:run_plan_opt}) in which the fake (Asimov) dataset is the same as the nominal MC. In these, the true value of all systematic uncertainties and oscillation parameters are set to their nominal value (see Table~\ref{tab:oscpar_nufit}) except the parameters of interest, which are set to a test point. Then a fit is carried out in which all parameters can vary, constrained by their prior uncertainty where applicable. Toy throw studies are performed (in Sections~\ref{sec:cp_sens} and~\ref{sec:mh_sens}) where many statistical and systematic throws are made according to their pre-fit Gaussian uncertainties, and fits of all parameters are carried out for each throw. A distribution of post-fit values is built up for the parameter of interest. In these, the expected resolution for oscillation parameters is determined from the spread in best-fit values obtained from an ensemble of throws that vary according to both the statistical and systematic uncertainties.
%For each throw, the true value of each nuisance parameter is chosen randomly from a distribution determined by the {\it a priori} uncertainty on the parameter.
For some studies, oscillation parameters are also randomly chosen as described in Table~\ref{table:OA_throw}. Poisson fluctuations are then applied to all analysis bins, based on the mean event count for each bin after the systematic adjustments have been applied. For each throw in the ensemble, the test statistic is minimized, and the best-fit value of all parameters is determined. Asimov studies are computationally efficient, and for Gaussian parameters and uncertainties, give a good sense of the median sensitivity of an experiment. Toy throwing studies are computationally expensive, fully explore the parameter space, and make fewer assumptions about the behavior of parameters and uncertainties.


\subsection{Near and far detector samples and statistics}
In this work, we explore the sensitivity as a function of FD exposure and report results in terms of kt-MW-yrs, so do not assume any specific FD or beam intensity staging scenario. However, ND-LAr is assumed to be fully operational when the beam turns on, and as such the ND sample size used in the analysis will vary based on the staging scenario, and not simply on the FD exposure in kt-MW-yrs. We therefore retain the nominal staging scenario from Ref.~\cite{Abi:2020qib} for the purpose of normalizing the ND samples. In that scenario, a 7 year exposure corresponds to 336 kt-MW-yrs at the FD, and 350 ton-years at the ND, summed over both beam modes. The ND statistics used in this analysis are scaled assuming this ratio throughout, using the same fraction of exposure in each beam mode as used at the FD. We note that the ND samples used in this analysis are relatively quickly systematics limited in both beam modes, and so these approximations are unlikely to have a significant impact on the results.

\begin{figure*}
  \centering
  \subfloat[FHC]{\includegraphics[width=0.8\linewidth]{ND_FHC_ndfd100ktMWyr_allsyst_asimov0_th13_constraint.png}}\\
  \subfloat[RHC]{\includegraphics[width=0.8\linewidth]{ND_RHC_ndfd100ktMWyr_allsyst_asimov0_th13_constraint.png}}
  \caption{ND samples in both FHC and RHC, shown in the reconstructed neutrino energy and reconstructed inelasticity binning ($y_{\mathrm{rec}}$) used in the analysis, shown for a $\sim$105 ton-year exposure (equivalent to a 100 kt-MW-yrs exposure at the FD), with an equal split between FHC and RHC. The size of the systematic uncertainty bands from all of the flux, cross-section and ND detector systematics used in the analysis are shown, as well as the postfit uncertainty bands obtained by performing an Asimov fit to the ND data. Background are also shown, which are dominated by NC events, although there is some contribution from wrong-sign \numu background events in RHC.}
 \label{fig:nd_samples}
\end{figure*}
The oscillation analysis presented here includes samples of $\nu_{\mu}$ and $\bar{\nu}_{\mu}$ charged-current interactions originating in the ND-LAr FV. These samples are binned in two dimensions, as a function of reconstructed neutrino energy and inelasticity, $y_{\mathrm{rec}} = 1 - E^{\mathrm{rec}}_{\mu}/E^{\mathrm{rec}}_{\nu}$, where $E^{\mathrm{rec}}_{\mu}$ and $E^{\mathrm{rec}}_{\nu}$ are the reconstructed muon and neutrino energies, respectively. The sample distributions for both FHC and RHC are shown in Figure~\ref{fig:nd_samples} for an exposure of $\sim$105 ton-years, corresponding to 100 kt-MW-yrs at the far detector with the assumptions stated above. The size of the systematic uncertainty bands from all of the flux, cross-section and ND detector systematics used in the analysis and described above are shown, as well as the postfit uncertainty bands obtained by performing an Asimov fit to the ND data. It is clear that even after a relatively small exposure of 105 ton-years, the ND samples are very high statistics, and are systematics limited in the binning used in the analysis. Backgrounds in the $\nu^{\bracketbar}_{\mu}$-CC samples are also shown in Figure~\ref{fig:nd_samples}. NC backgrounds are predominantly from NC $\pi^{\pm}$ production where the pion leaves a long track and does not shower. Wrong-sign contamination in the beam is a background where the charge of the muon is not reconstructed, which particularly affects low reconstructed neutrino energies in RHC. The wrong-sign background is also larger at high reconstructed inelasticity, $y_{\mathrm{rec}}$, due to the kinematics of neutrino and antineutrino scattering.

\begin{figure}[htbp]
 \subfloat[FHC]{\includegraphics[width=0.8\linewidth]{FD_nue_FHC_ndfd100ktMWyr_allsyst_asimov0_th13_constraint.png}}\\
 \subfloat[RHC]{\includegraphics[width=0.8\linewidth]{FD_nue_RHC_ndfd100ktMWyr_allsyst_asimov0_th13_constraint.png}}
 \caption{Reconstructed energy distribution of selected CC $\nue^{\bracketbar}$-like events in the FD, for a 50 kt-MW-yrs exposure in both neutrino-enhanced FHC beam mode and antineutrino-enhanced RHC beam mode, for a total 100 kt-MW-yrs year exposure. The plots are shown for NO, all other oscillation parameters are set to their NuFIT 4.0 best-fit values (see Table~\ref{tab:oscpar_nufit}). The size of the systematic uncertainty bands from all of the flux, cross-section and FD detector systematics used in the analysis are shown, as well as the postfit uncertainty bands with parameters constrained by ND data. Backgrounds are also shown, the largest contribution comes from intrinsic $\nue^{\bracketbar}$ contamination in the beam, although NC and other flavors, $\numu^{\bracketbar} + \nutau^{\bracketbar}$, also contribute.}
 \label{fig:appspectra}
\end{figure}
The expected FD FHC \nue and RHC \anue samples are shown in Figure~\ref{fig:appspectra} for a 100 kt-MW-yrs total FD exposure, split equally between FHC and RHC beam modes. The systematic uncertainty bands with and without the ND constraint applied are shown, as well as the background contributions. Note that there are contributions from both \nue and \anue in both beam modes. The NC, intrinsic beam $\nue^{\bracketbar}$, and wrong flavor contamination is also shown; the largest background comes from the intrinsic $\nue^{\bracketbar}$ beam contribution in both modes. After a 50 kt-MW-yrs exposure in FHC, the \nue sample statistical uncertainty is close to the systematic uncertainty before the ND constraint, although is still clearly statistics limited when the ND constraint is applied. The \anue sample is still strongly statistics limited after 50 kt-MW-yrs exposure in RHC. The difference is largely due to the difference in the \nue and \anue cross sections. %The $\nue^{\bracketbar}$ samples in both modes are clearly statistics limited until much larger exposures.

\begin{figure}[htbp]
  \subfloat[FHC]{\includegraphics[width=0.8\linewidth]{FD_numu_FHC_ndfd100ktMWyr_allsyst_asimov0_th13_constraint.png}}\\
  \subfloat[RHC]{\includegraphics[width=0.8\linewidth]{FD_numu_RHC_ndfd100ktMWyr_allsyst_asimov0_th13_constraint.png}}
\caption{Reconstructed energy distribution of selected CC $\nu^{\bracketbar}_{\mu}$-like events in the FD, for 50 kt-MW-yrs exposure in both neutrino-enhanced FHC beam mode and antineutrino-enhanced RHC beam mode, for a total 100 kt-MW-yrs exposure. The plots are shown for NO, all other oscillation parameters are set to their NuFIT 4.0 best-fit values (see Table~\ref{tab:oscpar_nufit}). The size of the systematic uncertainty bands from all of the flux, cross-section and ND detector systematics used in the analysis are shown, as well as the postfit uncertainty bands with parameters constrained by ND data. NC and wrong-sign backgrounds are also shown, the only sizeable background is the wrong-sign (\numu) contribution to the RHC sample.}
\label{fig:disspectra}
\end{figure}
The expected FD FHC \numu and RHC \anumu samples are shown in Figure~\ref{fig:disspectra} for a 100 kt-MW-yrs total FD exposure, split equally between FHC and RHC beam modes. The systematic uncertainty bands with and without the ND constraint applied are shown, as well as the background contributions. Note that although the wrong-sign \numu contribution to the RHC \anumu sample is shown as a separate contribution, it still provides useful information for constraining the oscillation parameters and is included in the analysis. The statistics are much higher than in Figure~\ref{fig:appspectra}; the statistical uncertainty on the \numu FHC sample is smaller than the systematic uncertainty band for some regions of phase space, even after the ND constraint is applied, although the statistical uncertainty is larger than the constrained systematic uncertainty in the ``dip'' region, around 2.5 GeV, which is likely to have the most impact on the analysis. The statistical uncertainty on the \anumu RHC sample is larger, again due to the smaller \anumu (than \numu) cross section. The statistical uncertainty around the 2.5 GeV dip region is significantly larger than the systematic uncertainty band, although as for the FHC \numu sample, the statistical uncertainty is smaller than the systematics for some regions of the parameter space.

Note that the events with reconstructed neutrino energy less than 0.5 GeV (which are shown in Figures~\ref{fig:nd_samples}, ~\ref{fig:appspectra} and~\ref{fig:disspectra}) or neutrino energies greater than 10 GeV are not included in the analysis for any of the FD or ND samples.



