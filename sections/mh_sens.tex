\section{Mass hierarchy sensitivity}
\label{sec:mh_sens}

In this section, the toy throwing approach described in Section~\ref{sec:analysis_framework} is used to explore the mass hierarchy sensitivity as a function of exposure in detail.

\begin{figure}[htbp]
  \centering
  \includegraphics[width=0.8\linewidth, trim={0cm 0cm 0cm 2.3cm}, clip]{mh_two_exps_throws_nh_2019_v4_lowexp.png}
  \includegraphics[width=0.8\linewidth, trim={0cm 0cm 0cm 2.3cm}, clip]{mh_two_exps_throws_ih_2019_v4_lowexp.png}
  \caption{Significance of the DUNE determination of the neutrino mass ordering, as a function of the true value of \deltacp, for 66 ktMWyr (blue) and 100 ktMWyr (orange) exposures. The width of the transparent bands cover 68\% of fits in which random throws are used to simulate systematic, oscillation parameter and statistical variations, with independent fits performed for each throw constrained by pre-fit uncertainties. The solid lines show the median sensitivity.}
  \label{fig:mh_bands}
\end{figure}
Figure~\ref{fig:mh_bands} shows the significance with which the neutrino mass ordering can be observed for both true NH and IH, for exposures of 66 kTMWyr and 100 ktMWyr. For each throw of the systematic, other oscillation parameters and statistics, two fits are carried out, one using each hierarchy. The difference in the best-fit $\chi^{2}$ values is calculated:
\begin{equation}
  \Delta\chi^{2} = \chi^{2}_{\mathrm{IH}} - \chi^{2}_{\mathrm{NH}},
  \label{eq:mh_chi2}
\end{equation}
\noindent and the square root of the difference is used as the figure of merit on the y-axis of Figure~\ref{fig:mh_bands}.

The characteristic shape of the MH sensitivity in Figure~\ref{fig:mh_bands} results from near degeneracy between matter and CPV effects that occurs near $\deltacp=\pi/2$ ($-\deltacp=\pi/2$) for true normal (inverted) ordering. We note that dedicated studies have shown that special attention must be paid to the statistical interpretation of neutrino mass ordering sensitivities~\cite{Ciuffoli:2013rza,Qian:2012zn,Blennow:2013oma} because the \dchisq metric does not follow the expected chi-square function for one degree of freedom, so the interpretation of the $\sqrt{\dchisq}$ as the sensitivity is complicated.

\begin{figure*}[htbp]
  \centering
  \subfloat[6 ktMWyr]   {\includegraphics[width=0.33\linewidth]{MH_comp_ndfd_6ktMWyr_th13.png}}
  \subfloat[12 ktMWyr]  {\includegraphics[width=0.33\linewidth]{MH_comp_ndfd_12ktMWyr_th13.png}}
  \subfloat[24 ktMWyr]  {\includegraphics[width=0.33\linewidth]{MH_comp_ndfd_24ktMWyr_th13.png}}\\
  \subfloat[66 ktMWyr]  {\includegraphics[width=0.33\linewidth]{MH_comp_ndfd_66ktMWyr_th13.png}}
  \subfloat[100 ktMWyr] {\includegraphics[width=0.33\linewidth]{MH_comp_ndfd_100ktMWyr_th13.png}}
  \subfloat[336 ktMWyr] {\includegraphics[width=0.33\linewidth]{MH_comp_ndfd_336ktMWyr_th13.png}}
  \caption{The distribution of $\dchisq = \chi^{2}_{\mathrm{IH}} - \chi^{2}_{\mathrm{NH}}$ values shown for both true normal (red) and true inverted (blue) hierarchies where . The fraction of throws for which the value of \dchisq is greater than (less than) 0 is also given for inverted (normal) hierarchies. For each hierarchy and exposure, approximately 100,000 throws were used.}
  \label{fig:mh_comp_over_time}
\end{figure*}
Given the complications with the interpretation of significance for mass hierarchy determination, it is instructive to look at the distribution of the test-statistic (Equation~\ref{eq:mh_chi2}), which gives more information than the 68\% central band and median throw shown in Figure~\ref{eq:mh_bands}. Figure~\ref{fig:mh_comp_over_time} shows the distribution of $\Delta\chi^{2}$ obtained for a large ensemble of throws, for both true and inverted orderings, for a number of different exposures. Note that there is a uniform distribution of true \deltacp used in the throws. The separation between normal and inverted orderings shown increases over time, but the orderings are not degenerate even at very low exposures. The shape of the throw distribution is highly non-Gaussian, which makes it difficult to apply simple corrections as described in Ref.~\cite{Blennow:2013oma}


\begin{figure*}[htbp]
  \centering
  \subfloat[6 ktMWyr]   {\includegraphics[width=0.33\linewidth]{mh_throws_6ktMWyr_NH_th13.png}}
  \subfloat[12 ktMWyr]  {\includegraphics[width=0.33\linewidth]{mh_throws_12ktMWyr_NH_th13.png}}
  \subfloat[24 ktMWyr]  {\includegraphics[width=0.33\linewidth]{mh_throws_24ktMWyr_NH_th13.png}}\\
  \subfloat[66 ktMWyr]  {\includegraphics[width=0.33\linewidth]{mh_throws_66ktMWyr_NH_th13.png}}
  \subfloat[100 ktMWyr] {\includegraphics[width=0.33\linewidth]{mh_throws_100ktMWyr_NH_th13.png}}
  \subfloat[336 ktMWyr] {\includegraphics[width=0.33\linewidth]{mh_throws_336ktMWyr_NH_th13.png}}
  \caption{Fraction of throws for which the DUNE sensitivity to the mass hierarchy exceeds 1--5$\sigma$ significance, as a function of the true value of \deltacp. Shown for NH, for a number of different exposures. The number of throws used to make each figure is also shown.}
  \label{fig:mh_nh_over_time}
\end{figure*}
