\RequirePackage{lineno}
\documentclass[aps,prd,twocolumn,showpacs,superscriptaddress,nofootinbib,floatfix,letterpaper]{revtex4-1}
\pdfoutput=1
\usepackage{placeins}
\usepackage{multirow}
\usepackage[utf8]{inputenc}
\usepackage{color}
\usepackage[caption=false]{subfig}
\usepackage{xspace}
\usepackage{graphicx}
\usepackage[pdftex,bookmarks,hidelinks]{hyperref}
\usepackage{amsmath}

\graphicspath{ {graphics/} }

\newcommand{\numu}{\ensuremath{\nu_\mu}\xspace}
\newcommand{\nue}{\ensuremath{\nu_e}\xspace}
\newcommand{\nutau}{\ensuremath{\nu_\tau}\xspace}
\newcommand{\anumu}{\ensuremath{\bar\nu_\mu}\xspace}
\newcommand{\anue}{\ensuremath{\bar\nu_e}\xspace}
\newcommand{\anutau}{\ensuremath{\bar\nu_\tau}\xspace}

\newcommand{\dm}[1]{\ensuremath{\Delta m^2_{#1}}\xspace} % example: \dm{12}
\newcommand{\sinst}[1]{\ensuremath{\sin^2\theta_{#1}}\xspace} % example \sinst{12}
\newcommand{\sinstt}[1]{\ensuremath{\sin^22\theta_{#1}}\xspace}  % example \sinstt{12}
\newcommand{\deltacp}{\ensuremath{\delta_{\rm CP}}\xspace}   % example \deltacp

\newcommand{\numutonumu}{\ensuremath{\numu\rightarrow\numu}\xspace}
\newcommand{\numutonue}{\ensuremath{\numu\rightarrow\nue}\xspace}

\newcommand{\numubartonumubar}{
\ensuremath{\overline{\numu}\rightarrow\overline{\numu}}\xspace
}

\newcommand{\numubartonuebar}{
\ensuremath{\overline{\numu}\rightarrow\overline{\nue}}\xspace
}

\newcommand{\todo}[1]{\textcolor{blue}{#1}}
\newcommand\addcite{[{\color{blue} \underline{CITATION NEEDED}}]~}
\def\bracketbar{\hbox{\kern-8pt\raise1pt%
    \hbox{{\tiny(}{\lower1.5pt\hbox{\bf--}}{\tiny)}}}}
\newcommand{\dchisq}{\ensuremath{\Delta\chi^{2}}\xspace}
\newcommand{\dchisqcrit}{\ensuremath{\Delta\chi^{2}_{c}}\xspace}

\begin{document}

\title{Long-baseline neutrino oscillation physics potential of the DUNE experiment}
\date{\today}
\collaboration{DUNE Collaboration}
\noaffiliation
% % List of institutions, need not be alphabetical
% Need to also add institution as \affiliation below this to get alphabetical order right
\newcommand{\Abilene}{Abilene Christian University, Abilene, TX 79601, USA}
\newcommand{\Albanysuny}{University of Albany, SUNY, Albany, NY 12222, USA}
\newcommand{\Amsterdam}{University of Amsterdam, NL-1098 XG Amsterdam, The Netherlands}
\newcommand{\Antalya}{Antalya Bilim University, 07190 D{\"o}{\c{s}}emealt{\i}/Antalya, Turkey}
\newcommand{\Antananarivo}{University of Antananarivo, Antananarivo 101, Madagascar}
\newcommand{\AntonioNarino}{Universidad Antonio Nari{\~n}o, Bogot{\'a}, Colombia}
\newcommand{\Argonne}{Argonne National Laboratory, Argonne, IL 60439, USA}
\newcommand{\Arizona}{University of Arizona, Tucson, AZ 85721, USA}
\newcommand{\Asuncion}{Universidad Nacional de Asunci{\'o}n, San Lorenzo, Paraguay}
\newcommand{\Athens}{University of Athens, Zografou GR 157 84, Greece}
\newcommand{\Atlantico}{Universidad del Atl{\'a}ntico, Barranquilla, Atl{\'a}ntico, Colombia}
\newcommand{\Augustana}{Augustana University, Sioux Falls, SD 57197, USA}
\newcommand{\Banaras}{Banaras Hindu University, Varanasi - 221 005, India}
\newcommand{\Basel}{University of Basel, CH-4056 Basel, Switzerland}
\newcommand{\Bern}{University of Bern, CH-3012 Bern, Switzerland}
\newcommand{\Beykent}{Beykent University, Istanbul, Turkey}
\newcommand{\Birmingham}{University of Birmingham, Birmingham B15 2TT, United Kingdom}
\newcommand{\BolognaUniversity}{Universit{\`a} del Bologna, 40127 Bologna, Italy}
\newcommand{\Boston}{Boston University, Boston, MA 02215, USA}
\newcommand{\Bristol}{University of Bristol, Bristol BS8 1TL, United Kingdom}
\newcommand{\Brookhaven}{Brookhaven National Laboratory, Upton, NY 11973, USA}
\newcommand{\Bucharest}{University of Bucharest, Bucharest, Romania}
\newcommand{\CBPF}{Centro Brasileiro de Pesquisas F\'isicas, Rio de Janeiro, RJ 22290-180, Brazil}
\newcommand{\CEASaclay}{IRFU, CEA, Universit{\'e} Paris-Saclay, F-91191 Gif-sur-Yvette, France}
\newcommand{\CERN}{CERN, The European Organization for Nuclear Research, 1211 Meyrin, Switzerland}
\newcommand{\CIEMAT}{CIEMAT, Centro de Investigaciones Energ{\'e}ticas, Medioambientales y Tecnol{\'o}gicas, E-28040 Madrid, Spain}
\newcommand{\CUSB}{Central University of South Bihar, Gaya, 824236, India }
\newcommand{\CalBerkeley}{University of California Berkeley, Berkeley, CA 94720, USA}
\newcommand{\CalDavis}{University of California Davis, Davis, CA 95616, USA}
\newcommand{\CalIrvine}{University of California Irvine, Irvine, CA 92697, USA}
\newcommand{\CalLosangeles}{University of California Los Angeles, Los Angeles, CA 90095, USA}
\newcommand{\CalRiverside}{University of California Riverside, Riverside CA 92521, USA}
\newcommand{\CalSantabarbara}{University of California Santa Barbara, Santa Barbara, California 93106 USA}
\newcommand{\Caltech}{California Institute of Technology, Pasadena, CA 91125, USA}
\newcommand{\Cambridge}{University of Cambridge, Cambridge CB3 0HE, United Kingdom}
\newcommand{\Campinas}{Universidade Estadual de Campinas, Campinas - SP, 13083-970, Brazil}
\newcommand{\CataniaUniversitadi}{Universit{\`a} di Catania, 2 - 95131 Catania, Italy}
\newcommand{\Charles}{Institute of Particle and Nuclear Physics of the Faculty of Mathematics and Physics of the Charles University, 180 00 Prague 8, Czech Republic }
\newcommand{\Chicago}{University of Chicago, Chicago, IL 60637, USA}
\newcommand{\ChungAng}{Chung-Ang University, Seoul 06974, South Korea}
\newcommand{\Cincinnati}{University of Cincinnati, Cincinnati, OH 45221, USA}
\newcommand{\Cinvestav}{Centro de Investigaci{\'o}n y de Estudios Avanzados del Instituto Polit{\'e}cnico Nacional (Cinvestav), Mexico City, Mexico}
\newcommand{\Colima}{Universidad de Colima, Colima, Mexico}
\newcommand{\ColoradoBoulder}{University of Colorado Boulder, Boulder, CO 80309, USA}
\newcommand{\ColoradoState}{Colorado State University, Fort Collins, CO 80523, USA}
\newcommand{\Columbia}{Columbia University, New York, NY 10027, USA}
\newcommand{\CzechAcademyofSciences}{Institute of Physics, Czech Academy of Sciences, 182 00 Prague 8, Czech Republic}
\newcommand{\CzechTechnical}{Czech Technical University, 115 19 Prague 1, Czech Republic}
\newcommand{\DakotaState}{Dakota State University, Madison, SD 57042, USA}
\newcommand{\Dallas}{University of Dallas, Irving, TX 75062-4736, USA}
\newcommand{\DannecyleVieux}{Laboratoire d{\textquoteright}Annecy de Physique des Particules, Univ. Grenoble Alpes, Univ. Savoie Mont Blanc, CNRS, LAPP-IN2P3, 74000 Annecy, France}
\newcommand{\Daresbury}{Daresbury Laboratory, Cheshire WA4 4AD, United Kingdom}
\newcommand{\Drexel}{Drexel University, Philadelphia, PA 19104, USA}
\newcommand{\Duke}{Duke University, Durham, NC 27708, USA}
\newcommand{\Durham}{Durham University, Durham DH1 3LE, United Kingdom}
\newcommand{\EIA}{Universidad EIA, Envigado, Antioquia, Colombia}
\newcommand{\ETH}{ETH Zurich, Zurich, Switzerland}
\newcommand{\Edinburgh}{University of Edinburgh, Edinburgh EH8 9YL, United Kingdom}
\newcommand{\FCULport}{Faculdade de Ci{\^e}ncias da Universidade de Lisboa - FCUL, 1749-016 Lisboa, Portugal}
\newcommand{\FederaldeAlfenas}{Universidade Federal de Alfenas, Po{\c{c}}os de Caldas - MG, 37715-400, Brazil}
\newcommand{\FederaldeGoias}{Universidade Federal de Goias, Goiania, GO 74690-900, Brazil}
\newcommand{\FederaldeSaoCarlos}{Universidade Federal de S{\~a}o Carlos, Araras - SP, 13604-900, Brazil}
\newcommand{\FederaldoABC}{Universidade Federal do ABC, Santo Andr{\'e} - SP, 09210-580, Brazil}
\newcommand{\FederaldoRio}{Universidade Federal do Rio de Janeiro,  Rio de Janeiro - RJ, 21941-901, Brazil}
\newcommand{\Fermi}{Fermi National Accelerator Laboratory, Batavia, IL 60510, USA}
\newcommand{\Ferrarauniv}{University of Ferrara, Ferrara, Italy}
\newcommand{\Florida}{University of Florida, Gainesville, FL 32611-8440, USA}
\newcommand{\Fluminense}{Fluminense Federal University, 9 Icara{\'\i} Niter{\'o}i - RJ, 24220-900, Brazil }
\newcommand{\Genova}{Universit{\`a} degli Studi di Genova, Genova, Italy}
\newcommand{\Georgian}{Georgian Technical University, Tbilisi, Georgia}
\newcommand{\GranSasso}{Gran Sasso Science Institute, L'Aquila, Italy}
\newcommand{\GranSassoLab}{Laboratori Nazionali del Gran Sasso, L'Aquila AQ, Italy}
\newcommand{\Granada}{University of Granada {\&} CAFPE, 18002 Granada, Spain}
\newcommand{\Grenoble}{University Grenoble Alpes, CNRS, Grenoble INP, LPSC-IN2P3, 38000 Grenoble, France}
\newcommand{\Guanajuato}{Universidad de Guanajuato, Guanajuato, C.P. 37000, Mexico}
\newcommand{\Harish}{Harish-Chandra Research Institute, Jhunsi, Allahabad 211 019, India}
\newcommand{\Harvard}{Harvard University, Cambridge, MA 02138, USA}
\newcommand{\Hawaii}{University of Hawaii, Honolulu, HI 96822, USA}
\newcommand{\Houston}{University of Houston, Houston, TX 77204, USA}
\newcommand{\Hyderabad}{University of  Hyderabad, Gachibowli, Hyderabad - 500 046, India}
\newcommand{\IFAE}{Institut de F{\'\i}sica d{\textquoteright}Altes Energies (IFAE){\textemdash}Barcelona Institute of Science and Technology (BIST), Barcelona, Spain}
\newcommand{\IFIC}{Instituto de F{\'\i}sica Corpuscular, CSIC and Universitat de Val{\`e}ncia, 46980 Paterna, Valencia, Spain}
\newcommand{\IGFAE}{Instituto Galego de Fisica de Altas Enerxias, A Coru{\~n}a, Spain}
\newcommand{\INFNBologna}{Istituto Nazionale di Fisica Nucleare Sezione di Bologna, 40127 Bologna BO, Italy}
\newcommand{\INFNCatania}{Istituto Nazionale di Fisica Nucleare Sezione di Catania, I-95123 Catania, Italy}
\newcommand{\INFNFerrara}{Istituto Nazionale di Fisica Nucleare Sezione di Ferrara, I-44122 Ferrara, Italy}
\newcommand{\INFNGenova}{Istituto Nazionale di Fisica Nucleare Sezione di Genova, 16146 Genova GE, Italy}
\newcommand{\INFNLecce}{Istituto Nazionale di Fisica Nucleare Sezione di Lecce, 73100 - Lecce, Italy}
\newcommand{\INFNMilanBicocca}{Istituto Nazionale di Fisica Nucleare Sezione di Milano Bicocca, 3 - I-20126 Milano, Italy}
\newcommand{\INFNMilano}{Istituto Nazionale di Fisica Nucleare Sezione di Milano, 20133 Milano, Italy}
\newcommand{\INFNNapoli}{Istituto Nazionale di Fisica Nucleare Sezione di Napoli, I-80126 Napoli, Italy}
\newcommand{\INFNPadova}{Istituto Nazionale di Fisica Nucleare Sezione di Padova, 35131 Padova, Italy}
\newcommand{\INFNPavia}{Istituto Nazionale di Fisica Nucleare Sezione di Pavia,  I-27100 Pavia, Italy}
\newcommand{\INFNSud}{Istituto Nazionale di Fisica Nucleare Laboratori Nazionali del Sud, 95123 Catania, Italy}
\newcommand{\INR}{Institute for Nuclear Research of the Russian Academy of Sciences, Moscow 117312, Russia}
\newcommand{\IPLyon}{Institut de Physique des 2 Infinis de Lyon, 69622 Villeurbanne, France}
\newcommand{\IPM}{Institute for Research in Fundamental Sciences, Tehran, Iran}
\newcommand{\ISTlisboa}{Instituto Superior T{\'e}cnico - IST, Universidade de Lisboa, Portugal}
\newcommand{\Idaho}{Idaho State University, Pocatello, ID 83209, USA}
\newcommand{\Illinoisinstitute}{Illinois Institute of Technology, Chicago, IL 60616, USA}
\newcommand{\Imperial}{Imperial College of Science Technology and Medicine, London SW7 2BZ, United Kingdom}
\newcommand{\IndGuwahati}{Indian Institute of Technology Guwahati, Guwahati, 781 039, India}
\newcommand{\IndHyderabad}{Indian Institute of Technology Hyderabad, Hyderabad, 502285, India}
\newcommand{\Indiana}{Indiana University, Bloomington, IN 47405, USA}
\newcommand{\Ingenieria}{Universidad Nacional de Ingenier{\'\i}a, Lima 25, Per{\'u}}
\newcommand{\Insubria }{University of Insubria, Via Ravasi, 2, 21100 Varese VA, Italy}
\newcommand{\Iowa}{University of Iowa, Iowa City, IA 52242, USA}
\newcommand{\IowaState}{Iowa State University, Ames, Iowa 50011, USA}
\newcommand{\Iwate}{Iwate University, Morioka, Iwate 020-8551, Japan}
\newcommand{\JINR}{Joint Institute for Nuclear Research, Dzhelepov Laboratory of Nuclear Problems 6 Joliot-Curie, Dubna, Moscow Region, 141980 RU }
\newcommand{\Jammu}{University of Jammu, Jammu-180006, India}
\newcommand{\Jawaharlal}{Jawaharlal Nehru University, New Delhi 110067, India}
\newcommand{\Jeonbuk}{Jeonbuk National University, Jeonrabuk-do 54896, South Korea}
\newcommand{\Jyvaskyla}{University of Jyvaskyla, FI-40014, Finland}
\newcommand{\KEK}{High Energy Accelerator Research Organization (KEK), Ibaraki, 305-0801, Japan}
\newcommand{\KISTI}{Korea Institute of Science and Technology Information, Daejeon, 34141, South Korea}
\newcommand{\KL}{K L University, Vaddeswaram, Andhra Pradesh 522502, India}
\newcommand{\Kansasstate}{Kansas State University, Manhattan, KS 66506, USA}
\newcommand{\Kavli}{Kavli Institute for the Physics and Mathematics of the Universe, Kashiwa, Chiba 277-8583, Japan}
\newcommand{\Kure}{National Institute of Technology, Kure College, Hiroshima, 737-8506, Japan}
\newcommand{\Kyiv}{Taras Shevchenko National University of Kyiv, 01601 Kyiv, Ukraine}
\newcommand{\LIP}{Laborat{\'o}rio de Instrumenta{\c{c}}{\~a}o e F{\'\i}sica Experimental de Part{\'\i}culas, 1649-003 Lisboa and 3004-516 Coimbra, Portugal}
\newcommand{\Lancaster}{Lancaster University, Lancaster LA1 4YB, United Kingdom}
\newcommand{\LawrenceBerkeley}{Lawrence Berkeley National Laboratory, Berkeley, CA 94720, USA}
\newcommand{\Liverpool}{University of Liverpool, L69 7ZE, Liverpool, United Kingdom}
\newcommand{\LosAlmos}{Los Alamos National Laboratory, Los Alamos, NM 87545, USA}
\newcommand{\Louisanastate}{Louisiana State University, Baton Rouge, LA 70803, USA}
\newcommand{\Lucknow}{University of Lucknow, Uttar Pradesh 226007, India}
\newcommand{\Madrid}{Madrid Autonoma University and IFT UAM/CSIC, 28049 Madrid, Spain}
\newcommand{\Manchester}{University of Manchester, Manchester M13 9PL, United Kingdom}
\newcommand{\Massinsttech}{Massachusetts Institute of Technology, Cambridge, MA 02139, USA}
\newcommand{\Maxplanck}{Max-Planck-Institut, Munich, 80805, Germany}
\newcommand{\Medellin}{University of Medell{\'\i}n, Medell{\'\i}n, 050026 Colombia }
\newcommand{\Michigan}{University of Michigan, Ann Arbor, MI 48109, USA}
\newcommand{\Michiganstate}{Michigan State University, East Lansing, MI 48824, USA}
\newcommand{\MilanoBicocca}{Universit{\`a} del Milano-Bicocca, 20126 Milano, Italy}
\newcommand{\MilanoUniv}{Universit{\`a} degli Studi di Milano, I-20133 Milano, Italy}
\newcommand{\Minnduluth}{University of Minnesota Duluth, Duluth, MN 55812, USA}
\newcommand{\Minntwin}{University of Minnesota Twin Cities, Minneapolis, MN 55455, USA}
\newcommand{\Mississippi}{University of Mississippi, University, MS 38677 USA}
\newcommand{\Newmexico}{University of New Mexico, Albuquerque, NM 87131, USA}
\newcommand{\Niewodniczanski}{H. Niewodnicza{\'n}ski Institute of Nuclear Physics, Polish Academy of Sciences, Cracow, Poland}
\newcommand{\Nikhef}{Nikhef National Institute of Subatomic Physics, 1098 XG Amsterdam, Netherlands}
\newcommand{\Northdakota}{University of North Dakota, Grand Forks, ND 58202-8357, USA}
\newcommand{\Northernillinois}{Northern Illinois University, DeKalb, IL 60115, USA}
\newcommand{\Northwestern}{Northwestern University, Evanston, Il 60208, USA}
\newcommand{\NotreDame}{University of Notre Dame, Notre Dame, IN 46556, USA}
\newcommand{\Occidental}{Occidental College, Los Angeles, CA  90041}
\newcommand{\Ohiostate}{Ohio State University, Columbus, OH 43210, USA}
\newcommand{\OregonState}{Oregon State University, Corvallis, OR 97331, USA}
\newcommand{\Oxford}{University of Oxford, Oxford, OX1 3RH, United Kingdom}
\newcommand{\PacificNorthwest}{Pacific Northwest National Laboratory, Richland, WA 99352, USA}
\newcommand{\Padova}{Universt{\`a} degli Studi di Padova, I-35131 Padova, Italy}
\newcommand{\Panjab}{Panjab University, Chandigarh, 160014 U.T., India}
\newcommand{\Parissaclay}{Universit{\'e} Paris-Saclay, CNRS/IN2P3, IJCLab, 91405 Orsay, France}
\newcommand{\Parisuniversite}{Universit{\'e} de Paris, CNRS, Astroparticule et Cosmologie, F-75006, Paris, France}
\newcommand{\Pavia}{Universit{\`a} degli Studi di Pavia, 27100 Pavia PV, Italy}
\newcommand{\Penn}{University of Pennsylvania, Philadelphia, PA 19104, USA}
\newcommand{\PennState}{Pennsylvania State University, University Park, PA 16802, USA}
\newcommand{\PhysicalResearchLaboratory}{Physical Research Laboratory, Ahmedabad 380 009, India}
\newcommand{\Pisa}{Universit{\`a} di Pisa, I-56127 Pisa, Italy}
\newcommand{\Pitt}{University of Pittsburgh, Pittsburgh, PA 15260, USA}
\newcommand{\Pontificia}{Pontificia Universidad Cat{\'o}lica del Per{\'u}, Lima, Per{\'u}}
\newcommand{\PuertoRico}{University of Puerto Rico, Mayaguez 00681, Puerto Rico, USA}
\newcommand{\Punjab}{Punjab Agricultural University, Ludhiana 141004, India}
\newcommand{\QMUL}{Queen Mary University of London, London E1 4NS, United Kingdom }
\newcommand{\Radboud}{Radboud University, NL-6525 AJ Nijmegen, Netherlands}
\newcommand{\Rochester}{University of Rochester, Rochester, NY 14627, USA}
\newcommand{\Royalholloway}{Royal Holloway College London, TW20 0EX, United Kingdom}
\newcommand{\Rutgers}{Rutgers University, Piscataway, NJ, 08854, USA}
\newcommand{\Rutherford}{STFC Rutherford Appleton Laboratory, Didcot OX11 0QX, United Kingdom}
\newcommand{\SLAC}{SLAC National Accelerator Laboratory, Menlo Park, CA 94025, USA}
\newcommand{\SURF}{Sanford Underground Research Facility, Lead, SD, 57754, USA}
\newcommand{\Salento}{Universit{\`a} del Salento, 73100 Lecce, Italy}
\newcommand{\Sanjosestate}{San Jose State University, San Jos{\'e}, CA 95192-0106, USA}
\newcommand{\SergioArboleda}{Universidad Sergio Arboleda, 11022 Bogot{\'a}, Colombia}
\newcommand{\Sheffield}{University of Sheffield, Sheffield S3 7RH, United Kingdom}
\newcommand{\SouthDakotaSchool}{South Dakota School of Mines and Technology, Rapid City, SD 57701, USA}
\newcommand{\SouthDakotaState}{South Dakota State University, Brookings, SD 57007, USA}
\newcommand{\Southcarolina}{University of South Carolina, Columbia, SC 29208, USA}
\newcommand{\SouthernMethodist}{Southern Methodist University, Dallas, TX 75275, USA}
\newcommand{\StonyBrook}{Stony Brook University, SUNY, Stony Brook, NY 11794, USA}
\newcommand{\Sunyatsen}{Sun Yat-Sen University, Guangzhou, 510275}
\newcommand{\Sussex}{University of Sussex, Brighton, BN1 9RH, United Kingdom}
\newcommand{\Syracuse}{Syracuse University, Syracuse, NY 13244, USA}
\newcommand{\Tecnologica }{Universidade Tecnol{\'o}gica Federal do Paran{\'a}, Curitiba, Brazil}
\newcommand{\TexasAMcollege}{Texas A{\&}M University, College Station, Texas 77840}
\newcommand{\TexasAMcorpuscristi}{Texas A{\&}M University - Corpus Christi, Corpus Christi, TX 78412, USA}
\newcommand{\TexasArlington}{University of Texas at Arlington, Arlington, TX 76019, USA}
\newcommand{\Texasaustin}{University of Texas at Austin, Austin, TX 78712, USA}
\newcommand{\Toronto}{University of Toronto, Toronto, Ontario M5S 1A1, Canada}
\newcommand{\Tufts}{Tufts University, Medford, MA 02155, USA}
\newcommand{\UNIST}{Ulsan National Institute of Science and Technology, Ulsan 689-798, South Korea}
\newcommand{\Unifesp}{Universidade Federal de S{\~a}o Paulo, 09913-030, S{\~a}o Paulo, Brazil}
\newcommand{\UniversityCollegeLondon}{University College London, London, WC1E 6BT, United Kingdom}
\newcommand{\ValleyCity}{Valley City State University, Valley City, ND 58072, USA}
\newcommand{\VariableEnergy}{Variable Energy Cyclotron Centre, 700 064 West Bengal, India}
\newcommand{\VirginiaTech}{Virginia Tech, Blacksburg, VA 24060, USA}
\newcommand{\Warsaw}{University of Warsaw, 02-093 Warsaw, Poland}
\newcommand{\Warwick}{University of Warwick, Coventry CV4 7AL, United Kingdom}
\newcommand{\Wellesley}{Wellesley College, Wellesley, MA 02481, USA}
\newcommand{\Wichita}{Wichita State University, Wichita, KS 67260, USA}
\newcommand{\WilliamMary}{William and Mary, Williamsburg, VA 23187, USA}
\newcommand{\Wisconsin}{University of Wisconsin Madison, Madison, WI 53706, USA}
\newcommand{\Yale}{Yale University, New Haven, CT 06520, USA}
\newcommand{\Yerevan}{Yerevan Institute for Theoretical Physics and Modeling, Yerevan 0036, Armenia}
\newcommand{\York}{York University, Toronto M3J 1P3, Canada}
%----------------------------------------------------
% So that institutions appear in alphabetical order:
\affiliation{\Abilene}
\affiliation{\Albanysuny}
\affiliation{\Amsterdam}
\affiliation{\Antalya}
\affiliation{\Antananarivo}
\affiliation{\AntonioNarino}
\affiliation{\Argonne}
\affiliation{\Arizona}
\affiliation{\Asuncion}
\affiliation{\Athens}
\affiliation{\Atlantico}
\affiliation{\Augustana}
\affiliation{\Banaras}
\affiliation{\Basel}
\affiliation{\Bern}
\affiliation{\Beykent}
\affiliation{\Birmingham}
\affiliation{\BolognaUniversity}
\affiliation{\Boston}
\affiliation{\Bristol}
\affiliation{\Brookhaven}
\affiliation{\Bucharest}
\affiliation{\CBPF}
\affiliation{\CEASaclay}
\affiliation{\CERN}
\affiliation{\CIEMAT}
\affiliation{\CUSB}
\affiliation{\CalBerkeley}
\affiliation{\CalDavis}
\affiliation{\CalIrvine}
\affiliation{\CalLosangeles}
\affiliation{\CalRiverside}
\affiliation{\CalSantabarbara}
\affiliation{\Caltech}
\affiliation{\Cambridge}
\affiliation{\Campinas}
\affiliation{\CataniaUniversitadi}
\affiliation{\Charles}
\affiliation{\Chicago}
\affiliation{\ChungAng}
\affiliation{\Cincinnati}
\affiliation{\Cinvestav}
\affiliation{\Colima}
\affiliation{\ColoradoBoulder}
\affiliation{\ColoradoState}
\affiliation{\Columbia}
\affiliation{\CzechAcademyofSciences}
\affiliation{\CzechTechnical}
\affiliation{\DakotaState}
\affiliation{\Dallas}
\affiliation{\DannecyleVieux}
\affiliation{\Daresbury}
\affiliation{\Drexel}
\affiliation{\Duke}
\affiliation{\Durham}
\affiliation{\EIA}
\affiliation{\ETH}
\affiliation{\Edinburgh}
\affiliation{\FCULport}
\affiliation{\FederaldeAlfenas}
\affiliation{\FederaldeGoias}
\affiliation{\FederaldeSaoCarlos}
\affiliation{\FederaldoABC}
\affiliation{\FederaldoRio}
\affiliation{\Fermi}
\affiliation{\Ferrarauniv}
\affiliation{\Florida}
\affiliation{\Fluminense}
\affiliation{\Genova}
\affiliation{\Georgian}
\affiliation{\GranSasso}
\affiliation{\GranSassoLab}
\affiliation{\Granada}
\affiliation{\Grenoble}
\affiliation{\Guanajuato}
\affiliation{\Harish}
\affiliation{\Harvard}
\affiliation{\Hawaii}
\affiliation{\Houston}
\affiliation{\Hyderabad}
\affiliation{\IFAE}
\affiliation{\IFIC}
\affiliation{\IGFAE}
\affiliation{\INFNBologna}
\affiliation{\INFNCatania}
\affiliation{\INFNFerrara}
\affiliation{\INFNGenova}
\affiliation{\INFNLecce}
\affiliation{\INFNMilanBicocca}
\affiliation{\INFNMilano}
\affiliation{\INFNNapoli}
\affiliation{\INFNPadova}
\affiliation{\INFNPavia}
\affiliation{\INFNSud}
\affiliation{\INR}
\affiliation{\IPLyon}
\affiliation{\IPM}
\affiliation{\ISTlisboa}
\affiliation{\Idaho}
\affiliation{\Illinoisinstitute}
\affiliation{\Imperial}
\affiliation{\IndGuwahati}
\affiliation{\IndHyderabad}
\affiliation{\Indiana}
\affiliation{\Ingenieria}
\affiliation{\Insubria }
\affiliation{\Iowa}
\affiliation{\IowaState}
\affiliation{\Iwate}
\affiliation{\JINR}
\affiliation{\Jammu}
\affiliation{\Jawaharlal}
\affiliation{\Jeonbuk}
\affiliation{\Jyvaskyla}
\affiliation{\KEK}
\affiliation{\KISTI}
\affiliation{\KL}
\affiliation{\Kansasstate}
\affiliation{\Kavli}
\affiliation{\Kure}
\affiliation{\Kyiv}
\affiliation{\LIP}
\affiliation{\Lancaster}
\affiliation{\LawrenceBerkeley}
\affiliation{\Liverpool}
\affiliation{\LosAlmos}
\affiliation{\Louisanastate}
\affiliation{\Lucknow}
\affiliation{\Madrid}
\affiliation{\Manchester}
\affiliation{\Massinsttech}
\affiliation{\Maxplanck}
\affiliation{\Medellin}
\affiliation{\Michigan}
\affiliation{\Michiganstate}
\affiliation{\MilanoBicocca}
\affiliation{\MilanoUniv}
\affiliation{\Minnduluth}
\affiliation{\Minntwin}
\affiliation{\Mississippi}
\affiliation{\Newmexico}
\affiliation{\Niewodniczanski}
\affiliation{\Nikhef}
\affiliation{\Northdakota}
\affiliation{\Northernillinois}
\affiliation{\Northwestern}
\affiliation{\NotreDame}
\affiliation{\Occidental}
\affiliation{\Ohiostate}
\affiliation{\OregonState}
\affiliation{\Oxford}
\affiliation{\PacificNorthwest}
\affiliation{\Padova}
\affiliation{\Panjab}
\affiliation{\Parissaclay}
\affiliation{\Parisuniversite}
\affiliation{\Pavia}
\affiliation{\Penn}
\affiliation{\PennState}
\affiliation{\PhysicalResearchLaboratory}
\affiliation{\Pisa}
\affiliation{\Pitt}
\affiliation{\Pontificia}
\affiliation{\PuertoRico}
\affiliation{\Punjab}
\affiliation{\QMUL}
\affiliation{\Radboud}
\affiliation{\Rochester}
\affiliation{\Royalholloway}
\affiliation{\Rutgers}
\affiliation{\Rutherford}
\affiliation{\SLAC}
\affiliation{\SURF}
\affiliation{\Salento}
\affiliation{\Sanjosestate}
\affiliation{\SergioArboleda}
\affiliation{\Sheffield}
\affiliation{\SouthDakotaSchool}
\affiliation{\SouthDakotaState}
\affiliation{\Southcarolina}
\affiliation{\SouthernMethodist}
\affiliation{\StonyBrook}
\affiliation{\Sunyatsen}
\affiliation{\Sussex}
\affiliation{\Syracuse}
\affiliation{\Tecnologica }
\affiliation{\TexasAMcollege}
\affiliation{\TexasAMcorpuscristi}
\affiliation{\TexasArlington}
\affiliation{\Texasaustin}
\affiliation{\Toronto}
\affiliation{\Tufts}
\affiliation{\UNIST}
\affiliation{\Unifesp}
\affiliation{\UniversityCollegeLondon}
\affiliation{\ValleyCity}
\affiliation{\VariableEnergy}
\affiliation{\VirginiaTech}
\affiliation{\Warsaw}
\affiliation{\Warwick}
\affiliation{\Wellesley}
\affiliation{\Wichita}
\affiliation{\WilliamMary}
\affiliation{\Wisconsin}
\affiliation{\Yale}
\affiliation{\Yerevan}
\affiliation{\York}
%----------------------------------------------------
% Authors in alphabetical order
\author{A.~Abed Abud} \affiliation{\Liverpool}\affiliation{\CERN}
\author{B.~Abi} \affiliation{\Oxford}
\author{R.~Acciarri} \affiliation{\Fermi}
\author{M.~A.~Acero} \affiliation{\Atlantico}
\author{M.~R.~Adames} \affiliation{\Tecnologica }
\author{G.~Adamov} \affiliation{\Georgian}
\author{D.~Adams} \affiliation{\Brookhaven}
\author{M.~Adinolfi} \affiliation{\Bristol}
\author{A.~Aduszkiewicz} \affiliation{\Houston}
\author{J.~Aguilar} \affiliation{\LawrenceBerkeley}
\author{Z.~Ahmad} \affiliation{\VariableEnergy}
\author{J.~Ahmed} \affiliation{\Warwick}
\author{B.~Ali-Mohammadzadeh} \affiliation{\INFNCatania}\affiliation{\CataniaUniversitadi}
\author{T.~Alion} \affiliation{\Sussex}
\author{K.~Allison} \affiliation{\ColoradoBoulder}
\author{S.~Alonso Monsalve} \affiliation{\CERN}\affiliation{\ETH}
\author{M.~Alrashed} \affiliation{\Kansasstate}
\author{C.~Alt} \affiliation{\ETH}
\author{A.~Alton} \affiliation{\Augustana}
\author{P.~Amedo} \affiliation{\IGFAE}
\author{J.~Anderson} \affiliation{\Argonne}
\author{C.~Andreopoulos} \affiliation{\Rutherford}\affiliation{\Liverpool}
\author{M.~Andreotti} \affiliation{\INFNFerrara}\affiliation{\Ferrarauniv}
\author{M.~P.~Andrews} \affiliation{\Fermi}
\author{F.~Andrianala} \affiliation{\Antananarivo}
\author{S.~Andringa} \affiliation{\LIP}
\author{N.~Anfimov} \affiliation{\JINR}
\author{A.~Ankowski} \affiliation{\SLAC}
\author{M.~Antoniassi} \affiliation{\Tecnologica }
\author{M.~Antonova} \affiliation{\IFIC}
\author{A.~Antoshkin} \affiliation{\JINR}
\author{S.~Antusch} \affiliation{\Basel}
\author{A.~Aranda-Fernandez} \affiliation{\Colima}
\author{A.~Ariga} \affiliation{\Bern}
\author{L.~O.~Arnold} \affiliation{\Columbia}
\author{M.~A.~Arroyave} \affiliation{\EIA}
\author{J.~Asaadi} \affiliation{\TexasArlington}
\author{L.~Asquith} \affiliation{\Sussex}
\author{A.~Aurisano} \affiliation{\Cincinnati}
\author{V.~Aushev} \affiliation{\Kyiv}
\author{D.~Autiero} \affiliation{\IPLyon}
\author{M.~Ayala-Torres} \affiliation{\Cinvestav}
\author{F.~Azfar} \affiliation{\Oxford}
\author{A.~Back} \affiliation{\Indiana}
\author{H.~Back} \affiliation{\PacificNorthwest}
\author{J.~J.~Back} \affiliation{\Warwick}
\author{C.~Backhouse} \affiliation{\UniversityCollegeLondon}
\author{P.~Baesso} \affiliation{\Bristol}
\author{I.~Bagaturia} \affiliation{\Georgian}
\author{L.~Bagby} \affiliation{\Fermi}
\author{N.~Balashov} \affiliation{\JINR}
\author{S.~Balasubramanian} \affiliation{\Fermi}
\author{P.~Baldi} \affiliation{\CalIrvine}
\author{B.~Baller} \affiliation{\Fermi}
\author{B.~Bambah} \affiliation{\Hyderabad}
\author{F.~Barao} \affiliation{\LIP}\affiliation{\ISTlisboa}
\author{G.~Barenboim} \affiliation{\IFIC}
\author{G.~J.~Barker} \affiliation{\Warwick}
\author{W.~Barkhouse} \affiliation{\Northdakota}
\author{C.~Barnes} \affiliation{\Michigan}
\author{G.~Barr} \affiliation{\Oxford}
\author{J.~Barranco Monarca} \affiliation{\Guanajuato}
\author{A.~Barros} \affiliation{\Tecnologica }
\author{N.~Barros} \affiliation{\LIP}\affiliation{\FCULport}
\author{J.~L.~Barrow} \affiliation{\Massinsttech}
\author{A.~Basharina-Freshville} \affiliation{\UniversityCollegeLondon}
\author{A.~Bashyal} \affiliation{\OregonState}
\author{V.~Basque} \affiliation{\Manchester}
\author{E.~Belchior} \affiliation{\Campinas}
\author{J.B.R.~Battat} \affiliation{\Wellesley}
\author{F.~Battisti} \affiliation{\Oxford}
\author{F.~Bay} \affiliation{\Antalya}
\author{J.~L.~Bazo~Alba} \affiliation{\Pontificia}
\author{J.~F.~Beacom} \affiliation{\Ohiostate}
\author{E.~Bechetoille} \affiliation{\IPLyon}
\author{B.~Behera} \affiliation{\ColoradoState}
\author{L.~Bellantoni} \affiliation{\Fermi}
\author{G.~Bellettini} \affiliation{\Pisa}
\author{V.~Bellini} \affiliation{\INFNCatania}\affiliation{\CataniaUniversitadi}
\author{O.~Beltramello} \affiliation{\CERN}
\author{D.~Belver} \affiliation{\CIEMAT}
\author{N.~Benekos} \affiliation{\CERN}
\author{C.~Benitez Montiel} \affiliation{\Asuncion}
\author{F.~Bento Neves} \affiliation{\LIP}
\author{J.~Berger} \affiliation{\ColoradoState}
\author{S.~Berkman} \affiliation{\Fermi}
\author{P.~Bernardini} \affiliation{\INFNLecce}\affiliation{\Salento}
\author{R.~M.~Berner} \affiliation{\Bern}
\author{H.~Berns} \affiliation{\CalDavis}
\author{S.~Bertolucci} \affiliation{\INFNBologna}\affiliation{\BolognaUniversity}
\author{M.~Betancourt} \affiliation{\Fermi}
\author{A.~Betancur Rodríguez} \affiliation{\EIA}
\author{A.~Bevan} \affiliation{\QMUL}
\author{T.J.C.~Bezerra} \affiliation{\Sussex}
\author{V.~Bhatnagar} \affiliation{\Panjab}
\author{M.~Bhattacharjee} \affiliation{\IndGuwahati}
\author{S.~Bhuller} \affiliation{\Bristol}
\author{B.~Bhuyan} \affiliation{\IndGuwahati}
\author{S.~Biagi} \affiliation{\INFNSud}
\author{J.~Bian} \affiliation{\CalIrvine}
\author{M.~Biassoni} \affiliation{\INFNMilanBicocca}
\author{K.~Biery} \affiliation{\Fermi}
\author{B.~Bilki} \affiliation{\Beykent}\affiliation{\Iowa}
\author{M.~Bishai} \affiliation{\Brookhaven}
\author{A.~Bitadze} \affiliation{\Manchester}
\author{A.~Blake} \affiliation{\Lancaster}
\author{F.~D.~M.~Blaszczyk} \affiliation{\Fermi}
\author{G.~C.~Blazey} \affiliation{\Northernillinois}
\author{E.~Blucher} \affiliation{\Chicago}
\author{J.~Boissevain} \affiliation{\LosAlmos}
\author{S.~Bolognesi} \affiliation{\CEASaclay}
\author{T.~Bolton} \affiliation{\Kansasstate}
\author{L.~Bomben} \affiliation{\INFNMilanBicocca}\affiliation{\Insubria }
\author{M.~Bonesini} \affiliation{\INFNMilanBicocca}\affiliation{\MilanoBicocca}
\author{M.~Bongrand} \affiliation{\Parissaclay}
\author{F.~Bonini} \affiliation{\Brookhaven}
\author{A.~Booth} \affiliation{\QMUL}
\author{C.~Booth} \affiliation{\Sheffield}
\author{F.~Boran} \affiliation{\Beykent}
\author{S.~Bordoni} \affiliation{\CERN}
\author{A.~Borkum} \affiliation{\Sussex}
\author{T.~Boschi} \affiliation{\Durham}
\author{N.~Bostan} \affiliation{\Iowa}\affiliation{\NotreDame}
\author{P.~Bour} \affiliation{\CzechTechnical}
\author{C.~Bourgeois} \affiliation{\Parissaclay}
\author{S.~B.~Boyd} \affiliation{\Warwick}
\author{D.~Boyden} \affiliation{\Northernillinois}
\author{J.~Bracinik} \affiliation{\Birmingham}
\author{D.~Braga} \affiliation{\Fermi}
\author{D.~Brailsford} \affiliation{\Lancaster}
\author{A.~Branca} \affiliation{\INFNMilanBicocca}
\author{A.~Brandt} \affiliation{\TexasArlington}
\author{J.~Bremer} \affiliation{\CERN}
\author{C.~Brew} \affiliation{\Rutherford}
\author{E.~Brianne} \affiliation{\Manchester}
\author{S.~J.~Brice} \affiliation{\Fermi}
\author{C.~Brizzolari} \affiliation{\INFNMilanBicocca}\affiliation{\MilanoBicocca}
\author{C.~Bromberg} \affiliation{\Michiganstate}
\author{G.~Brooijmans} \affiliation{\Columbia}
\author{J.~Brooke} \affiliation{\Bristol}
\author{A.~Bross} \affiliation{\Fermi}
\author{G.~Brunetti} \affiliation{\INFNMilanBicocca}\affiliation{\MilanoBicocca}
\author{M.~Brunetti} \affiliation{\Warwick}
\author{N.~Buchanan} \affiliation{\ColoradoState}
\author{H.~Budd} \affiliation{\Rochester}
\author{I.~Butorov} \affiliation{\JINR}
\author{I.~Cagnoli} \affiliation{\INFNBologna}\affiliation{\BolognaUniversity}
\author{D.~Caiulo} \affiliation{\IPLyon}
\author{R.~Calabrese} \affiliation{\INFNFerrara}\affiliation{\Ferrarauniv}
\author{P.~Calafiura} \affiliation{\LawrenceBerkeley}
\author{J.~Calcutt} \affiliation{\Michiganstate}
\author{M.~Calin} \affiliation{\Bucharest}
\author{S.~Calvez} \affiliation{\ColoradoState}
\author{E.~Calvo} \affiliation{\CIEMAT}
\author{A.~Caminata} \affiliation{\INFNGenova}
\author{M.~Campanelli} \affiliation{\UniversityCollegeLondon}
\author{K.~Cankocak} \affiliation{\Iowa}
\author{D.~Caratelli} \affiliation{\Fermi}
\author{G.~Carini} \affiliation{\Brookhaven}
\author{B.~Carlus} \affiliation{\IPLyon}
\author{M.~F.~Carneiro} \affiliation{\Brookhaven}
\author{P.~Carniti} \affiliation{\INFNMilanBicocca}
\author{I.~Caro Terrazas} \affiliation{\ColoradoState}
\author{H.~Carranza} \affiliation{\TexasArlington}
\author{T.~Carroll} \affiliation{\Wisconsin}
\author{J.~F.~Casta{\~n}o Forero} \affiliation{\AntonioNarino}
\author{A.~Castillo} \affiliation{\SergioArboleda}
\author{C.~Castromonte} \affiliation{\Ingenieria}
\author{E.~Catano-Mur} \affiliation{\WilliamMary}
\author{C.~Cattadori} \affiliation{\INFNMilanBicocca}
\author{F.~Cavalier} \affiliation{\Parissaclay}
\author{F.~Cavanna} \affiliation{\Fermi}
\author{S.~Centro} \affiliation{\Padova}
\author{G.~Cerati} \affiliation{\Fermi}
\author{A.~Cervelli} \affiliation{\INFNBologna}
\author{A.~Cervera Villanueva} \affiliation{\IFIC}
\author{M.~Chalifour} \affiliation{\CERN}
\author{A.~Chappell} \affiliation{\Warwick}
\author{E.~Chardonnet} \affiliation{\Parisuniversite}
\author{N.~Charitonidis} \affiliation{\CERN}
\author{A.~Chatterjee} \affiliation{\Pitt}
\author{S.~Chattopadhyay} \affiliation{\VariableEnergy}
\author{H.~Chen} \affiliation{\Brookhaven}
\author{M.~Chen} \affiliation{\CalIrvine}
\author{Y.~Chen} \affiliation{\Bern}
\author{Z.~Chen} \affiliation{\StonyBrook}
\author{Y.~Cheon} \affiliation{\UNIST}
\author{D.~Cherdack} \affiliation{\Houston}
\author{C.~Chi} \affiliation{\Columbia}
\author{S.~Childress} \affiliation{\Fermi}
\author{A.~Chiriacescu} \affiliation{\Bucharest}
\author{G.~Chisnall} \affiliation{\Sussex}
\author{K.~Cho} \affiliation{\KISTI}
\author{S.~Choate} \affiliation{\Northernillinois}
\author{D.~Chokheli} \affiliation{\Georgian}
\author{P.~S.~Chong} \affiliation{\Penn}
\author{S.~Choubey} \affiliation{\Harish}
\author{A.~Christensen} \affiliation{\ColoradoState}
\author{D.~Christian} \affiliation{\Fermi}
\author{G.~Christodoulou} \affiliation{\CERN}
\author{A.~Chukanov} \affiliation{\JINR}
\author{M.~Chung} \affiliation{\UNIST}
\author{E.~Church} \affiliation{\PacificNorthwest}
\author{V.~Cicero} \affiliation{\INFNBologna}\affiliation{\BolognaUniversity}
\author{P.~Clarke} \affiliation{\Edinburgh}
\author{T.~E.~Coan} \affiliation{\SouthernMethodist}
\author{A.~G.~Cocco} \affiliation{\INFNNapoli}
\author{J.~A.~B.~Coelho} \affiliation{\Parisuniversite}
\author{E.~Conley} \affiliation{\Duke}
\author{R.~Conley} \affiliation{\SLAC}
\author{J.~M.~Conrad} \affiliation{\Massinsttech}
\author{M.~Convery} \affiliation{\SLAC}
\author{S.~Copello} \affiliation{\INFNGenova}
\author{L.~Corwin} \affiliation{\SouthDakotaSchool}
\author{R.~Valentim} \affiliation{\Unifesp}
\author{L.~Cremaldi} \affiliation{\Mississippi}
\author{L.~Cremonesi} \affiliation{\QMUL}
\author{J.~I.~Crespo-Anadón} \affiliation{\CIEMAT}
\author{M.~Crisler} \affiliation{\Fermi}
\author{E.~Cristaldo} \affiliation{\Asuncion}
\author{R.~Cross} \affiliation{\Lancaster}
\author{A.~Cudd} \affiliation{\ColoradoBoulder}
\author{C.~Cuesta} \affiliation{\CIEMAT}
\author{Y.~Cui} \affiliation{\CalRiverside}
\author{D.~Cussans} \affiliation{\Bristol}
\author{O.~Dalager} \affiliation{\CalIrvine}
\author{H.~da Motta} \affiliation{\CBPF}
\author{L.~Da Silva Peres} \affiliation{\FederaldoRio}
\author{C.~David} \affiliation{\York}\affiliation{\Fermi}
\author{Q.~David} \affiliation{\IPLyon}
\author{G.~S.~Davies} \affiliation{\Mississippi}
\author{S.~Davini} \affiliation{\INFNGenova}
\author{J.~Dawson} \affiliation{\Parisuniversite}
\author{K.~De} \affiliation{\TexasArlington}
\author{P.~Debbins} \affiliation{\Iowa}
\author{I.~De Bonis} \affiliation{\DannecyleVieux}
\author{M.~P.~Decowski} \affiliation{\Nikhef}\affiliation{\Amsterdam}
\author{A.~de Gouv\^ea} \affiliation{\Northwestern}
\author{P.~C.~De Holanda} \affiliation{\Campinas}
\author{I.~L.~De Icaza Astiz} \affiliation{\Sussex}
\author{A.~Deisting} \affiliation{\Royalholloway}
\author{P.~De Jong} \affiliation{\Nikhef}\affiliation{\Amsterdam}
\author{A.~Delbart} \affiliation{\CEASaclay}
\author{D.~Delepine} \affiliation{\Guanajuato}
\author{M.~Delgado} \affiliation{\AntonioNarino}
\author{A.~Dell’Acqua} \affiliation{\CERN}
\author{P.~De Lurgio} \affiliation{\Argonne}
\author{J.~R.~T.~de Mello Neto} \affiliation{\FederaldoRio}
\author{D.~M.~DeMuth} \affiliation{\ValleyCity}
\author{S.~Dennis} \affiliation{\Cambridge}
\author{C.~Densham} \affiliation{\Rutherford}
\author{G.~W.~Deptuch} \affiliation{\Brookhaven}
\author{A.~De Roeck} \affiliation{\CERN}
\author{V.~De Romeri} \affiliation{\IFIC}
\author{G.~De Souza} \affiliation{\Campinas}
\author{R.~Devi} \affiliation{\Jammu}
\author{R.~Dharmapalan} \affiliation{\Hawaii}
\author{M.~Dias} \affiliation{\Unifesp}
\author{F.~Diaz} \affiliation{\Pontificia}
\author{J.~S.~D\'iaz} \affiliation{\Indiana}
\author{S.~Di Domizio} \affiliation{\INFNGenova}\affiliation{\Genova}
\author{L.~Di Giulio} \affiliation{\CERN}
\author{P.~Ding} \affiliation{\Fermi}
\author{L.~Di Noto} \affiliation{\INFNGenova}\affiliation{\Genova}
\author{C.~Distefano} \affiliation{\INFNSud}
\author{R.~Diurba} \affiliation{\Minntwin}
\author{M.~Diwan} \affiliation{\Brookhaven}
\author{Z.~Djurcic} \affiliation{\Argonne}
\author{D.~Doering} \affiliation{\SLAC}
\author{S.~Dolan} \affiliation{\CERN}
\author{F.~Dolek} \affiliation{\Beykent}
\author{M.~J.~Dolinski} \affiliation{\Drexel}
\author{L.~Domine} \affiliation{\SLAC}
\author{D.~Douglas} \affiliation{\Michiganstate}
\author{D.~Douillet} \affiliation{\Parissaclay}
\author{G.~Drake} \affiliation{\Fermi}
\author{F.~Drielsma} \affiliation{\SLAC}
\author{L.~Duarte} \affiliation{\Unifesp}
\author{D.~Duchesneau} \affiliation{\DannecyleVieux}
\author{K.~Duffy} \affiliation{\Fermi}
\author{P.~Dunne} \affiliation{\Imperial}
\author{T.~Durkin} \affiliation{\Rutherford}
\author{H.~Duyang} \affiliation{\Southcarolina}
\author{O.~Dvornikov} \affiliation{\Hawaii}
\author{D.~A.~Dwyer} \affiliation{\LawrenceBerkeley}
\author{A.~S.~Dyshkant} \affiliation{\Northernillinois}
\author{M.~Eads} \affiliation{\Northernillinois}
\author{A.~Earle} \affiliation{\Sussex}
\author{D.~Edmunds} \affiliation{\Michiganstate}
\author{J.~Eisch} \affiliation{\Fermi}
\author{L.~Emberger} \affiliation{\Manchester}\affiliation{\Maxplanck}
\author{S.~Emery} \affiliation{\CEASaclay}
\author{A.~Ereditato} \affiliation{\Yale}
\author{T.~Erjavec} \affiliation{\CalDavis}
\author{C.~O.~Escobar} \affiliation{\Fermi}
\author{G.~Eurin} \affiliation{\CEASaclay}
\author{J.~J.~Evans} \affiliation{\Manchester}
\author{E.~Ewart} \affiliation{\Indiana}
\author{A.~C.~Ezeribe} \affiliation{\Sheffield}
\author{K.~Fahey} \affiliation{\Fermi}
\author{A.~Falcone} \affiliation{\INFNMilanBicocca}\affiliation{\MilanoBicocca}
\author{M.~Fani'} \affiliation{\LosAlmos}
\author{C.~Farnese} \affiliation{\INFNPadova}
\author{Y.~Farzan} \affiliation{\IPM}
\author{D.~Fedoseev} \affiliation{\JINR}
\author{J.~Felix} \affiliation{\Guanajuato}
\author{Y.~Feng} \affiliation{\IowaState}
\author{E.~Fernandez-Martinez} \affiliation{\Madrid}
\author{P.~Fernandez Menendez} \affiliation{\IFIC}
\author{M.~Fernandez Morales} \affiliation{\IGFAE}
\author{F.~Ferraro} \affiliation{\INFNGenova}\affiliation{\Genova}
\author{L.~Fields} \affiliation{\NotreDame}
\author{P.~Filip} \affiliation{\CzechAcademyofSciences}
\author{F.~Filthaut} \affiliation{\Nikhef}\affiliation{\Radboud}
\author{A.~Fiorentini} \affiliation{\SouthDakotaSchool}
\author{M.~Fiorini} \affiliation{\INFNFerrara}\affiliation{\Ferrarauniv}
\author{R.~S.~Fitzpatrick} \affiliation{\Michigan}
\author{W.~Flanagan} \affiliation{\Dallas}
\author{B.~Fleming} \affiliation{\Yale}
\author{R.~Flight} \affiliation{\Rochester}
\author{D.~V.~Forero} \affiliation{\Medellin}
\author{J.~Fowler} \affiliation{\Duke}
\author{W.~Fox} \affiliation{\Indiana}
\author{J.~Franc} \affiliation{\CzechTechnical}
\author{K.~Francis} \affiliation{\Northernillinois}
\author{D.~Franco} \affiliation{\Yale}
\author{J.~Freeman} \affiliation{\Fermi}
\author{J.~Freestone} \affiliation{\Manchester}
\author{J.~Fried} \affiliation{\Brookhaven}
\author{A.~Friedland} \affiliation{\SLAC}
\author{F.~Fuentes Robayo} \affiliation{\Bristol}
\author{S.~Fuess} \affiliation{\Fermi}
\author{I.~K.~Furic} \affiliation{\Florida}
\author{A.~P.~Furmanski} \affiliation{\Minntwin}
\author{A.~Gabrielli} \affiliation{\INFNBologna}
\author{A.~Gago} \affiliation{\Pontificia}
\author{H.~Gallagher} \affiliation{\Tufts}
\author{A.~Gallas} \affiliation{\Parissaclay}
\author{A.~Gallego-Ros} \affiliation{\CIEMAT}
\author{N.~Gallice} \affiliation{\INFNMilano}\affiliation{\MilanoUniv}
\author{V.~Galymov} \affiliation{\IPLyon}
\author{E.~Gamberini} \affiliation{\CERN}
\author{T.~Gamble} \affiliation{\Sheffield}
\author{F.~Ganacim} \affiliation{\Tecnologica }
\author{R.~Gandhi} \affiliation{\Harish}
\author{R.~Gandrajula} \affiliation{\Michiganstate}
\author{F.~Gao} \affiliation{\Pitt}
\author{S.~Gao} \affiliation{\Brookhaven}
\author{A.~C.~Garcia~B.} \affiliation{\Campinas}
\author{D.~Garcia-Gamez} \affiliation{\Granada}
\author{M.~Á.~García-Peris} \affiliation{\IFIC}
\author{S.~Gardiner} \affiliation{\Fermi}
\author{D.~Gastler} \affiliation{\Boston}
\author{J.~Gauvreau} \affiliation{\Occidental}
\author{G.~Ge} \affiliation{\Columbia}
\author{B.~Gelli} \affiliation{\Campinas}
\author{A.~Gendotti} \affiliation{\ETH}
\author{S.~Gent} \affiliation{\SouthDakotaState}
\author{Z.~Ghorbani-Moghaddam} \affiliation{\INFNGenova}
\author{P.~Giammaria} \affiliation{\Campinas}
\author{T.~Giammaria} \affiliation{\INFNFerrara}\affiliation{\Ferrarauniv}
\author{D.~Gibin} \affiliation{\Padova}
\author{I.~Gil-Botella} \affiliation{\CIEMAT}
\author{S.~Gilligan} \affiliation{\OregonState}
\author{C.~Girerd} \affiliation{\IPLyon}
\author{A.~K.~Giri} \affiliation{\IndHyderabad}
\author{D.~Gnani} \affiliation{\LawrenceBerkeley}
\author{O.~Gogota} \affiliation{\Kyiv}
\author{M.~Gold} \affiliation{\Newmexico}
\author{S.~Gollapinni} \affiliation{\LosAlmos}
\author{K.~Gollwitzer} \affiliation{\Fermi}
\author{R.~A.~Gomes} \affiliation{\FederaldeGoias}
\author{L.~V.~Gomez Bermeo} \affiliation{\SergioArboleda}
\author{L.~S.~Gomez Fajardo} \affiliation{\SergioArboleda}
\author{F.~Gonnella} \affiliation{\Birmingham}
\author{J.~A.~Gonzalez-Cuevas} \affiliation{\Asuncion}
\author{D.~Gonzalez Diaz} \affiliation{\IGFAE}
\author{M.~Gonzalez-Lopez} \affiliation{\Madrid}
\author{M.~C.~Goodman} \affiliation{\Argonne}
\author{O.~Goodwin} \affiliation{\Manchester}
\author{S.~Goswami} \affiliation{\PhysicalResearchLaboratory}
\author{C.~Gotti} \affiliation{\INFNMilanBicocca}
\author{E.~Goudzovski} \affiliation{\Birmingham}
\author{C.~Grace} \affiliation{\LawrenceBerkeley}
\author{M.~Graham} \affiliation{\SLAC}
\author{R.~Gran} \affiliation{\Minnduluth}
\author{E.~Granados} \affiliation{\Guanajuato}
\author{P.~Granger} \affiliation{\CEASaclay}
\author{A.~Grant} \affiliation{\Daresbury}
\author{C.~Grant} \affiliation{\Boston}
\author{D.~Gratieri} \affiliation{\Fluminense}
\author{P.~Green} \affiliation{\Manchester}
\author{L.~Greenler} \affiliation{\Wisconsin}
\author{J.~Greer} \affiliation{\Bristol}
\author{J.~Grenard} \affiliation{\CERN}
\author{W.~C.~Griffith} \affiliation{\Sussex}
\author{M.~Groh} \affiliation{\ColoradoState}
\author{J.~Grudzinski} \affiliation{\Argonne}
\author{K.~Grzelak} \affiliation{\Warsaw}
\author{W.~Gu} \affiliation{\Brookhaven}
\author{E.~Guardincerri} \affiliation{\LosAlmos}
\author{V.~Guarino} \affiliation{\Argonne}
\author{M.~Guarise} \affiliation{\INFNFerrara}\affiliation{\Ferrarauniv}
\author{R.~Guenette} \affiliation{\Harvard}
\author{E.~Guerard} \affiliation{\Parissaclay}
\author{M.~Guerzoni} \affiliation{\INFNBologna}
\author{A.~Guglielmi} \affiliation{\INFNPadova}
\author{B.~Guo} \affiliation{\Southcarolina}
\author{K.~K.~Guthikonda} \affiliation{\KL}
\author{R.~Gutierrez} \affiliation{\AntonioNarino}
\author{P.~Guzowski} \affiliation{\Manchester}
\author{M.~M.~Guzzo} \affiliation{\Campinas}
\author{S.~Gwon} \affiliation{\ChungAng}
\author{C.~Ha} \affiliation{\ChungAng}
\author{A.~Habig} \affiliation{\Minnduluth}
\author{H.~Hadavand} \affiliation{\TexasArlington}
\author{R.~Haenni} \affiliation{\Bern}
\author{A.~Hahn} \affiliation{\Fermi}
\author{J.~Haiston} \affiliation{\SouthDakotaSchool}
\author{P.~Hamacher-Baumann} \affiliation{\Oxford}
\author{T.~Hamernik} \affiliation{\Fermi}
\author{P.~Hamilton} \affiliation{\Imperial}
\author{J.~Han} \affiliation{\Pitt}
\author{D.~A.~Harris} \affiliation{\York}\affiliation{\Fermi}
\author{J.~Hartnell} \affiliation{\Sussex}
\author{J.~Harton} \affiliation{\ColoradoState}
\author{T.~Hasegawa} \affiliation{\KEK}
\author{C.~Hasnip} \affiliation{\Oxford}
\author{R.~Hatcher} \affiliation{\Fermi}
\author{K.~W.~Hatfield} \affiliation{\CalIrvine}
\author{A.~Hatzikoutelis} \affiliation{\Sanjosestate}
\author{C.~Hayes} \affiliation{\Indiana}
\author{K.~Hayrapetyan} \affiliation{\QMUL}
\author{J.~Hays} \affiliation{\QMUL}
\author{E.~Hazen} \affiliation{\Boston}
\author{M.~He} \affiliation{\Houston}
\author{A.~Heavey} \affiliation{\Fermi}
\author{K.~M.~Heeger} \affiliation{\Yale}
\author{J.~Heise} \affiliation{\SURF}
\author{K.~Hennessy} \affiliation{\Liverpool}
\author{S.~Henry} \affiliation{\Rochester}
\author{M.~A.~Hernandez Morquecho} \affiliation{\Illinoisinstitute}
\author{K.~Herner} \affiliation{\Fermi}
\author{L.~Hertel} \affiliation{\CalIrvine}
\author{J.~Hewes} \affiliation{\Cincinnati}
\author{A.~Higuera} \affiliation{\Houston}
\author{T.~Hill} \affiliation{\Idaho}
\author{S.~J.~Hillier} \affiliation{\Birmingham}
\author{A.~Himmel} \affiliation{\Fermi}
\author{L.R.~Hirsch} \affiliation{\Tecnologica }
\author{J.~Ho} \affiliation{\Harvard}
\author{J.~Hoff} \affiliation{\Fermi}
\author{A.~Holin} \affiliation{\Rutherford}
\author{E.~Hoppe} \affiliation{\PacificNorthwest}
\author{G.~A.~Horton-Smith} \affiliation{\Kansasstate}
\author{M.~Hostert} \affiliation{\Minntwin}
\author{A.~Hourlier} \affiliation{\Massinsttech}
\author{B.~Howard} \affiliation{\Fermi}
\author{R.~Howell} \affiliation{\Rochester}
\author{I.~Hristova} \affiliation{\Rutherford}
\author{M.~S.~Hronek} \affiliation{\Fermi}
\author{J.~Huang} \affiliation{\Texasaustin}
\author{J.~Huang} \affiliation{\CalDavis}
\author{J.~Hugon} \affiliation{\Louisanastate}
\author{G.~Iles} \affiliation{\Imperial}
\author{N.~Ilic} \affiliation{\Toronto}
\author{A.~M.~Iliescu} \affiliation{\INFNBologna}
\author{R.~Illingworth} \affiliation{\Fermi}
\author{G.~Ingratta} \affiliation{\INFNBologna}\affiliation{\BolognaUniversity}
\author{A.~Ioannisian} \affiliation{\Yerevan}
\author{L.~Isenhower} \affiliation{\Abilene}
\author{R.~Itay} \affiliation{\SLAC}
\author{A.~Izmaylov} \affiliation{\IFIC}
\author{C.M.~Jackson} \affiliation{\PacificNorthwest}
\author{V.~Jain} \affiliation{\Albanysuny}
\author{E.~James} \affiliation{\Fermi}
\author{W.~Jang} \affiliation{\TexasArlington}
\author{B.~Jargowsky} \affiliation{\CalIrvine}
\author{F.~Jediny} \affiliation{\CzechTechnical}
\author{D.~Jena} \affiliation{\Fermi}
\author{Y.~S.~Jeong} \affiliation{\ChungAng}\affiliation{\Iowa}
\author{C.~Jes\'{u}s-Valls} \affiliation{\IFAE}
\author{X.~Ji} \affiliation{\Brookhaven}
\author{L.~Jiang} \affiliation{\VirginiaTech}
\author{S.~Jiménez} \affiliation{\CIEMAT}
\author{A.~Jipa} \affiliation{\Bucharest}
\author{R.~Johnson} \affiliation{\Cincinnati}
\author{N.~Johnston} \affiliation{\Indiana}
\author{B.~Jones} \affiliation{\TexasArlington}
\author{S.~B.~Jones} \affiliation{\UniversityCollegeLondon}
\author{M.~Judah} \affiliation{\Pitt}
\author{C.~K.~Jung} \affiliation{\StonyBrook}
\author{T.~Junk} \affiliation{\Fermi}
\author{Y.~Jwa} \affiliation{\Columbia}
\author{M.~Kabirnezhad} \affiliation{\Oxford}
\author{A.~Kaboth} \affiliation{\Royalholloway}\affiliation{\Rutherford}
\author{I.~Kadenko} \affiliation{\Kyiv}
\author{D.~Kaira} \affiliation{\Columbia}
\author{I.~Kakorin} \affiliation{\JINR}
\author{A.~Kalitkina} \affiliation{\JINR}
\author{F.~Kamiya} \affiliation{\FederaldoABC}
\author{N.~Kaneshige} \affiliation{\CalSantabarbara}
\author{G.~Karagiorgi} \affiliation{\Columbia}
\author{G.~Karaman} \affiliation{\Iowa}
\author{A.~Karcher} \affiliation{\LawrenceBerkeley}
\author{M.~Karolak} \affiliation{\CEASaclay}
\author{Y.~Karyotakis} \affiliation{\DannecyleVieux}
\author{S.~Kasai} \affiliation{\Kure}
\author{S.~P.~Kasetti} \affiliation{\Louisanastate}
\author{L.~Kashur} \affiliation{\ColoradoState}
\author{N.~Kazaryan} \affiliation{\Yerevan}
\author{E.~Kearns} \affiliation{\Boston}
\author{P.~Keener} \affiliation{\Penn}
\author{K.J.~Kelly} \affiliation{\Fermi}
\author{E.~Kemp} \affiliation{\Campinas}
\author{O.~Kemularia} \affiliation{\Georgian}
\author{W.~Ketchum} \affiliation{\Fermi}
\author{S.~H.~Kettell} \affiliation{\Brookhaven}
\author{M.~Khabibullin} \affiliation{\INR}
\author{A.~Khotjantsev} \affiliation{\INR}
\author{A.~Khvedelidze} \affiliation{\Georgian}
\author{D.~Kim} \affiliation{\TexasAMcollege}
\author{B.~King} \affiliation{\Fermi}
\author{B.~Kirby} \affiliation{\Columbia}
\author{M.~Kirby} \affiliation{\Fermi}
\author{J.~Klein} \affiliation{\Penn}
\author{K.~Koehler} \affiliation{\Wisconsin}
\author{L.~W.~Koerner} \affiliation{\Houston}
\author{S.~Kohn} \affiliation{\CalBerkeley}\affiliation{\LawrenceBerkeley}
\author{P.~P.~Koller} \affiliation{\Bern}
\author{L.~Kolupaeva} \affiliation{\JINR}
\author{D.~Korablev} \affiliation{\JINR}
\author{M.~Kordosky} \affiliation{\WilliamMary}
\author{T.~Kosc} \affiliation{\IPLyon}
\author{U.~Kose} \affiliation{\CERN}
\author{V.~A.~Kosteleck\'y} \affiliation{\Indiana}
\author{K.~Kothekar} \affiliation{\Bristol}
\author{F.~Krennrich} \affiliation{\IowaState}
\author{I.~Kreslo} \affiliation{\Bern}
\author{W.~Kropp} \affiliation{\CalIrvine}
\author{Y.~Kudenko} \affiliation{\INR}
\author{V.~A.~Kudryavtsev} \affiliation{\Sheffield}
\author{S.~Kulagin} \affiliation{\INR}
\author{J.~Kumar} \affiliation{\Hawaii}
\author{P.~Kumar} \affiliation{\Sheffield}
\author{P.~Kunze} \affiliation{\DannecyleVieux}
\author{C.~Kuruppu} \affiliation{\Southcarolina}
\author{V.~Kus} \affiliation{\CzechTechnical}
\author{T.~Kutter} \affiliation{\Louisanastate}
\author{J.~Kvasnicka} \affiliation{\CzechAcademyofSciences}
\author{D.~Kwak} \affiliation{\UNIST}
\author{A.~Lambert} \affiliation{\LawrenceBerkeley}
\author{B.~J.~Land} \affiliation{\Penn}
\author{K.~Lande} \affiliation{\Penn}
\author{C.~E.~Lane} \affiliation{\Drexel}
\author{K.~Lang} \affiliation{\Texasaustin}
\author{T.~Langford} \affiliation{\Yale}
\author{M.~Langstaff} \affiliation{\Manchester}
\author{J.~Larkin} \affiliation{\Brookhaven}
\author{P.~Lasorak} \affiliation{\Sussex}
\author{D.~Last} \affiliation{\Penn}
\author{C.~Lastoria} \affiliation{\CIEMAT}
\author{A.~Laundrie} \affiliation{\Wisconsin}
\author{G.~Laurenti} \affiliation{\INFNBologna}
\author{A.~Lawrence} \affiliation{\LawrenceBerkeley}
\author{I.~Lazanu} \affiliation{\Bucharest}
\author{R.~LaZur} \affiliation{\ColoradoState}
\author{M.~Lazzaroni} \affiliation{\INFNMilano}\affiliation{\MilanoUniv}
\author{T.~Le} \affiliation{\Tufts}
\author{S.~Leardini} \affiliation{\IGFAE}
\author{J.~Learned} \affiliation{\Hawaii}
\author{P.~LeBrun} \affiliation{\IPLyon}
\author{T.~LeCompte} \affiliation{\Argonne}
\author{C.~Lee} \affiliation{\Fermi}
\author{S.~Y.~Lee} \affiliation{\Jeonbuk}
\author{G.~Lehmann Miotto} \affiliation{\CERN}
\author{R.~Lehnert} \affiliation{\Indiana}
\author{M.~A.~Leigui de Oliveira} \affiliation{\FederaldoABC}
\author{M.~Leitner} \affiliation{\LawrenceBerkeley}
\author{L.~M.~Lepin} \affiliation{\Manchester}
\author{L.~Li} \affiliation{\CalIrvine}
\author{S.~W.~Li} \affiliation{\SLAC}
\author{T.~Li} \affiliation{\Edinburgh}
\author{Y.~Li} \affiliation{\Brookhaven}
\author{H.~Liao} \affiliation{\Kansasstate}
\author{C.~S.~Lin} \affiliation{\LawrenceBerkeley}
\author{Q.~Lin} \affiliation{\SLAC}
\author{S.~Lin} \affiliation{\Louisanastate}
\author{J.~Ling} \affiliation{\Sunyatsen}
\author{A.~Lister} \affiliation{\Wisconsin}
\author{B.~R.~Littlejohn} \affiliation{\Illinoisinstitute}
\author{J.~Liu} \affiliation{\CalIrvine}
\author{S.~Lockwitz} \affiliation{\Fermi}
\author{T.~Loew} \affiliation{\LawrenceBerkeley}
\author{M.~Lokajicek} \affiliation{\CzechAcademyofSciences}
\author{I.~Lomidze} \affiliation{\Georgian}
\author{K.~Long} \affiliation{\Imperial}
\author{K.~Loo} \affiliation{\Jyvaskyla}
\author{T.~Lord} \affiliation{\Warwick}
\author{J.~M.~LoSecco} \affiliation{\NotreDame}
\author{W.~C.~Louis} \affiliation{\LosAlmos}
\author{X.-G.~Lu} \affiliation{\Oxford}
\author{K.B.~Luk} \affiliation{\CalBerkeley}\affiliation{\LawrenceBerkeley}
\author{X.~Luo} \affiliation{\CalSantabarbara}
\author{E.~Luppi} \affiliation{\INFNFerrara}\affiliation{\Ferrarauniv}
\author{N.~Lurkin} \affiliation{\Birmingham}
\author{T.~Lux} \affiliation{\IFAE}
\author{V.~P.~Luzio} \affiliation{\FederaldoABC}
\author{D.~MacFarlane} \affiliation{\SLAC}
\author{A.~A.~Machado} \affiliation{\Campinas}
\author{P.~Machado} \affiliation{\Fermi}
\author{C.~T.~Macias} \affiliation{\Indiana}
\author{J.~R.~Macier} \affiliation{\Fermi}
\author{A.~Maddalena} \affiliation{\GranSassoLab}
\author{A.~Madera} \affiliation{\CERN}
\author{P.~Madigan} \affiliation{\CalBerkeley}\affiliation{\LawrenceBerkeley}
\author{S.~Magill} \affiliation{\Argonne}
\author{K.~Mahn} \affiliation{\Michiganstate}
\author{A.~Maio} \affiliation{\LIP}\affiliation{\FCULport}
\author{A.~Major} \affiliation{\Duke}
\author{J.~A.~Maloney} \affiliation{\DakotaState}
\author{G.~Mandrioli} \affiliation{\INFNBologna}
\author{R.~C.~Mandujano} \affiliation{\CalIrvine}
\author{J.~Maneira} \affiliation{\LIP}\affiliation{\FCULport}
\author{L.~Manenti} \affiliation{\UniversityCollegeLondon}
\author{S.~Manly} \affiliation{\Rochester}
\author{A.~Mann} \affiliation{\Tufts}
\author{K.~Manolopoulos} \affiliation{\Rutherford}
\author{M.~Manrique Plata} \affiliation{\Indiana}
\author{V.~N.~Manyam} \affiliation{\Brookhaven}
\author{L.~Manzanillas} \affiliation{\Parissaclay}
\author{M.~Marchan} \affiliation{\Fermi}
\author{A.~Marchionni} \affiliation{\Fermi}
\author{W.~Marciano} \affiliation{\Brookhaven}
\author{D.~Marfatia} \affiliation{\Hawaii}
\author{C.~Mariani} \affiliation{\VirginiaTech}
\author{J.~Maricic} \affiliation{\Hawaii}
\author{R.~Marie} \affiliation{\Parissaclay}
\author{F.~Marinho} \affiliation{\FederaldeSaoCarlos}
\author{A.~D.~Marino} \affiliation{\ColoradoBoulder}
\author{D.~Marsden} \affiliation{\Manchester}
\author{M.~Marshak} \affiliation{\Minntwin}
\author{C.~M.~Marshall} \affiliation{\Rochester}
\author{J.~Marshall} \affiliation{\Warwick}
\author{J.~Marteau} \affiliation{\IPLyon}
\author{J.~Martin-Albo} \affiliation{\IFIC}
\author{N.~Martinez} \affiliation{\Kansasstate}
\author{D.A.~Martinez Caicedo } \affiliation{\SouthDakotaSchool}
\author{S.~Martynenko} \affiliation{\StonyBrook}
\author{V.~Mascagna} \affiliation{\INFNMilanBicocca}\affiliation{\Insubria }
\author{K.~Mason} \affiliation{\Tufts}
\author{A.~Mastbaum} \affiliation{\Rutgers}
\author{M.~Masud} \affiliation{\IFIC}
\author{F.~Matichard} \affiliation{\LawrenceBerkeley}
\author{S.~Matsuno} \affiliation{\Hawaii}
\author{J.~Matthews} \affiliation{\Louisanastate}
\author{C.~Mauger} \affiliation{\Penn}
\author{N.~Mauri} \affiliation{\INFNBologna}\affiliation{\BolognaUniversity}
\author{K.~Mavrokoridis} \affiliation{\Liverpool}
\author{I.~Mawby} \affiliation{\Warwick}
\author{R.~Mazza} \affiliation{\INFNMilanBicocca}
\author{A.~Mazzacane} \affiliation{\Fermi}
\author{E.~Mazzucato} \affiliation{\CEASaclay}
\author{T.~McAskill} \affiliation{\Wellesley}
\author{E.~McCluskey} \affiliation{\Fermi}
\author{N.~McConkey} \affiliation{\Manchester}
\author{K.~S.~McFarland} \affiliation{\Rochester}
\author{C.~McGrew} \affiliation{\StonyBrook}
\author{A.~McNab} \affiliation{\Manchester}
\author{A.~Mefodiev} \affiliation{\INR}
\author{P.~Mehta} \affiliation{\Jawaharlal}
\author{P.~Melas} \affiliation{\Athens}
\author{O.~Mena} \affiliation{\IFIC}
\author{S.~Menary} \affiliation{\York}
\author{H.~Mendez} \affiliation{\PuertoRico}
\author{P.~Mendez} \affiliation{\CERN}
\author{D.~P.~M{\'e}ndez} \affiliation{\Brookhaven}
\author{A.~Menegolli} \affiliation{\INFNPavia}\affiliation{\Pavia}
\author{G.~Meng} \affiliation{\INFNPadova}
\author{M.~D.~Messier} \affiliation{\Indiana}
\author{W.~Metcalf} \affiliation{\Louisanastate}
\author{T.~Mettler} \affiliation{\Bern}
\author{M.~Mewes} \affiliation{\Indiana}
\author{H.~Meyer} \affiliation{\Wichita}
\author{T.~Miao} \affiliation{\Fermi}
\author{G.~Michna} \affiliation{\SouthDakotaState}
\author{T.~Miedema} \affiliation{\Nikhef}\affiliation{\Radboud}
\author{V.~Mikola} \affiliation{\UniversityCollegeLondon}
\author{R.~Milincic} \affiliation{\Hawaii}
\author{G.~Miller} \affiliation{\Manchester}
\author{W.~Miller} \affiliation{\Minntwin}
\author{J.~Mills} \affiliation{\Tufts}
\author{C.~Milne} \affiliation{\Idaho}
\author{O.~Mineev} \affiliation{\INR}
\author{O.~G.~Miranda} \affiliation{\Cinvestav}
\author{S.~Miryala} \affiliation{\Brookhaven}
\author{C.~S.~Mishra} \affiliation{\Fermi}
\author{S.~R.~Mishra} \affiliation{\Southcarolina}
\author{A.~Mislivec} \affiliation{\Minntwin}
\author{D.~Mladenov} \affiliation{\CERN}
\author{I.~Mocioiu} \affiliation{\PennState}
\author{K.~Moffat} \affiliation{\Durham}
\author{N.~Moggi} \affiliation{\INFNBologna}\affiliation{\BolognaUniversity}
\author{R.~Mohanta} \affiliation{\Hyderabad}
\author{T.~A.~Mohayai} \affiliation{\Fermi}
\author{N.~Mokhov} \affiliation{\Fermi}
\author{J.~Molina} \affiliation{\Asuncion}
\author{L.~Molina Bueno} \affiliation{\IFIC}
\author{E.~Montagna} \affiliation{\INFNBologna}\affiliation{\BolognaUniversity}
\author{A.~Montanari} \affiliation{\INFNBologna}
\author{C.~Montanari} \affiliation{\INFNPavia}\affiliation{\Fermi}\affiliation{\Pavia}
\author{D.~Montanari} \affiliation{\Fermi}
\author{L.~M.~Montano Zetina} \affiliation{\Cinvestav}
\author{J.~Moon} \affiliation{\Massinsttech}
\author{S.~H.~Moon} \affiliation{\UNIST}
\author{M.~Mooney} \affiliation{\ColoradoState}
\author{A.~F.~Moor} \affiliation{\Cambridge}
\author{D.~Moreno} \affiliation{\AntonioNarino}
\author{C.~Morris} \affiliation{\Houston}
\author{C.~Mossey} \affiliation{\Fermi}
\author{E.~Motuk} \affiliation{\UniversityCollegeLondon}
\author{C.~A.~Moura} \affiliation{\FederaldoABC}
\author{J.~Mousseau} \affiliation{\Michigan}
\author{G.~Mouster} \affiliation{\Lancaster}
\author{W.~Mu} \affiliation{\Fermi}
\author{L.~Mualem} \affiliation{\Caltech}
\author{J.~Mueller} \affiliation{\ColoradoState}
\author{M.~Muether} \affiliation{\Wichita}
\author{S.~Mufson} \affiliation{\Indiana}
\author{F.~Muheim} \affiliation{\Edinburgh}
\author{A.~Muir} \affiliation{\Daresbury}
\author{M.~Mulhearn} \affiliation{\CalDavis}
\author{D.~Munford} \affiliation{\Houston}
\author{H.~Muramatsu} \affiliation{\Minntwin}
\author{S.~Murphy} \affiliation{\ETH}
\author{J.~Musser} \affiliation{\Indiana}
\author{J.~Nachtman} \affiliation{\Iowa}
\author{S.~Nagu} \affiliation{\Lucknow}
\author{M.~Nalbandyan} \affiliation{\Yerevan}
\author{R.~Nandakumar} \affiliation{\Rutherford}
\author{D.~Naples} \affiliation{\Pitt}
\author{S.~Narita} \affiliation{\Iwate}
\author{A.~Nath} \affiliation{\IndGuwahati}
\author{D.~Navas-Nicolás} \affiliation{\CIEMAT}
\author{A.~Navrer-Agasson} \affiliation{\Manchester}
\author{N.~Nayak} \affiliation{\CalIrvine}
\author{M.~Nebot-Guinot} \affiliation{\Edinburgh}
\author{K.~Negishi} \affiliation{\Iwate}
\author{J.~K.~Nelson} \affiliation{\WilliamMary}
\author{J.~Nesbit} \affiliation{\Wisconsin}
\author{M.~Nessi} \affiliation{\CERN}
\author{D.~Newbold} \affiliation{\Rutherford}
\author{M.~Newcomer} \affiliation{\Penn}
\author{D.~Newhart} \affiliation{\Fermi}
\author{H.~Newton} \affiliation{\Daresbury}
\author{R.~Nichol} \affiliation{\UniversityCollegeLondon}
\author{F.~Nicolas-Arnaldos} \affiliation{\Granada}
\author{E.~Niner} \affiliation{\Fermi}
\author{K.~Nishimura} \affiliation{\Hawaii}
\author{A.~Norman} \affiliation{\Fermi}
\author{A.~Norrick} \affiliation{\Fermi}
\author{R.~Northrop} \affiliation{\Chicago}
\author{P.~Novella} \affiliation{\IFIC}
\author{J.~A.~Nowak} \affiliation{\Lancaster}
\author{M.~Oberling} \affiliation{\Argonne}
\author{J.~P.~Ochoa-Ricoux} \affiliation{\CalIrvine}
\author{A.~Olivares Del Campo} \affiliation{\Durham}
\author{A.~Olivier} \affiliation{\Rochester}
\author{A.~Olshevskiy} \affiliation{\JINR}
\author{Y.~Onel} \affiliation{\Iowa}
\author{Y.~Onishchuk} \affiliation{\Kyiv}
\author{J.~Ott} \affiliation{\CalIrvine}
\author{L.~Pagani} \affiliation{\CalDavis}
\author{S.~Pakvasa} \affiliation{\Hawaii}
\author{G.~Palacio} \affiliation{\EIA}
\author{O.~Palamara} \affiliation{\Fermi}
\author{S.~Palestini} \affiliation{\CERN}
\author{J.~M.~Paley} \affiliation{\Fermi}
\author{M.~Pallavicini} \affiliation{\INFNGenova}\affiliation{\Genova}
\author{C.~Palomares} \affiliation{\CIEMAT}
\author{J.~L.~Palomino-Gallo} \affiliation{\StonyBrook}
\author{W.~Panduro Vazquez} \affiliation{\Royalholloway}
\author{E.~Pantic} \affiliation{\CalDavis}
\author{V.~Paolone} \affiliation{\Pitt}
\author{V.~Papadimitriou} \affiliation{\Fermi}
\author{R.~Papaleo} \affiliation{\INFNSud}
\author{A.~Papanestis} \affiliation{\Rutherford}
\author{S.~Paramesvaran} \affiliation{\Bristol}
\author{S.~Parke} \affiliation{\Fermi}
\author{E.~Parozzi} \affiliation{\INFNMilanBicocca}\affiliation{\MilanoBicocca}
\author{Z.~Parsa} \affiliation{\Brookhaven}
\author{M.~Parvu} \affiliation{\Bucharest}
\author{S.~Pascoli} \affiliation{\Durham}\affiliation{\BolognaUniversity}
\author{L.~Pasqualini} \affiliation{\INFNBologna}\affiliation{\BolognaUniversity}
\author{J.~Pasternak} \affiliation{\Imperial}
\author{J.~Pater} \affiliation{\Manchester}
\author{C.~Patrick} \affiliation{\UniversityCollegeLondon}
\author{L.~Patrizii} \affiliation{\INFNBologna}
\author{R.~B.~Patterson} \affiliation{\Caltech}
\author{S.~J.~Patton} \affiliation{\LawrenceBerkeley}
\author{T.~Patzak} \affiliation{\Parisuniversite}
\author{A.~Paudel} \affiliation{\Fermi}
\author{B.~Paulos} \affiliation{\Wisconsin}
\author{L.~Paulucci} \affiliation{\FederaldoABC}
\author{Z.~Pavlovic} \affiliation{\Fermi}
\author{G.~Pawloski} \affiliation{\Minntwin}
\author{D.~Payne} \affiliation{\Liverpool}
\author{V.~Pec} \affiliation{\Sheffield}
\author{S.~J.~M.~Peeters} \affiliation{\Sussex}
\author{E.~Pennacchio} \affiliation{\IPLyon}
\author{A.~Penzo} \affiliation{\Iowa}
\author{O.~L.~G.~Peres} \affiliation{\Campinas}
\author{J.~Perry} \affiliation{\Edinburgh}
\author{D.~Pershey} \affiliation{\Duke}
\author{G.~Pessina} \affiliation{\INFNMilanBicocca}
\author{G.~Petrillo} \affiliation{\SLAC}
\author{C.~Petta} \affiliation{\INFNCatania}\affiliation{\CataniaUniversitadi}
\author{R.~Petti} \affiliation{\Southcarolina}
\author{F.~Piastra} \affiliation{\Bern}
\author{L.~Pickering} \affiliation{\Michiganstate}
\author{F.~Pietropaolo} \affiliation{\CERN}\affiliation{\INFNPadova}
\author{R.~Plunkett} \affiliation{\Fermi}
\author{R.~Poling} \affiliation{\Minntwin}
\author{X.~Pons} \affiliation{\CERN}
\author{N.~Poonthottathil} \affiliation{\IowaState}
\author{F.~Poppi} \affiliation{\INFNBologna}\affiliation{\BolognaUniversity}
\author{S.~Pordes} \affiliation{\Fermi}
\author{J.~Porter} \affiliation{\Sussex}
\author{M.~Potekhin} \affiliation{\Brookhaven}
\author{R.~Potenza} \affiliation{\INFNCatania}\affiliation{\CataniaUniversitadi}
\author{B.~V.~K.~S.~Potukuchi} \affiliation{\Jammu}
\author{J.~Pozimski} \affiliation{\Imperial}
\author{M.~Pozzato} \affiliation{\INFNBologna}\affiliation{\BolognaUniversity}
\author{S.~Prakash} \affiliation{\Campinas}
\author{T.~Prakash} \affiliation{\LawrenceBerkeley}
\author{M.~Prest} \affiliation{\INFNMilanBicocca}
\author{S.~Prince} \affiliation{\Harvard}
\author{F.~Psihas} \affiliation{\Fermi}
\author{D.~Pugnere} \affiliation{\IPLyon}
\author{X.~Qian} \affiliation{\Brookhaven}
\author{M.~C.~Queiroga Bazetto} \affiliation{\Campinas}
\author{J.~L.~Raaf} \affiliation{\Fermi}
\author{V.~Radeka} \affiliation{\Brookhaven}
\author{J.~Rademacker} \affiliation{\Bristol}
\author{B.~Radics} \affiliation{\ETH}
\author{A.~Rafique} \affiliation{\Argonne}
\author{E.~Raguzin} \affiliation{\Brookhaven}
\author{M.~Rai} \affiliation{\Warwick}
\author{M.~Rajaoalisoa} \affiliation{\Cincinnati}
\author{I.~Rakhno} \affiliation{\Fermi}
\author{A.~Rakotonandrasana} \affiliation{\Antananarivo}
\author{L.~Rakotondravohitra} \affiliation{\Antananarivo}
\author{Y.~A.~Ramachers} \affiliation{\Warwick}
\author{R.~Rameika} \affiliation{\Fermi}
\author{M.~A.~Ramirez Delgado} \affiliation{\Penn}
\author{B.~Ramson} \affiliation{\Fermi}
\author{A.~Rappoldi} \affiliation{\INFNPavia}\affiliation{\Pavia}
\author{G.~Raselli} \affiliation{\INFNPavia}\affiliation{\Pavia}
\author{P.~Ratoff} \affiliation{\Lancaster}
\author{S.~Raut} \affiliation{\StonyBrook}
\author{R.~F.~Razakamiandra} \affiliation{\Antananarivo}
\author{E.~Rea} \affiliation{\Minntwin}
\author{J.S.~Real} \affiliation{\Grenoble}
\author{B.~Rebel} \affiliation{\Wisconsin}\affiliation{\Fermi}
\author{M.~Reggiani-Guzzo} \affiliation{\Manchester}
\author{T.~Rehak} \affiliation{\Drexel}
\author{J.~Reichenbacher} \affiliation{\SouthDakotaSchool}
\author{S.~D.~Reitzner} \affiliation{\Fermi}
\author{H.~Rejeb Sfar} \affiliation{\CERN}
\author{A.~Renshaw} \affiliation{\Houston}
\author{S.~Rescia} \affiliation{\Brookhaven}
\author{F.~Resnati} \affiliation{\CERN}
\author{A.~Reynolds} \affiliation{\Oxford}
\author{M.~Ribas} \affiliation{\Tecnologica }
\author{S.~Riboldi} \affiliation{\INFNMilano}
\author{C.~Riccio} \affiliation{\StonyBrook}
\author{G.~Riccobene} \affiliation{\INFNSud}
\author{L.~C.~J.~Rice} \affiliation{\Pitt}
\author{J.~Ricol} \affiliation{\Grenoble}
\author{A.~Rigamonti} \affiliation{\CERN}
\author{Y.~Rigaut} \affiliation{\ETH}
\author{D.~Rivera} \affiliation{\Penn}
\author{A.~Robert} \affiliation{\Grenoble}
\author{L.~Rochester} \affiliation{\SLAC}
\author{M.~Roda} \affiliation{\Liverpool}
\author{P.~Rodrigues} \affiliation{\Oxford}
\author{M.~J.~Rodriguez Alonso} \affiliation{\CERN}
\author{E.~Rodriguez Bonilla} \affiliation{\AntonioNarino}
\author{J.~Rodriguez Rondon} \affiliation{\SouthDakotaSchool}
\author{S.~Rosauro-Alcaraz} \affiliation{\Madrid}
\author{M.~Rosenberg} \affiliation{\Pitt}
\author{P.~Rosier} \affiliation{\Parissaclay}
\author{B.~Roskovec} \affiliation{\CalIrvine}
\author{M.~Rossella} \affiliation{\INFNPavia}\affiliation{\Pavia}
\author{M.~Rossi} \affiliation{\CERN}
\author{J.~Rout} \affiliation{\Jawaharlal}
\author{P.~Roy} \affiliation{\Wichita}
\author{S.~Roy} \affiliation{\Harish}
\author{A.~Rubbia} \affiliation{\ETH}
\author{C.~Rubbia} \affiliation{\GranSasso}
\author{F.~C.~Rubio} \affiliation{\IFIC}
\author{B.~Russell} \affiliation{\LawrenceBerkeley}
\author{D.~Ruterbories} \affiliation{\Rochester}
\author{A.~Rybnikov} \affiliation{\JINR}
\author{A.~Saa-Hernandez} \affiliation{\IGFAE}
\author{R.~Saakyan} \affiliation{\UniversityCollegeLondon}
\author{S.~Sacerdoti} \affiliation{\Parisuniversite}
\author{T.~Safford} \affiliation{\Michiganstate}
\author{N.~Sahu} \affiliation{\IndHyderabad}
\author{P.~Sala} \affiliation{\INFNMilano}\affiliation{\CERN}
\author{N.~Samios} \affiliation{\Brookhaven}
\author{O.~Samoylov} \affiliation{\JINR}
\author{M.~C.~Sanchez} \affiliation{\IowaState}
\author{V.~Sandberg} \affiliation{\LosAlmos}
\author{D.~A.~Sanders} \affiliation{\Mississippi}
\author{D.~Sankey} \affiliation{\Rutherford}
\author{S.~Santana} \affiliation{\PuertoRico}
\author{M.~Santos-Maldonado} \affiliation{\PuertoRico}
\author{N.~Saoulidou} \affiliation{\Athens}
\author{P.~Sapienza} \affiliation{\INFNSud}
\author{C.~Sarasty} \affiliation{\Cincinnati}
\author{I.~Sarcevic} \affiliation{\Arizona}
\author{G.~Savage} \affiliation{\Fermi}
\author{V.~Savinov} \affiliation{\Pitt}
\author{A.~Scaramelli} \affiliation{\INFNPavia}
\author{A.~Scarff} \affiliation{\Sheffield}
\author{A.~Scarpelli} \affiliation{\Brookhaven}
\author{T.~Schaffer} \affiliation{\Minnduluth}
\author{H.~Schellman} \affiliation{\OregonState}\affiliation{\Fermi}
\author{S.~Schifano} \affiliation{\INFNFerrara}\affiliation{\Ferrarauniv}
\author{P.~Schlabach} \affiliation{\Fermi}
\author{D.~Schmitz} \affiliation{\Chicago}
\author{K.~Scholberg} \affiliation{\Duke}
\author{A.~Schukraft} \affiliation{\Fermi}
\author{E.~Segreto} \affiliation{\Campinas}
\author{A.~Selyunin} \affiliation{\JINR}
\author{C.~R.~Senise} \affiliation{\Unifesp}
\author{J.~Sensenig} \affiliation{\Penn}
\author{M.~Seoane} \affiliation{\IGFAE}
\author{I.~Seong} \affiliation{\CalIrvine}
\author{A.~Sergi} \affiliation{\Birmingham}
\author{D.~Sgalaberna} \affiliation{\ETH}
\author{M.~H.~Shaevitz} \affiliation{\Columbia}
\author{S.~Shafaq} \affiliation{\Jawaharlal}
\author{M.~Shamma} \affiliation{\CalRiverside}
\author{R.~Sharankova} \affiliation{\Tufts}
\author{H.~R.~Sharma} \affiliation{\Jammu}
\author{R.~Sharma} \affiliation{\Brookhaven}
\author{R.~Kumar} \affiliation{\Punjab}
\author{T.~Shaw} \affiliation{\Fermi}
\author{C.~Shepherd-Themistocleous} \affiliation{\Rutherford}
\author{A.~Sheshukov} \affiliation{\JINR}
\author{S.~Shin} \affiliation{\Jeonbuk}
\author{I.~Shoemaker} \affiliation{\VirginiaTech}
\author{D.~Shooltz} \affiliation{\Michiganstate}
\author{R.~Shrock} \affiliation{\StonyBrook}
\author{H.~Siegel} \affiliation{\Columbia}
\author{L.~Simard} \affiliation{\Parissaclay}
\author{F.~Simon} \affiliation{\Fermi}\affiliation{\Maxplanck}
\author{J.~Sinclair} \affiliation{\Bern}
\author{G.~Sinev} \affiliation{\SouthDakotaSchool}
\author{J.~Singh} \affiliation{\Lucknow}
\author{J.~Singh} \affiliation{\Lucknow}
\author{L.~Singh} \affiliation{\CUSB}
\author{V.~Singh} \affiliation{\CUSB}\affiliation{\Banaras}
\author{R.~Sipos} \affiliation{\CERN}
\author{F.~W.~Sippach} \affiliation{\Columbia}
\author{G.~Sirri} \affiliation{\INFNBologna}
\author{A.~Sitraka} \affiliation{\SouthDakotaSchool}
\author{K.~Siyeon} \affiliation{\ChungAng}
\author{K.~Skarpaas} \affiliation{\SLAC}
\author{A.~Smith} \affiliation{\Cambridge}
\author{E.~Smith} \affiliation{\Indiana}
\author{P.~Smith} \affiliation{\Indiana}
\author{J.~Smolik} \affiliation{\CzechTechnical}
\author{M.~Smy} \affiliation{\CalIrvine}
\author{E.L.~Snider} \affiliation{\Fermi}
\author{P.~Snopok} \affiliation{\Illinoisinstitute}
\author{D.~Snowden-Ifft} \affiliation{\Occidental}
\author{M.~Soares Nunes} \affiliation{\Syracuse}
\author{H.~Sobel} \affiliation{\CalIrvine}
\author{M.~Soderberg} \affiliation{\Syracuse}
\author{S.~Sokolov} \affiliation{\JINR}
\author{C.~J.~Solano Salinas} \affiliation{\Ingenieria}
\author{S.~Söldner-Rembold} \affiliation{\Manchester}
\author{S.R.~Soleti} \affiliation{\LawrenceBerkeley}
\author{N.~Solomey} \affiliation{\Wichita}
\author{V.~Solovov} \affiliation{\LIP}
\author{W.~E.~Sondheim} \affiliation{\LosAlmos}
\author{M.~Sorel} \affiliation{\IFIC}
\author{A.~Sotnikov} \affiliation{\JINR}
\author{J.~Soto-Oton} \affiliation{\CIEMAT}
\author{A.~Sousa} \affiliation{\Cincinnati}
\author{K.~Soustruznik} \affiliation{\Charles}
\author{F.~Spagliardi} \affiliation{\Oxford}
\author{M.~Spanu} \affiliation{\INFNMilanBicocca}\affiliation{\MilanoBicocca}
\author{J.~Spitz} \affiliation{\Michigan}
\author{N.~J.~C.~Spooner} \affiliation{\Sheffield}
\author{K.~Spurgeon} \affiliation{\Syracuse}
\author{R.~Staley} \affiliation{\Birmingham}
\author{M.~Stancari} \affiliation{\Fermi}
\author{L.~Stanco} \affiliation{\INFNPadova}\affiliation{\Padova}
\author{R.~Stanley} \affiliation{\Bristol}
\author{R.~Stein} \affiliation{\Bristol}
\author{H.~M.~Steiner} \affiliation{\LawrenceBerkeley}
\author{A.~F.~Steklain Lisbôa} \affiliation{\Tecnologica }
\author{J.~Stewart} \affiliation{\Brookhaven}
\author{B.~Stillwell} \affiliation{\Chicago}
\author{J.~Stock} \affiliation{\SouthDakotaSchool}
\author{F.~Stocker} \affiliation{\CERN}
\author{T.~Stokes} \affiliation{\Louisanastate}
\author{M.~Strait} \affiliation{\Minntwin}
\author{T.~Strauss} \affiliation{\Fermi}
\author{S.~Striganov} \affiliation{\Fermi}
\author{A.~Stuart} \affiliation{\Colima}
\author{J.~G.~Suarez} \affiliation{\EIA}
\author{H.~Sullivan} \affiliation{\TexasArlington}
\author{D.~Summers} \affiliation{\Mississippi}
\author{A.~Surdo} \affiliation{\INFNLecce}
\author{V.~Susic} \affiliation{\Basel}
\author{L.~Suter} \affiliation{\Fermi}
\author{C.~M.~Sutera} \affiliation{\INFNCatania}\affiliation{\CataniaUniversitadi}
\author{R.~Svoboda} \affiliation{\CalDavis}
\author{B.~Szczerbinska} \affiliation{\TexasAMcorpuscristi}
\author{A.~M.~Szelc} \affiliation{\Edinburgh}
\author{H. A.~Tanaka} \affiliation{\SLAC}
\author{B.~Tapia Oregui} \affiliation{\Texasaustin}
\author{A.~Tapper} \affiliation{\Imperial}
\author{S.~Tariq} \affiliation{\Fermi}
\author{E.~Tatar} \affiliation{\Idaho}
\author{R.~Tayloe} \affiliation{\Indiana}
\author{A.~M.~Teklu} \affiliation{\StonyBrook}
\author{M.~Tenti} \affiliation{\INFNBologna}
\author{K.~Terao} \affiliation{\SLAC}
\author{C.~A.~Ternes} \affiliation{\IFIC}
\author{F.~Terranova} \affiliation{\INFNMilanBicocca}\affiliation{\MilanoBicocca}
\author{G.~Testera} \affiliation{\INFNGenova}
\author{T.~Thakore} \affiliation{\Cincinnati}
\author{A.~Thea} \affiliation{\Rutherford}
\author{J.~L.~Thompson} \affiliation{\Sheffield}
\author{C.~Thorn} \affiliation{\Brookhaven}
\author{S.~C.~Timm} \affiliation{\Fermi}
\author{V.~Tishchenko} \affiliation{\Brookhaven}
\author{J.~Todd} \affiliation{\Cincinnati}
\author{L.~Tomassetti} \affiliation{\INFNFerrara}\affiliation{\Ferrarauniv}
\author{A.~Tonazzo} \affiliation{\Parisuniversite}
\author{D.~Torbunov} \affiliation{\Minntwin}
\author{M.~Torti} \affiliation{\INFNMilanBicocca}\affiliation{\MilanoBicocca}
\author{M.~Tortola} \affiliation{\IFIC}
\author{F.~Tortorici} \affiliation{\INFNCatania}\affiliation{\CataniaUniversitadi}
\author{N.~Tosi} \affiliation{\INFNBologna}
\author{D.~Totani} \affiliation{\CalSantabarbara}
\author{M.~Toups} \affiliation{\Fermi}
\author{C.~Touramanis} \affiliation{\Liverpool}
\author{R.~Travaglini} \affiliation{\INFNBologna}
\author{J.~Trevor} \affiliation{\Caltech}
\author{S.~Trilov} \affiliation{\Bristol}
\author{A.~Tripathi} \affiliation{\TexasArlington}
\author{W.~H.~Trzaska} \affiliation{\Jyvaskyla}
\author{Y.~Tsai} \affiliation{\Fermi}
\author{Y.-T.~Tsai} \affiliation{\SLAC}
\author{Z.~Tsamalaidze} \affiliation{\Georgian}
\author{K.~V.~Tsang} \affiliation{\SLAC}
\author{N.~Tsverava} \affiliation{\Georgian}
\author{S.~Tufanli} \affiliation{\CERN}
\author{C.~Tull} \affiliation{\LawrenceBerkeley}
\author{E.~Tyley} \affiliation{\Sheffield}
\author{M.~Tzanov} \affiliation{\Louisanastate}
\author{L.~Uboldi} \affiliation{\CERN}
\author{M.~A.~Uchida} \affiliation{\Cambridge}
\author{J.~Urheim} \affiliation{\Indiana}
\author{T.~Usher} \affiliation{\SLAC}
\author{S.~Uzunyan} \affiliation{\Northernillinois}
\author{M.~R.~Vagins} \affiliation{\Kavli}
\author{P.~Vahle} \affiliation{\WilliamMary}
\author{G.~A.~Valdiviesso} \affiliation{\FederaldeAlfenas}
\author{E.~Valencia} \affiliation{\WilliamMary}
\author{P.~Valerio} \affiliation{\INFNBologna}\affiliation{\BolognaUniversity}
\author{Z.~Vallari} \affiliation{\Caltech}
\author{E.~Vallazza} \affiliation{\INFNMilanBicocca}
\author{J.~W.~F.~Valle} \affiliation{\IFIC}
\author{S.~Vallecorsa} \affiliation{\CERN}
\author{R.~Van Berg} \affiliation{\Penn}
\author{R.~G.~Van de Water} \affiliation{\LosAlmos}
\author{F.~Varanini} \affiliation{\INFNPadova}
\author{D.~Vargas} \affiliation{\IFAE}
\author{G.~Varner} \affiliation{\Hawaii}
\author{J.~Vasel} \affiliation{\Indiana}
\author{S.~Vasina} \affiliation{\JINR}
\author{G.~Vasseur} \affiliation{\CEASaclay}
\author{N.~Vaughan} \affiliation{\OregonState}
\author{K.~Vaziri} \affiliation{\Fermi}
\author{S.~Ventura} \affiliation{\INFNPadova}
\author{A.~Verdugo} \affiliation{\CIEMAT}
\author{S.~Vergani} \affiliation{\Cambridge}
\author{M.~A.~Vermeulen} \affiliation{\Nikhef}
\author{M.~Verzocchi} \affiliation{\Fermi}
\author{M.~Vicenzi} \affiliation{\INFNGenova}\affiliation{\Genova}
\author{H.~Vieira de Souza} \affiliation{\Campinas}\affiliation{\INFNMilanBicocca}
\author{C.~Vignoli} \affiliation{\GranSassoLab}
\author{C.~Vilela} \affiliation{\CERN}
\author{B.~Viren} \affiliation{\Brookhaven}
\author{T.~Vrba} \affiliation{\CzechTechnical}
\author{T.~Wachala} \affiliation{\Niewodniczanski}
\author{A.~V.~Waldron} \affiliation{\Imperial}
\author{M.~Wallbank} \affiliation{\Cincinnati}
\author{C.~Wallis} \affiliation{\ColoradoState}
\author{H.~Wang} \affiliation{\CalLosangeles}
\author{J.~Wang} \affiliation{\SouthDakotaSchool}
\author{L.~Wang} \affiliation{\LawrenceBerkeley}
\author{M.H.L.S.~Wang} \affiliation{\Fermi}
\author{Y.~Wang} \affiliation{\CalLosangeles}
\author{Y.~Wang} \affiliation{\StonyBrook}
\author{K.~Warburton} \affiliation{\IowaState}
\author{D.~Warner} \affiliation{\ColoradoState}
\author{M.O.~Wascko} \affiliation{\Imperial}
\author{D.~Waters} \affiliation{\UniversityCollegeLondon}
\author{A.~Watson} \affiliation{\Birmingham}
\author{P.~Weatherly} \affiliation{\Drexel}
\author{A.~Weber} \affiliation{\Rutherford}\affiliation{\Oxford}
\author{M.~Weber} \affiliation{\Bern}
\author{H.~Wei} \affiliation{\Brookhaven}
\author{A.~Weinstein} \affiliation{\IowaState}
\author{D.~Wenman} \affiliation{\Wisconsin}
\author{M.~Wetstein} \affiliation{\IowaState}
\author{A.~White} \affiliation{\TexasArlington}
\author{L.~H.~Whitehead} \affiliation{\Cambridge}
\author{D.~Whittington} \affiliation{\Syracuse}
\author{M.~J.~Wilking} \affiliation{\StonyBrook}
\author{C.~Wilkinson} \email[Corresponding author: ]{cwilkinson@lbl.gov}\affiliation{\LawrenceBerkeley}
\author{Z.~Williams} \affiliation{\TexasArlington}
\author{F.~Wilson} \affiliation{\Rutherford}
\author{R.~J.~Wilson} \affiliation{\ColoradoState}
\author{W.~Wisniewski} \affiliation{\SLAC}
\author{J.~Wolcott} \affiliation{\Tufts}
\author{T.~Wongjirad} \affiliation{\Tufts}
\author{A.~Wood} \affiliation{\Houston}
\author{K.~Wood} \affiliation{\StonyBrook}
\author{E.~Worcester} \affiliation{\Brookhaven}
\author{M.~Worcester} \affiliation{\Brookhaven}
\author{C.~Wret} \affiliation{\Rochester}
\author{W.~Wu} \affiliation{\Fermi}
\author{W.~Wu} \affiliation{\CalIrvine}
\author{Y.~Xiao} \affiliation{\CalIrvine}
\author{F.~Xie} \affiliation{\Sussex}
\author{E.~Yandel} \affiliation{\CalSantabarbara}
\author{G.~Yang} \affiliation{\StonyBrook}
\author{K.~Yang} \affiliation{\Oxford}
\author{S.~Yang} \affiliation{\Cincinnati}
\author{T.~Yang} \affiliation{\Fermi}
\author{A.~Yankelevich} \affiliation{\CalIrvine}
\author{N.~Yershov} \affiliation{\INR}
\author{K.~Yonehara} \affiliation{\Fermi}
\author{T.~Young} \affiliation{\Northdakota}
\author{B.~Yu} \affiliation{\Brookhaven}
\author{H.~Yu} \affiliation{\Brookhaven}
\author{H.~Yu} \affiliation{\Sunyatsen}
\author{J.~Yu} \affiliation{\TexasArlington}
\author{W.~Yuan} \affiliation{\Edinburgh}
\author{R.~Zaki} \affiliation{\York}
\author{J.~Zalesak} \affiliation{\CzechAcademyofSciences}
\author{L.~Zambelli} \affiliation{\DannecyleVieux}
\author{B.~Zamorano} \affiliation{\Granada}
\author{A.~Zani} \affiliation{\INFNMilano}
\author{L.~Zazueta} \affiliation{\WilliamMary}
\author{G.~P.~Zeller} \affiliation{\Fermi}
\author{J.~Zennamo} \affiliation{\Fermi}
\author{K.~Zeug} \affiliation{\Wisconsin}
\author{C.~Zhang} \affiliation{\Brookhaven}
\author{M.~Zhao} \affiliation{\Brookhaven}
\author{E.~Zhivun} \affiliation{\Brookhaven}
\author{G.~Zhu} \affiliation{\Ohiostate}
\author{P.~Zilberman} \affiliation{\StonyBrook}
\author{E.~D.~Zimmerman} \affiliation{\ColoradoBoulder}
\author{M.~Zito} \affiliation{\CEASaclay}
\author{S.~Zucchelli} \affiliation{\INFNBologna}\affiliation{\BolognaUniversity}
\author{J.~Zuklin} \affiliation{\CzechAcademyofSciences}
\author{V.~Zutshi} \affiliation{\Northernillinois}
\author{R.~Zwaska} \affiliation{\Fermi}
%1179 authors
%----------------------------------------------------
\collaboration{The DUNE Collaboration}
\noaffiliation



\begin{abstract}

It has been shown elsewhere that the Deep Underground Neutrino Experiment (DUNE) will produce world-leading neutrino oscillation measurements over the lifetime of the experiment. In this work, we explore DUNE's sensitivity to observe charge-parity violation (CPV) in the neutrino sector, and to resolve the mass hierarchy for short exposures. The analysis carried out includes detailed uncertainties on the flux prediction, neutrino interaction model, and detector effects. We demonstrate that DUNE will we able to unambigously resolve the neutrino mass hierarchy at a 3$\sigma$ (5$\sigma$) level, with only a 66 (100) kiloton-megawatt-year (ktMWyr) far detector exposure, and has the ability to make strong statements at significantly shorter exposures depending on the true value of other oscillation parameters. Additionally, we show that DUNE has the potential to make a robust measurement of CPV at a 3$\sigma$ level with a 100 ktMWyr exposure for the maximally CP-violating values $\deltacp = \pm\pi/2$. Additionally, we discuss ways to enhance the sensitivity with specific run plans if there is a compelling reason to search a specific area of parameter space provided by other experiments.
\end{abstract}

\maketitle

%% Rationalise the structure a bit
\section{Introduction}
\label{sec:intro}

The Deep Underground Neutrino Experiment (DUNE) is a next-generation, long-baseline neutrino oscillation experiment which will utilize high-intensity \numu and \anumu beams with peak neutrino energies of $\sim$2.5 GeV over a 1285 km baseline to carry out a detailed study of neutrino mixing. DUNE's key scientific goals are the definitive determination of the neutrino mass ordering, the definitive observation of charge-parity symmetry violation (CPV) for more than 50\% of possible true values of the charge-parity violating phase, \deltacp, and the precise measurement of other three-neutrino oscillation parameters.
These measurements will help guide theory in understanding if there are new symmetries in the neutrino sector and whether there is a relationship between the generational structure of quarks and leptons~\cite{Qian:2015waa}. Observation of CPV in neutrinos would be an important step in understanding the origin of the baryon asymmetry of the universe~\cite{Fukugita:1986hr, Davidson:2008bu}. We note that DUNE has a rich physics program beyond the three-neutrino oscillation accelerator neutrino program described here, which are discussed in other works, including beyond standard model searches~\cite{Abi:2020kei}, supernova neutrino detection~\cite{Abi:2020lpk}, and solar neutrino detection~\cite{Capozzi:2018dat}. Additional physics possibilities with DUNE are discussed in Refs.~\cite{Abi:2020evt} and~\cite{AbedAbud:2021hpb}.

\todo{Update refs}
Neutrino oscillation experiments have so far measured five of the neutrino mixing parameters~\cite{Esteban:2018azc,deSalas:2017kay,Capozzi:2017yic}: the three mixing angles $\theta_{12}$, $\theta_{23}$, and $\theta_{13}$; and the two squared-mass differences $\Delta m^{2}_{21}$ and $|\Delta m^{2}_{31}|$, where $\Delta m^2_{ij} = m^2_{i} - m^{2}_{j}$ is the difference between the squares of the neutrino mass states in eV$^{2}$.
\todo{Still true?}
The neutrino mass ordering (the sign of $\Delta m^{2}_{31}$) is not currently known, though recent results show a weak preference for the normal ordering (NO) over the inverted ordering (IO)~\cite{Abe:2021gky,PhysRevD.97.072001,PhysRevLett.123.151803}.
The value of \deltacp is not well known, though neutrino oscillation data are beginning to provide some information on its value~\cite{Abe:2019vii,Abe:2021gky}.

The oscillation probability of $\;\nu^{\bracketbar}_\mu \rightarrow \nu^{\bracketbar}_e$ through matter in the standard three-flavor model and a constant density approximation is, to first order~\cite{Nunokawa:2007qh}:
\begin{equation}
  \begin{aligned}
    P(\;\nu^{\bracketbar}_\mu \rightarrow \nu^{\bracketbar}_e) & \simeq \sin^2 \theta_{23} \sin^2 2 \theta_{13} 
    \frac{ \sin^2(\Delta_{31} - aL)}{(\Delta_{31}-aL)^2} \Delta_{31}^2 \\
    & + \sin 2 \theta_{23} \sin 2 \theta_{13} \sin 2 \theta_{12}\frac{ \sin(\Delta_{31} - aL)}{(\Delta_{31}-aL)} \Delta_{31} \\
    &\times \frac{\sin(aL)}{(aL)} \Delta_{21} \cos (\Delta_{31} \pm \deltacp) & \\
    & + \cos^2 \theta_{23} \sin^2 2 \theta_{12} \frac {\sin^2(aL)}{(aL)^2} \Delta_{21}^2,
  \end{aligned}
  \label{eqn:appprob}
\end{equation}
where
\begin{equation*}
  a = \pm \frac{G_{\mathrm{F}}N_e}{\sqrt{2}} \approx \pm\frac{1}{3500~\mathrm{km}}\left(\frac{\rho}{3.0~\mathrm{g/cm}^{3}}\right),
\end{equation*}
$G_{\mathrm{F}}$ is the Fermi constant, $N_e$ is the number density of electrons in the Earth's crust, $\Delta_{ij} = 1.267 \Delta m^2_{ij} L/E_\nu$, $L$ is the baseline in km, and $E_\nu$ is the neutrino energy in GeV. 
Both \deltacp and $a$ terms are positive (negative) for $\numu \to \nue$ ($\anumu \to \anue$) oscillations. The matter effect asymmetry arises from the presence of electrons and absence of positrons in the Earth~\cite{Wolfenstein:1977ue,Mikheev:1986gs}.

In Ref.~\cite{Abi:2020qib}, DUNE's sensitivity to CPV and the neutrino mass ordering, as well as other oscillation parameters was explored for large exposures, showing the ultimate sensitivity of the experiment. Sophisticated studies with a detailed systematics treatment were carried out only at large exposures, along with simple studies that showed DUNE's expected sensitivity for lower exposures. In this work, we explore DUNE's sensitivity at low exposures further, with a detailed systematics treatment, and an investigation into how the run plan may be optimized to enhance sensitivity to CPV and/or mass ordering. We show that DUNE will produce world-leading results at relatively short exposures, which highlights the need for a highly performant near detector complex from the beginning of the experiment. As the DUNE far detector deployment schedule and beam power scenarios are both subject to change, the results shown in this work are consistently given in terms of exposure in units of kiloton-megawatt-years (ktMWyrs), which is agnostic to the exact staging scenario, but can easily be expressed in terms of experiment years for any desired scenario.

The analysis framwork used in this work is described in Section~\ref{sec:analysis_framework}, including a description of the flux, neutrino interaction and detector models and associated uncertainties. A study on the dependence of the sensitivity to CPV and mass ordering to the fraction of data collected in neutrino-enhanced or antineutrino-enhanced running is given in Section~\ref{sec:run_plan_opt}. A detailed study on the CPV and mass ordering sensitivities at low exposures are described in Sections~\ref{sec:cp_sens} and~\ref{sec:mh_sens}, respectively. Finally, we present our conclusions in Section~\ref{sec:conclude}.


\section{Analyis framework}\label{sec:analysis_framework}

This work uses the flux, neutrino interaction and detector model described in Ref.~\cite{Abi:2020qib}, implemented in the CAFAna framework. In this section, an overview of the analysis will be given. Additional details can be found in Ref.~\cite{Abi:2020qib}.

%% Flux in brief
The neutrino flux is generated with G4LBNF~\cite{Aliaga:2016oaz,Abi:2020evt}, using the \dword{lbnf} optimized beam design~\cite{Abi:2020evt}. Flux uncertainties are due to uncertainties in hadrons produced off the target and uncertainties in the design parameters of the beamline, such as horn currents and horn and target positioning (commonly called ``focusing uncertainties'')~\cite{Abi:2020evt}. They are evaluated using current measurements of hadron production and \dword{lbnf} estimates of alignment tolerances, giving flux uncertainties of approximately 8\% at the first oscillation maximum which are highly correlated across energy bins and neutrino flavors. A many universe approach is taken to evaluate the flux prediction for variations of the systematics propagated through the full beamline simulation, to build a flux covariance matrix as a function of neutrino energy, beam mode, detector, and neutrino species. To reduce the number of parameters used in the fit, the covariance matrix is then diagonalized, and each principal component is treated as an uncorrelated nuisance parameter. It was found that only the first $\sim$30 principal components have a significant effect in the analysis and need to be included. Note that the unoscillated fluxes at the ND and FD are similar, and differences between them are well understood.

%% Neutrino interactions in brief
The interaction simulation used is based on GENIE v2.12.10~\cite{Andreopoulos:2009rq,Andreopoulos:2015wxa}. The nuclear model used to describe the initial state of nucleons in the nucleus is the Bodek-Ritchie global Fermi gas model~\cite{BodekRitchie} which includes empirical modifications to the nucleon momentum distribution to account for short-range correlation effects. The quasi-elastic model uses the LLewelyn-Smith formalism~\cite{llewelyn-smith} with a simple dipole axial form factor, and BBBA05 vector form factors~\cite{bbba05}. Nuclear screening effects and uncertainties are included based on the T2K 2017/8 parameterization~\cite{Abe:2018wpn} of the Valencia group's~\cite{nieves1,nieves2} Random Phase Approximation model. The Valencia model of the multi-nucleon, $2p2h$, contribution to the cross section~\cite{nieves1,nieves2} is used, with the implementation in \dword{genie} as described in Ref.~\cite{Schwehr:2016pvn}. Both MINERvA~\cite{Rodrigues:2015hik} and NOvA~\cite{NOvA:2018gge} have shown that this model underpredicts observed event rates on carbon. Modifications to the model are constructed to produce agreement with MINERvA CC-inclusive data~\cite{Rodrigues:2015hik}, and are used to construct additional uncertainties on the $2p2h$ contribution, including energy dependent uncertainties, and extra freedom between neutrinos and antineutrinos. \dword{genie} uses a modified version of the Rein-Sehgal (R-S) model for pion production~\cite{Rein:1980wg}. We include a data-driven modification to the \dword{genie} model based on reanalyzed neutrino--deuterium bubble chamber data~\cite{Wilkinson:2014yfa,Rodrigues:2016xjj}. The \dword{dis} model implemented in \dword{genie} uses the Bodek-Yang parametrization~\cite{Bodek:2002ps}, using GRV98 parton distribution functions~\cite{Gluck:1998xa}. Hadronization is described by the AKGY model~\cite{Yang:2009zx}, which uses the KNO scaling model~\cite{Koba:1972ng} for invariant masses $W \leq 2.3$ GeV and PYTHIA6~\cite{Sjostrand:2006za} for invariant masses $W \geq 3$ GeV, with a smooth transition between the two for intermediate invariant masses. We include additional uncertainties developed by the NOvA experiment~\cite{nova_2018} to describe their resonance to \dword{dis} transition region data. We additionally include the large number of final state uncertainties on the final state cascade model provided by GENIE~\cite{Dytman:2011zz,Dytman:2015taa,intranuke_2009}.

The cross sections include terms proportional to the lepton mass, which are significant at low energies where quasielastic processes dominate. Some of the form factors in these terms have significant uncertainties in the nuclear environment. We adopt separate (and anticorrelated) uncertainties on the cross section ratio $\sigma_\mu/\sigma_e$ for neutrinos and antineutrinos from Ref.~\cite{Day:2012gb}. Additionally, some electron-neutrino interactions occur at four-momentum transfers where muon-neutrino interactions are kinematically forbidden, and so cannot be constrained by muon-neutrino cross-section measurements. A 100\% uncertainty is applied in the phase space present for $\nu_e$ but absent for $\nu_\mu$.

%% Need to dump these all out into this single file when suitably cut down
%\input{sections/nd}
%\input{sections/fd}
%\input{sections/syst}
%\input{sections/methods}

\section{Run plan optimization}
\label{sec:run_plan_opt}
In previous DUNE sensitivity studies~\cite{Abi:2020qib}, equal running times in FHC and RHC were assumed, based on early sensitivity estimates for different scenarios. In this section, the dependence on the median CPV and mass ordering sensitivities are studied, for different fractions of time spent in each beam mode, using the full analysis framework described in Section~\ref{sec:analysis_framework}.
\begin{figure*}[htbp]
  \centering
  \subfloat[NO, with $\theta_{13}$-penalty] {\includegraphics[width=0.4\linewidth]{{cpv_sens_ndfd100kTMWyr_th13_asimov0_nh}.png}}
  \subfloat[IO, with $\theta_{13}$-penalty] {\includegraphics[width=0.4\linewidth]{{cpv_sens_ndfd100kTMWyr_th13_asimov0_ih}.png}}\\
  \subfloat[NO, no $\theta_{13}$-penalty]   {\includegraphics[width=0.4\linewidth]{{cpv_sens_ndfd100kTMWyr_nopen_asimov0_nh}.png}}
  \subfloat[IO, no $\theta_{13}$-penalty]   {\includegraphics[width=0.4\linewidth]{{cpv_sens_ndfd100kTMWyr_nopen_asimov0_ih}.png}}
  \caption{The Asimov CPV sensitivity as a function of the true value of \deltacp, for a total exposure of 100 ktMWyr with different fractions of FHC and RHC running, with and without a $\theta_{13}$ penalty applied in the fit. Results are shown for both true normal and inverted ordering, with the true oscillation parameter values set to the NuFit 4.0 best fit point in each ordering (see Table~\ref{tab:oscpar_nufit}).}
  \label{fig:run_opt_cpv}
\end{figure*}
Figure~\ref{fig:run_opt_cpv} shows DUNE's Asimov sensitivity to CPV for a total 100 ktMWyr far detector exposure, with different fractions of FHC and RHC running, at the NuFIT 4.0 best fit value in both NO and IO (see Table~\ref{tab:oscpar_nufit}), shown with and without a penalty on $\theta_{13}$ applied. For each point tested, all oscillation and nuisance parameters are allowed to vary, and two fits are carried out, one where the \deltacp is fixed at CP-conserving values, and another where it is allowed to vary. The difference in the best-fit $\chi^{2}$ values is calculated:
\begin{equation}
  \Delta\chi^{2} = \chi^{2}_{0,\pm\pi} - \chi^{2}_{\mathrm{CPV}},
  \label{eq:cpv_chi2}
\end{equation}
\noindent and the square root of the difference is  used as the figure of merit on the y-axis in Figure~\ref{fig:run_opt_mh}. We note that there are some caveats associated with this figure of merit, which are discussed in Section~\ref{sec:cp_sens}. A 100 ktMWyr exposure is shown as it was identified in Ref~\cite{Abi:2020qib} as the exposure at which DUNE's median CPV sensitivity exceeds 3$\sigma$ at $\deltacp = \pm\pi/2$, an important milestone in DUNE's physics program (with equal beam mode running). 

Figure~\ref{fig:run_opt_cpv} shows that with a $\theta_{13}$ penalty applied, the sensitivity to CPV can be increased in some regions of \deltacp parameter space, with more FHC than RHC running. However this degrades the sensitivity in other regions, most notably for $\deltacp > 0$ in both orderings, where the octant degeneracy starts to strongly impact the results. For regions of phase space where the octant degeneracy does not affect the result (e.g., $\sin^{2}\theta_{23} \approx 0.5$), there is no degredation in the sensitivity, and enhanced FHC running increases the sensitivity for all values of \deltacp. Increasing the fraction of RHC decreases the sensitivity for the entire \deltacp range when the $\theta_{13}$ penalty is applied, relative to equal beam mode running, which can be understood as being due to the lower statistics of the \anue sample (see Figure~\ref{fig:appspectra}). Although for low exposures, DUNE will not have a strong constraint on $\theta_{13}$, so the main analysis will include a $\theta_{13}$ penalty, it is instructive to look at the results without the penalty applied. In the no penalty case, the sensitivity is severely degraded (as expected) for 100\% running in either beam mode.

\begin{figure*}[htbp]
  \centering
  \subfloat[NO, with $\theta_{13}$-penalty]  {\includegraphics[width=0.4\linewidth]{{mh_sens_ndfd24kTMWyr_th13_asimov0_nh}.png}}
  \subfloat[IO, with $\theta_{13}$-penalty]  {\includegraphics[width=0.4\linewidth]{{mh_sens_ndfd24kTMWyr_th13_asimov0_ih}.png}}\\
  \subfloat[NO, no $\theta_{13}$-penalty]    {\includegraphics[width=0.4\linewidth]{{mh_sens_ndfd24kTMWyr_nopen_asimov0_nh}.png}}
  \subfloat[IO, no $\theta_{13}$-penalty]    {\includegraphics[width=0.4\linewidth]{{mh_sens_ndfd24kTMWyr_nopen_asimov0_ih}.png}}
  \caption{The Asimov mass ordering sensitivity as a function of the true value of \deltacp, for a total exposure of 24 ktMWyr with different fractions of FHC and RHC running, with and without a $\theta_{13}$ penalty applied in the fit. Results are shown for both true normal and inverted ordering, with the true oscillation parameter values set to the NuFIT 4.0 best fit point in each ordering (see Table~\ref{tab:oscpar_nufit}).}
  \label{fig:run_opt_mh}
\end{figure*}
Figure~\ref{fig:run_opt_mh} shows DUNE's Asimov sensitivity to the mass ordering for a total 24 ktMWyr far detector exposure, with different fractions of FHC and RHC running, and the same four true oscillation parameter sets. For each point tested, all oscillation and nuisance parameters are allowed to vary, and two fits are carried out, one using each ordering. The difference in the best-fit $\chi^{2}$ values is calculated:
\begin{equation}
  \Delta\chi^{2} = \chi^{2}_{\mathrm{IO}} - \chi^{2}_{\mathrm{NO}},
  \label{eq:mh_chi2}
\end{equation}
\noindent and the square root of the difference is used as the figure of merit on the y-axis in Figure~\ref{fig:run_opt_mh}. We note that there are some caveats associated with this figure of merit, which are discussed in Section~\ref{sec:mh_sens}. A 24 ktMWyr exposure is used in Figure~\ref{fig:run_opt_mh} as it is around the exposure at which DUNE's median mass ordering sensitivity exceeds 5$\sigma$ for some vales of \deltacp~\cite{Abi:2020qib}.

 It is clear from Figure~\ref{fig:run_opt_mh} that the mass ordering sensitivity has a strong dependence on the fraction of running in each beam mode, and as in the CPV case, the effect is very different with and without a $\theta_{13}$ penalty applied. In normal ordering with the $\theta_{13}$ penalty applied, the sensitivity increases significantly with increasing FHC running, with a full 1$\sigma$ increase in the sensitivity between equal beam running and 100\% FHC for most values of \deltacp values. Conversely, in inverted ordering with the $\theta_{13}$ penalty applied, 100\% FHC running would degrade the sensitivity by $\geq$1$\sigma$ for all values of \deltacp at the NuFIT 4.0 best fit point. Overall the sensitivity in inverted ordering prefers a more equal split between the beam modes. It is clear that 100\% RHC running gives poor sensitivity for all values tested. It is also clear by comparison with the no penalty case, that the increased sensitivity with enhanced FHC running is entirely due to the $\theta_{13}$ penalty, and without it, a more equal split in beam running would be favored.

\begin{figure*}[htbp]
  \centering
  \subfloat[CPV, with $\theta_{13}$-penalty] {\includegraphics[width=0.4\linewidth]{{cpv_sens_ndfd334kTMWyr_th13_asimov0_nh}.png}}
  \subfloat[CPV, no $\theta_{13}$-penalty]   {\includegraphics[width=0.4\linewidth]{{cpv_sens_ndfd334kTMWyr_nopen_asimov0_nh}.png}}\\
  \subfloat[MO, with $\theta_{13}$-penalty]  {\includegraphics[width=0.4\linewidth]{{mh_sens_ndfd334kTMWyr_th13_asimov0_nh}.png}}
  \subfloat[MO, no $\theta_{13}$-penalty]    {\includegraphics[width=0.4\linewidth]{{mh_sens_ndfd334kTMWyr_nopen_asimov0_nh}.png}}
  \caption{The Asimov CPV and mass ordering sensitivities as a function of the true value of \deltacp, for a total exposure of 334 ktMWyr with different fractions of FHC and RHC running, with and without a $\theta_{13}$ penalty applied in the fit. Results are shown for both true normal ordering only, with the true oscillation parameter values set to the NuFIT 4.0 NO best fit point (see Table~\ref{tab:oscpar_nufit}).}
  \label{fig:run_opt_334ktmwyr}
\end{figure*}

For comparison, Figure~\ref{fig:run_opt_334ktmwyr} shows the Asimov CPV and mass ordering sensitivities, with and without a $\theta_{13}$ penalty applied, for true normal ordering only, for a large exposure of 334 ktMWyrs, with different fractions of FHC and RHC running. At large exposures, running with strongly enhanced FHC running no longer improves the sensitivity over equal beam mode running, with or without the $\theta_{!3}$ penalty applied, for either CPV or mass ordering determination. 

Overall, the sensitivity to CPV and the mass ordering is dependent on the division of running time between FHC and RHC, but a choice that increases the sensitivity in some region of parameter space can severely decrease the sensitivity in other regions. If there is strong reason to favor, for example, normal over inverted hierarchy when DUNE starts to take data, Figure~\ref{fig:run_opt_mh} shows that this could be more rapidly verified by running with more FHC data than RHC data, as a $\theta_{13}$ penalty will be used in the main low exposure analysis. However, if this choice is wrong, this might delay the results. Clearly this is an important consideration which should be revisited shortly before DUNE begins to collect data. Similarly, the CPV sensitivity shown in Figure~\ref{fig:run_opt_cpv} might be optimized if there is a strong reason to favor gaining sensitivity in a region greater than or less than \deltacp, at a cost to the other peak. But, it is clear from Figures~\ref{fig:run_opt_cpv} and~\ref{fig:run_opt_mh} that equal running in FHC and RHC gives a close to optimal sensitivity across all of the parameter space, and as such is a reasonable {\it a priori} choice of run plan for studies of the DUNE sensitivity. Additionally, it is clear from Figure~\ref{fig:run_opt_334ktmwyr} that the improvement in the sensitivity with unequal beam running is a feature at low exposures, but not at high exposures, particularly because at high exposures when DUNE is able to constrain all the oscillation parameters with precision~\cite{Abi:2020qib}, there is a stronger motivation to run a DUNE-only analysis, without a $\theta_{13}$ penalty.



\FloatBarrier
\section{CP violation sensitivity}
\label{sec:cp_sens}

In this section, CPV sensitivity results are presented. For simplicity, only true NO will be shown unless explicitly stated. In all cases, a joint ND+FD fit is performed, and a $\theta_{13}$ penalty is always applied to incorporate the reactor measurement, as described in Section~\ref{sec:analysis_framework}. An equal split between FHC and RHC running is assumed based on the results obtained in Section~\ref{sec:run_plan_opt}. Asimov sensitivities, as shown in Section~\ref{sec:run_plan_opt}, are informative but do not give information on how the expected sensitivity may vary with statistical or systematic uncertainties, or for variations in the other oscillation parameters of interest.

\begin{figure}[htbp]
  \centering
  \includegraphics[width=0.8\linewidth, trim={0cm 0cm 0cm 2.3cm}, clip]{cpv_two_exps_throws_nh_2019_v4_lowexp.png}\\
  \includegraphics[width=0.8\linewidth, trim={0cm 0cm 0cm 2.3cm}, clip]{cpv_two_exps_throws_ih_2019_v4_lowexp.png}
  \caption{Significance of the DUNE determination of CP-violation ($\deltacp \neq \{0,\pm\pi\}$) as a function of the true value of \deltacp, for 66 kt-MW-yr (blue) and 100 kt-MW-yr (orange) exposures, for normal (top) and inverted (bottom) orderings. The width of the transparent bands cover 68\% of fits in which random throws are used to simulate systematic, oscillation parameter and statistical variations, with independent fits performed for each throw constrained by prior uncertainties. The solid lines show the median significance.}
  \label{fig:cpv_bands}
\end{figure}
Figure~\ref{fig:cpv_bands} shows the significance with which CPV ($\deltacp \neq \{0,\pm\pi\}$) can be observed for both NO and IO, for exposures of 66 and 100 kt-MW-yr.  The sensitivity metric used is the square root of the difference between the best fit $\chi^{2}$ values obtained for a CP-conserving fit and one where \deltacp is allowed to vary, as shown in Equation~\ref{eq:mh_chi2}, which is calculated for each throw of the systematic, other oscillation parameters and statistics. This constant-\dchisq method is valid as long as Wilks' theorem can be applied~\cite{wilks}.

The sensitivity shown in Figure~\ref{fig:cpv_bands} has a characteristic double peak structure because the significance of a CPV measurement decreases around CP-conserving values. The systematic and statistical variations which mean that all throws have $\dchisqCPV \geq 0$, and therefore neither the median significance nor the band showing the central 68\% of throws reach exactly 0 at CP-conserving values. This is entirely expected, it simply means that random variations in the data will cause us to obtain a 1$\sigma$ measurement of CPV $\sim$32\% of the time for CP-conserving values. Median significances are slightly higher for IO than for NO, and by exposures of 100 kt-MW-yr, the median significance exceeds 3$\sigma$ for the maximal CP-violating values of $\pm\pi/2$. This presentation of the CPV sensitivity was followed in Ref.~\cite{Abi:2020qib}, and is very informative at high exposures. Around CP-conserving values ($\deltacp = \{0,\pm\pi\}$), the distribution of the sensitivity metric $\sqrt{\dchisqCPV}$ is highly non-Gaussian for all exposures. Additionally, at lower exposures, as shown in Figure~\ref{fig:cpv_bands}, the distribution of $\sqrt{\dchisqCPV}$ around maximally CP-violating values of $\deltacp = \pm\pi/2$ is increasingly non-Gaussian, making the spread in sensitivity harder to interpret with this presentation.

\begin{figure*}[htbp]
  \centering
  \subfloat[24 kt-MW-yr] {\includegraphics[width=0.33\linewidth]{cpv_throws_24ktMWyr_NH_th13.png}}
  \subfloat[66 kt-MW-yr] {\includegraphics[width=0.33\linewidth]{cpv_throws_66ktMWyr_NH_th13.png}}
  \subfloat[100 kt-MW-yr]{\includegraphics[width=0.33\linewidth]{cpv_throws_100ktMWyr_NH_th13.png}}\\
  \subfloat[150 kt-MW-yr]{\includegraphics[width=0.33\linewidth]{cpv_throws_150ktMWyr_NH_th13.png}}
  \subfloat[197 kt-MW-yr]{\includegraphics[width=0.33\linewidth]{cpv_throws_197ktMWyr_NH_th13.png}}
  \subfloat[334 kt-MW-yr]{\includegraphics[width=0.33\linewidth]{cpv_throws_334ktMWyr_NH_th13.png}}
  \caption{Fraction of throws for which the significance of DUNE's CP-violation test ($\deltacp \neq \{0,\pm\pi\}$) exceeds 1--5$\sigma$, as a function of the true value of \deltacp. Shown for NO, for a number of different exposures. The number of throws used to make each figure is also shown.}
  \label{fig:cpv_over_time}
\end{figure*}
\begin{figure*}[htbp]
  \centering
  \subfloat[$\deltacp = -\pi/2$]     {\includegraphics[width=0.8\columnwidth]{{fraction_throws_vs_exp_dcp-0.5}.pdf}}
  \subfloat[50\% of \deltacp values] {\includegraphics[width=0.8\columnwidth]{{fraction_throws_vs_exp_dcprange_0.5}.pdf}}
  \caption{Fraction of throws for which the significance of DUNE's CP-violation test ($\deltacp \neq \{0,\pm\pi\}$) exceeds 1--5$\sigma$, both assuming $\deltacp = -\pi/2$, and for 50\% of \deltacp values, shown as a function of exposure, for NO.}
  \label{fig:cpv_vs_exp}
\end{figure*}

Figure~\ref{fig:cpv_over_time} provides an alternative way to present the results of the throws. The fraction of throws for which the test statistic shown in Equation~\ref{eq:cpv_chi2} exceeds different confidence levels is shown, for 1--5$\sigma$ significances, for a variety of exposures. It shows the fraction of throws for which DUNE would observe a CPV significance above a discrete threshold, as a function of the true value of \deltacp. The point at which the median significance (50\% of throws) passes different significance thresholds can be easily read from the figures, and can be compared with those shown in Figure~\ref{fig:cpv_bands}. Note that the highest exposure shown, 334 kt-MW-yr, is equivalent to the seven year exposure using the staging scenario from Ref.~\cite{Abi:2020qib}. The same double peak structure seen in Figure~\ref{fig:cpv_bands} can be observed. The median significance for measuring CPV exceeds 3$\sigma$ after $\sim$100 kt-MW-yr, but a significant fraction of throws exceed 3$\sigma$ at shorter exposures. Likewise, although the median significance to CPV does not exceed 5$\sigma$ until $\sim$334 kt-MW-yr, there are significant fractions of throws at lower exposures which reach $5\sigma$ significance. This presentation also shows that by 334 kt-MW-yr exposures, the fraction of throws for which the significance is less than 3$\sigma$ at maximal values of \deltacp is very small. The number of throws carried out at each exposure is indicated on each plot. The number of throws decreases as a function of exposure because fixed computing resources were used for each configuration, and the time for the ensemble of fits carried out for each throw to complete increases slightly with exposure. The final 334 kt-MW-yr exposure has more throws because it was generated for the analysis presented in Ref.~\cite{Abi:2020qib}, where more than one projection was considered --- requiring more throws to sample the space.

Figure~\ref{fig:cpv_vs_exp} shows the fraction of throws which exceed different significance thresholds at the maximal CP-violating value of $\deltacp = -\pi/2$, and for 50\% of all \deltacp values, as a function of exposure. Figure~\ref{fig:cpv_vs_exp} was produced using the same throws used for Figure~\ref{fig:cpv_over_time}, with additional points from higher exposures used in Ref.~\cite{Abi:2020qib}, but not shown in Figure~\ref{fig:cpv_over_time} (646 kt-MW-yr and 1104 kt-MW-yr). After $\sim$200 kt-MW-yr, the median significance for 50\% of the \deltacp range is greater than 3$\sigma$ at which point the sensitivity at $\deltacp = -\pi/2$ exceeds 4$\sigma$. For NO, the significance at $\deltacp = -\pi/2$ is only slightly weaker than at $\deltacp = \pi/2$, as can be seen from Figure~\ref{fig:cpv_over_time}. For IO, the significance is more similar between the two points, which can be inferred from Figure~\ref{fig:cpv_bands}.

% \FloatBarrier
% \subsection{Feldman-Cousins studies}
Previous results have calculated confidence intervals and significances using constant \dchisq critical values. For example, in the CPV results shown in Figure~\ref{fig:cpv_over_time}, the confidence intervals are calculated assuming one degree of freedom, so $\dchisqCPV \leq 1, 4, 9$ corresponds to a significance of 1, 2 and 3$\sigma$, respectively. This assumption holds when Wilks' theorem can be applied~\cite{wilks}, but can lead to incorrect coverage where it cannot. It is known to break down for low-statistics samples, around physical boundaries, in the case of cyclic parameters, and where there are significant degeneracies. It is likely that a constant \dchisq treatment will break down for \deltacp, where all of these issues apply, as has indeed been shown by the T2K Collaboration~\cite{Abe:2021gky}.

The Feldman-Cousins method~\cite{Feldman:1997qc} is a brute force numerical method to calculate confidence intervals with correct coverage. A large number of toy experiments are produced, where the parameter(s) of interest (here \deltacp) is set to a desired true value, all other systematic and oscillation parameters are thrown, as described in Section~\ref{sec:analysis_framework}, and a statistical throw is made, for the two ND samples and four FD sample used in the analysis. Then two fits are performed, one where the parameter(s) of interest are fixed to the true value, and another where the test statistic is minimized with respect to the parameter(s) of interest. In both fits, all other parameters are allowed to vary. For each throw, the profile likelihood ratio \dchisqFC is calculated using the minimum $\chi^{2}$ values for those two fits, as in Equation~\ref{eq:dchisq_fc}.
\begin{equation}
  \dchisqFC = \chi^{2}(\theta_{\mathrm{true}}) - \min_{\theta}\chi^{2}(\theta)
  \label{eq:dchisq_fc}
\end{equation}
\begin{figure}[htbp]
  \centering
  \includegraphics[width=0.8\columnwidth]{nh_FC_ndfd_100ktMWyr_dcp0.png}
  \caption{Distribution of \dchisqFC values, calculated using Equation~\ref{eq:dchisq_fc}, for a large number of throws with true $\deltacp = 0$, and a 100 kt-MW-yr exposure. The \dchisqcrit values (vertical lines) obtained using the Feldman-Cousins method show the \dchisqFC value below which 68.27\% (1$\sigma$), 90\%, 95.45\% (2$\sigma$) and 99.73\% (3$\sigma$) of throws reside, with the calculated values given in the legend. The number of throws used is also given.}
  \label{fig:fc_throws}
\end{figure}
The distribution of these throws is used to calculate the \dchisqcrit that gives the desired coverage, with the appropriate fraction of toys above/below the calculated value. A distribution of \dchisqFC values is shown in Figure~\ref{fig:fc_throws} for an example ND+FD analysis with a 100kt-MW-yr exposure at the far detector, equal FHC and RHC run fractions, and the reactor $\theta_{13}$ constraint applied. In Figure~\ref{fig:fc_throws}, the \dchisqcrit values corresponding to for 68.27\% (1$\sigma$), 90\%, 95.45\% (2$\sigma$) and 99.73\% (3$\sigma$) of the throws are indicated. The \dchisqcrit values were only calculated up to the 3$\sigma$ level due to the very large number of throws required for higher confidence levels.

An uncertainty on the value of \dchisqcrit obtained from the toy throw distribution (e.g., Figure~\ref{fig:fc_throws}), is obtained using a bootstrap rethrowing method~\cite{rice2006mathematical}. The empirical PDF obtained from the throws is treated as the true PDF, and $B$ independent samples of size $n$ are drawn from it, where $n$ is the total number of throws used to build the empirical PDF. Note that each throw can be drawn multiple times in this method, so the ensemble of throws is different in each sample. Then, the standard deviation $s_{\hat{\vartheta}}$, on the \dchisqcrit values of interest, $\vartheta$, are calculated for each of the $B$ samples using:
\begin{equation}
  s_{\hat{\vartheta}} = \sqrt{\frac{1}{B-1} \sum^{B}_{i=0} (\vartheta_{i}^{*} - \bar{\vartheta}^{*})^{2}},
  \label{eq:fc_uncertainty}
\end{equation}
where $\vartheta_{i}^{*}$ denotes the calculated \dchisqcrit value of interest for each of the samples, and $\bar{\vartheta}^{*}$ is their average value.

\begin{figure}[htbp]
  \centering
  \includegraphics[width=0.8\columnwidth]{dchi2crit_vs_exp_dcp0.pdf}
  \caption{The \dchisqcrit values corresponding to 68.27\% (1$\sigma$), 90\%, 95.45\% (2$\sigma$) and 99.73\% (3$\sigma$) of throws, shown for true $\deltacp = 0$, as a function of exposure. A linear $x$-axis scale is used to highlight the stability of \dchisqcrit values for large exposures. The uncertainty on the \dchisqcrit values is obtained using Equation~\ref{eq:fc_uncertainty}, and is indicated as the shaded line. To guide the eye, horizontal dashed lines are included which indicate the 1$\sigma$, 90\%, 2$\sigma$ and 3$\sigma$ \dchisq values assumed using the constant-\dchisq method, with one degree of freedom. The distribution of throws used produced to calculate the \dchisqcrit values shown are given in Figure~\ref{fig:fc_throws_exp}.}
  \label{fig:fc_vs_exp}
\end{figure}
Figure~\ref{fig:fc_vs_exp} shows the evolution of the \dchisqcrit values as a function of exposure for $\deltacp = 0$, the relevant value for CPV sensitivity, for an ND+FD analysis with equal FHC and RHC running and the reactor $\theta_{13}$ constraint applied.
%Several values were checked for $\deltacp = \pm\pi$ and similar results were found, so Figure~\ref{fig:fc_vs_exp} can be thought of as the threshold for CPV significance.
For all significance levels tested, the \dchisqcrit rise quickly as a function of exposure, and stabilize at values slightly higher than those suggested by the constant \dchisq method by exposures of $\sim$100 kt-MW-yr. The initial rise in the \dchisqcrit values is due to the low statistics at those exposures. Overall, this implies that the CPV significance is slightly weaker than what would be inferred from $\sigma = \sqrt{\dchisqCPV}$, as used in earlier Figures shown in this work. Crucially, there is no constant increase in the \dchisqcrit values over time as has been reported by the T2K Collaboration~\cite{Abe:2021gky}. Details on the number of toy throws used at each point of Figure~\ref{fig:fc_vs_exp} are given in the Appendix, and the toy throw distributions from which the \dchisqcrit values and their uncertainties were calculated are shown in Figure~\ref{fig:fc_throws_exp}.

\begin{figure}[htbp]
  \centering
  \subfloat[100 kt-MW-yr]  {\label{fig:fc_vs_dcp_100}\includegraphics[width=0.8\columnwidth]{dchi2crit_vs_dcp_100ktMWyr.pdf}}\\
  \subfloat[334 kt-MW-yr]  {\label{fig:fc_vs_dcp_334}\includegraphics[width=0.8\columnwidth]{dchi2crit_vs_dcp_334ktMWyr.pdf}}
  \caption{The \dchisqcrit values corresponding to 68.27\% (1$\sigma$), 90\%, 95.45\% (2$\sigma$) and 99.73\% (3$\sigma$) of throws, shown as a function of true \deltacp, for exposures of 100 kt-MW-yr and 334 kt-MW-yr. The uncertainty on the \dchisqcrit values is obtained using Equation~\ref{eq:fc_uncertainty}, and is indicated as the shaded line. To guide the eye, horizontal dashed lines are included which indicate the 1$\sigma$, 90\%, 2$\sigma$ and 3$\sigma$ \dchisqCPV values assumed using the constant-\dchisq method, with one degree of freedom. The distribution of throws used produced to calculate the \dchisqcrit values shown are given in Figure~\ref{fig:fc_throws_100kt-MW-yr} (Figure~\ref{fig:fc_throws_334kt-MW-yr}) for 100 kt-MW-yr (334 kt-MW-yr).}
  \label{fig:fc_vs_dcp}
\end{figure}
As \deltacp is a cyclical parameter, with physical boundaries at $\pm\pi$, it is interesting to see how the \dchisqcrit values evolve as a function of it. Figure~\ref{fig:fc_vs_dcp} shows the \dchisqcrit as a function of true \deltacp, for an ND+FD analysis with equal FHC and RHC running including the reactor $\theta_{13}$ constraint, for both 100 kt-MW-yr and 334 kt-MW-yr exposures. There is a noticeable, although not large, depression in the \dchisqcrit values at $\deltacp = \pm\pi/2$ for all significance levels considered. This effect is larger at the lower, 100 kt-MW-yr, exposure, and is larger at higher significance levels. It is also clear from Figure~\ref{fig:fc_vs_dcp} that the \dchisqcrit behaviour is very similar at $\deltacp = \pm\pi/2$ as at $\deltacp = 0$. Although the \dchisqcrit values are relevant for CPV sensitivity, this evolution of the \dchisqcrit values with \deltacp will be important for estimating DUNE's \deltacp resolution. Details on the number of toy throws used at each point of Figure~\ref{fig:fc_vs_dcp} are given in the Appendix, and the toy throw distributions used to calculate the \dchisqcrit values and uncertainties are shown for the 100 kt-MW-yr (334 kt-MW-yr) test points in Figure~\ref{fig:fc_throws_100kt-MW-yr} (Figure~\ref{fig:fc_throws_334kt-MW-yr}).

\begin{figure*}[htbp]
  \centering
  \subfloat[24 kt-MW-yr]  {\includegraphics[width=0.33\linewidth]{cpv_throws_withFC_24ktMWyr_NH_th13.png}}
  \subfloat[66 kt-MW-yr]  {\includegraphics[width=0.33\linewidth]{cpv_throws_withFC_66ktMWyr_NH_th13.png}}
  \subfloat[100 kt-MW-yr] {\includegraphics[width=0.33\linewidth]{cpv_throws_withFC_100ktMWyr_NH_th13.png}}\\
  \subfloat[150 kt-MW-yr] {\includegraphics[width=0.33\linewidth]{cpv_throws_withFC_150ktMWyr_NH_th13.png}}
  \subfloat[197 kt-MW-yr] {\includegraphics[width=0.33\linewidth]{cpv_throws_withFC_197ktMWyr_NH_th13.png}}
  \subfloat[334 kt-MW-yr] {\includegraphics[width=0.33\linewidth]{cpv_throws_withFC_334ktMWyr_NH_th13.png}}
  \caption{Fraction of throws for which significance of DUNE's CP-violation test ($\deltacp \neq \{0,\pm\pi\}$) exceeds 1--3$\sigma$, calculated using constant-\dchisq (dashed lines) and \dchisqcrit values calculated using the Feldman-Cousins method (shaded histograms), as a function of the true value of \deltacp. Shown for NO, for a number of different exposures. The number of throws used to make each figure is also shown.}
  \label{fig:cpv_over_time_fc}
\end{figure*}
Having calculated the \dchisqcrit values necessary to achieve correct coverage, it is possible to re-evaluate the CPV sensitivity previously shown assuming a constant \dchisq method in Figure~\ref{fig:cpv_over_time}. Figure~\ref{fig:cpv_over_time_fc} shows the 1, 2 and 3$\sigma$ CPV sensitivities as a function of true \deltacp, with and without the Feldman-Cousins correction (e.g., the \dchisqcrit values given in Figure~\ref{fig:fc_vs_exp} applied). The uncorrected values are the same as in Figure~\ref{fig:cpv_over_time}. In general, the effect of the Feldman-Cousins correction is to reduce the fraction of toy throws that cross each significance threshold, by a maximum of $\sim$10\%, but the exact fraction changes as a function of true \deltacp value and exposure. An exception to this general trend is the 3$\sigma$ behaviour at 24 kt-MW-yr, the lowest exposure shown, where the significance increases. This is due to fall in the 3$\sigma$ \dchisqcrit value towards the lowest exposures observed in Figure~\ref{fig:fc_vs_exp}.

\begin{figure*}[htbp]
  \centering
  \subfloat[$\deltacp = -\pi/2$] {\includegraphics[width=0.8\columnwidth]{{fraction_throws_vs_exp_dcp-0.5_FC}.pdf}}
  \subfloat[50\% of \deltacp values] {\includegraphics[width=0.8\columnwidth]{{fraction_throws_vs_exp_dcprange_0.5_FC}.pdf}}
\caption{Fraction of throws for which the significance of DUNE's CP-violation test ($\deltacp \neq \{0,\pm\pi\}$) exceeds 1--3$\sigma$, both assuming $\deltacp = -\pi/2$ and for 50\% of \deltacp values, calculated using constant-\dchisq (dashed lines) and \dchisqcrit values calculated using the Feldman-Cousins methed (solid lines), as a function of exposure.}
  \label{fig:cpv_vs_exp_fc}
\end{figure*}
Figure~\ref{fig:cpv_vs_exp_fc} shows the fraction of throws which exceed different significance thresholds at the maximal \deltacp violation value of $\deltacp = -\pi/2$, and for 50\% of \deltacp values as a function of exposure, with and without Feldman-Cousins corrections, for 1--3$\sigma$ significance values. It is clear from Figure~\ref{fig:cpv_vs_exp_fc} that the effect of the correction is not large, and $\sim$10\% longer exposures are required for the median expected significance to cross each threshold than without correction, at both $\deltacp = -\pi/2$ and for the 50\% range of \deltacp values.

\FloatBarrier

\section{Mass hierarchy sensitivity}
\label{sec:mh_sens}

\begin{figure}[htbp]
  \centering
  \includegraphics[width=0.98\linewidth, trim={0cm 0cm 0cm 2.3cm}, clip]{mh_two_exps_throws_nh_2019_v4_lowexp.png}
  \includegraphics[width=0.98\linewidth, trim={0cm 0cm 0cm 2.3cm}, clip]{mh_two_exps_throws_ih_2019_v4_lowexp.png}
  \caption{Significance of the DUNE determination of the neutrino mass ordering, as a function of the true value of \deltacp, for 66 ktMWyr (blue) and 100 ktMWyr (orange) exposures. The width of the transparent bands cover 68\% of fits in which random throws are used to simulate statistical variations and select true values of the oscillation and systematic uncertainty parameters, constrained by pre-fit uncertainties. The solid lines show the median sensitivity.}
  \label{fig:mh_nominal}
\end{figure}
Figure~\ref{fig:mh_nominal} shows the significance with which the neutrino mass ordering can be determined in both \dword{no} and \dword{io} as a function of the true value of \deltacp, for both seven and ten year exposures, including the effect of all other oscillation and systematic parameters using the toy throwing method described in Section~\ref{sec:methods}. The characteristic shape results from near degeneracy between matter and \dword{cpv} effects that occurs near $\deltacp=\pi/2$ ($-\deltacp=\pi/2$) for true normal (inverted) ordering. Studies have indicated that special attention must be paid to the statistical interpretation of neutrino mass ordering sensitivities~\cite{Ciuffoli:2013rza,Qian:2012zn,Blennow:2013oma} because the $\Delta\chi^2$ metric does not follow the expected chi-square function for one degree of freedom, so the interpretation of the $\sqrt{\Delta \chi^{2}}$ as the sensitivity is complicated. However, it is clear from Figure~\ref{fig:mh_nominal} that \dword{dune} is able to distinguish the mass ordering for both true \dword{no} and \dword{io}, and using the corrections from, for example, Ref.~\cite{Ciuffoli:2013rza}, DUNE would still achieve 5$\sigma$ significance for the central 68\% of all throws shown in Figure~\ref{fig:mh_nominal}. We note that for both seven and ten years (it was not checked for lower exposures), there were no parameter throws used in generating the plots ($\sim$300,000 each) for which the incorrect mass ordering was preferred.

\begin{figure*}[htbp]
  \centering
  \subfloat[6 ktMWyr]   {\includegraphics[width=0.33\linewidth]{MH_comp_ndfd_6ktMWyr_th13.png}}
  \subfloat[12 ktMWyr]  {\includegraphics[width=0.33\linewidth]{MH_comp_ndfd_12ktMWyr_th13.png}}
  \subfloat[24 ktMWyr]  {\includegraphics[width=0.33\linewidth]{MH_comp_ndfd_24ktMWyr_th13.png}}\\
  \subfloat[66 ktMWyr]  {\includegraphics[width=0.33\linewidth]{MH_comp_ndfd_66ktMWyr_th13.png}}
  \subfloat[100 ktMWyr] {\includegraphics[width=0.33\linewidth]{MH_comp_ndfd_100ktMWyr_th13.png}}
  \subfloat[336 ktMWyr] {\includegraphics[width=0.33\linewidth]{MH_comp_ndfd_336ktMWyr_th13.png}}
  \caption{}
  \label{fig:mh_comp_over_time}
\end{figure*}

\begin{figure*}[htbp]
  \centering
  \subfloat[6 ktMWyr]   {\includegraphics[width=0.33\linewidth]{mh_throws_6ktMWyr_NH_th13.png}}
  \subfloat[12 ktMWyr]  {\includegraphics[width=0.33\linewidth]{mh_throws_12ktMWyr_NH_th13.png}}
  \subfloat[24 ktMWyr]  {\includegraphics[width=0.33\linewidth]{mh_throws_24ktMWyr_NH_th13.png}}\\
  \subfloat[66 ktMWyr]  {\includegraphics[width=0.33\linewidth]{mh_throws_66ktMWyr_NH_th13.png}}
  \subfloat[100 ktMWyr] {\includegraphics[width=0.33\linewidth]{mh_throws_100ktMWyr_NH_th13.png}}
  \subfloat[336 ktMWyr] {\includegraphics[width=0.33\linewidth]{mh_throws_336ktMWyr_NH_th13.png}}
  \caption{}
  \label{fig:mh_nh_over_time}
\end{figure*}

\section{Conclusion}
\label{sec:conclude}

In this work we have presented a detailed exploration of DUNE's sensitivity to CPV and the mass ordering at low exposures. The analysis uses the same framework, flux, cross section and detector models and selections as were used in Ref.~\cite{Abi:2020qib}, which showed the ultimate DUNE sensitivity to CPV, MO and other oscillation parameters, with large statistics samples after long exposures.

\todo{Add comment about run plan optimization...}

The studies presented here demonstrate that a full treatment of DUNE's sensitivity at low exposures supports the conclusions made in Refs.~\cite{Abi:2020qib} and~\cite{Abi:2020evt} using simple Asimov studies. In particular, the median CPV sensitivity is $\sim$3$\sigma$ for $\deltacp = \pm\pi/2$ after approximately a 100 ktMWyr FD exposure. We also explore the variations in the expected sensitivity around the median value. Additionally, we show that the CPV sensitivity is not significantly degraded when Feldman-Cousins corrections are included, leading to $\sim$10\% longer exposures to reach a given significance level. Crucially, we find that after an initial low-exposure rise, the Feldman-Cousins \dchisqcrit do not change as a function of exposure, as has been observed by the T2K experiment~\cite{Abe:2021gky}.

We have also shown that strong statements on the mass ordering can be expected with very short exposures of $\sim$12 ktMWyr, which supports the results shown in Refs.~\cite{Abi:2020qib} and~\cite{Abi:2020evt} with a more complete treatment of the systematic uncertainty.

We note that although the analysis used here makes no assumptions about the FD staging scenario, and results are given as a function of exposure only, the results are dependent on having a performant ND complex from the start of the experiment. In particular, the low-exposures necessary to make world-leading statements about the mass hierarchy can only be given with confidence with ND samples included in the fit. We note also that additional samples of events from other detectors in the DUNE ND complex are not explicitly included in this analysis, but there is an assumption that we will be able to control the uncertainties to the level used in the analysis, and it should be understood that that implicitly relies on having a highly capable ND.

\begin{acknowledgements}
\todo{Update! This version is from the first LBL paper}

This document was prepared by the DUNE collaboration using the
resources of the Fermi National Accelerator Laboratory
(Fermilab), a U.S. Department of Energy, Office of Science,
HEP User Facility. Fermilab is managed by Fermi Research Alliance,
LLC (FRA), acting under Contract No. DE-AC02-07CH11359.
This work was supported by
CNPq, FAPERJ, FAPEG and FAPESP,              Brazil;
CFI, IPP and NSERC,                          Canada;
CERN;
M\v{S}MT,                                        Czech Republic;
ERDF, H2020-EU and MSCA,                     European Union;
CNRS/IN2P3 and CEA,                          France;
INFN,                                        Italy;
FCT,                                         Portugal;
NRF,                                         South Korea;
CAM, Fundaci\'{o}n ``La Caixa'' and MICINN,  Spain;
SERI and SNSF,                               Switzerland;
T\"UB\.ITAK,                                 Turkey;
The Royal Society and UKRI/STFC,             United Kingdom;
DOE and NSF,                                 United States of America.
This research used resources of the
National Energy Research Scientific Computing Center (NERSC),
a U.S. Department of Energy Office of Science User Facility
operated under Contract No. DE-AC02-05CH11231.
\end{acknowledgements}

\bibliographystyle{utphys}
\bibliography{tdr-citedb}

\appendix*
\section{Feldman-Cousins throw distributions}\label{sec:fc_appendix}

\begin{figure*}[htbp]
  \centering
  \subfloat[24 ktMWyr] {\includegraphics[width=0.33\linewidth]{nh_FC_ndfd_24ktMWyr_dcp0.png}}
  \subfloat[66 ktMWyr] {\includegraphics[width=0.33\linewidth]{nh_FC_ndfd_66ktMWyr_dcp0.png}}
  \subfloat[100 ktMWyr]{\includegraphics[width=0.33\linewidth]{nh_FC_ndfd_100ktMWyr_dcp0.png}}\\
  \subfloat[150 ktMWyr]{\includegraphics[width=0.33\linewidth]{nh_FC_ndfd_150ktMWyr_dcp0.png}}
  \subfloat[197 ktMWyr]{\includegraphics[width=0.33\linewidth]{nh_FC_ndfd_197ktMWyr_dcp0.png}}
  \subfloat[336 ktMWyr]{\includegraphics[width=0.33\linewidth]{nh_FC_ndfd_334ktMWyr_dcp0.png}}\\
  \subfloat[500 ktMWyr]{\includegraphics[width=0.33\linewidth]{nh_FC_ndfd_500ktMWyr_dcp0.png}}
  \subfloat[646 ktMWyr]{\includegraphics[width=0.33\linewidth]{nh_FC_ndfd_646ktMWyr_dcp0.png}}
  \subfloat[936 ktMWyr]{\includegraphics[width=0.33\linewidth]{nh_FC_ndfd_936ktMWyr_dcp0.png}}
  \caption[]{}
  \label{fig:fc_throws_exp}
\end{figure*}

\begin{figure*}[htbp]
  \centering
  \subfloat[$\deltacp/\pi = -1$]    {\includegraphics[width=0.33\linewidth]{{nh_FC_ndfd_100ktMWyr_dcp-1}.png}}
  \subfloat[$\deltacp/\pi = -0.75$] {\includegraphics[width=0.33\linewidth]{{nh_FC_ndfd_100ktMWyr_dcp-0.75}.png}}
  \subfloat[$\deltacp/\pi = -0.5$]  {\includegraphics[width=0.33\linewidth]{{nh_FC_ndfd_100ktMWyr_dcp-0.5}.png}}\\
  \subfloat[$\deltacp/\pi = -0.25$] {\includegraphics[width=0.33\linewidth]{{nh_FC_ndfd_100ktMWyr_dcp-0.25}.png}}
  \subfloat[$\deltacp/\pi = 0$]     {\includegraphics[width=0.33\linewidth]{{nh_FC_ndfd_100ktMWyr_dcp0}.png}}
  \subfloat[$\deltacp/\pi = 0.25$]  {\includegraphics[width=0.33\linewidth]{{nh_FC_ndfd_100ktMWyr_dcp0.25}.png}}\\
  \subfloat[$\deltacp/\pi = 0.5$]   {\includegraphics[width=0.33\linewidth]{{nh_FC_ndfd_100ktMWyr_dcp0.5}.png}}
  \subfloat[$\deltacp/\pi = 0.75$]  {\includegraphics[width=0.33\linewidth]{{nh_FC_ndfd_100ktMWyr_dcp0.75}.png}}
  \subfloat[$\deltacp/\pi = 1$]     {\includegraphics[width=0.33\linewidth]{{nh_FC_ndfd_100ktMWyr_dcp1}.png}}
  \caption[]{}
  \label{fig:fc_throws_100ktMWyr}
\end{figure*}

\begin{figure*}[htbp]
  \centering
  \subfloat[$\deltacp/\pi = -1$]    {\includegraphics[width=0.33\linewidth]{{nh_FC_ndfd_334ktMWyr_dcp-1}.png}}
  \subfloat[$\deltacp/\pi = -0.75$] {\includegraphics[width=0.33\linewidth]{{nh_FC_ndfd_334ktMWyr_dcp-0.75}.png}}
  \subfloat[$\deltacp/\pi = -0.5$]  {\includegraphics[width=0.33\linewidth]{{nh_FC_ndfd_334ktMWyr_dcp-0.5}.png}}\\
  \subfloat[$\deltacp/\pi = -0.25$] {\includegraphics[width=0.33\linewidth]{{nh_FC_ndfd_334ktMWyr_dcp-0.25}.png}}
  \subfloat[$\deltacp/\pi = 0$]     {\includegraphics[width=0.33\linewidth]{{nh_FC_ndfd_334ktMWyr_dcp0}.png}}
  \subfloat[$\deltacp/\pi = 0.25$]  {\includegraphics[width=0.33\linewidth]{{nh_FC_ndfd_334ktMWyr_dcp0.25}.png}}\\
  \subfloat[$\deltacp/\pi = 0.5$]   {\includegraphics[width=0.33\linewidth]{{nh_FC_ndfd_334ktMWyr_dcp0.5}.png}}
  \subfloat[$\deltacp/\pi = 0.75$]  {\includegraphics[width=0.33\linewidth]{{nh_FC_ndfd_334ktMWyr_dcp0.75}.png}}
  \subfloat[$\deltacp/\pi = 1$]     {\includegraphics[width=0.33\linewidth]{{nh_FC_ndfd_334ktMWyr_dcp1}.png}}
  \caption[]{}
  \label{fig:fc_throws_334ktMWyr}
\end{figure*}


\end{document}
